%% LyX 1.4.3 created this file.  For more info, see http://www.lyx.org/.
%% Do not edit unless you really know what you are doing.
\documentclass[11pt,english,a4paper,onecolumn]{article}
\usepackage[T1]{fontenc}

\makeatletter

%%%%%%%%%%%%%%%%%%%%%%%%%%%%%% LyX specific LaTeX commands.
%% Because html converters don't know tabularnewline
\providecommand{\tabularnewline}{\\}

%%%%%%%%%%%%%%%%%%%%%%%%%%%%%% User specified LaTeX commands.
%% LyX 1.5.3 created this file.  For more info, see http://www.lyx.org/.
%% Do not edit unless you really know what you are doing.



\makeatletter

\makeatother

\usepackage{babel}
\makeatother
\begin{document}

\title{{\Large Specification and Implementation of a Python to SAGA Language
Binding}{\normalsize }\\
 {\normalsize Computer Science Master Thesis} {\Large }}


\author{P.F.A. van Zoolingen{\normalsize }\\
 {\normalsize 1284657, pzn400@few.vu.nl}}

\maketitle

\section{Abstract}

This thesis describes the effort to create a Python language binding
for SAGA, the Simple Api for Grid Applications, and its efforts to
implement the language binding on top of the Java reference implementation.
Using this functionality, Python programmers can use SAGA to program
grid aware applications and to shield themselves from all the details
which come with grids. This adds to the adopting of SAGA in a world
of with many different APIs and middleware layers.


\section{Introduction}


\subsection{Introduction English}

SAGA stands for Simple API for Grid Applications and was developed
to offer users a simple tool to program applications for of heterogeneous
grids. These grids often consist of different types of hardware, operating
systems and middleware software and are hard to program. This API
is developed to be independent of any underlying hardware or software
and to shield the user from all the details and to let him focus on
programming grid aware applications.

To use this API, the functionality described by SAGA has to be implemented
by another piece of software. This is called the Saga reference implementation.
At this point there are three different reference implementations
which are programmed in the programming languages Java and C++. Which
in a general sense means that only Java and C++ applications can easily
use SAGA to access the grid in an easy way. This thesis describes
the efforts to add another language to that list, namely Python. Python
is partially supported by the C++ reference implementation, but there
is no specific Python language binding available which describes the
SAGA functionality in Python specific way, independent of the chosen
reference implementation. During the course of my master project I
have specified the Python language binding and have implemented the
language binding for the Java reference implementation.

This thesis is divided into different pieces. First I will describe
and explain what SAGA is, where it comes from and how it is implemented.
Then I will continue with a description of Python and a special implementation
of Python called Jython, followed by the specification of the language
binding and its implementation. After that I will conclude with the
testing, discussion, future work and the conclusion.


\subsection{Introduction Dutch}

SAGA staat voor Simpele API voor Grid Applicaties en is ontwikkeld
als een simpel stuk gereedschap om het programmeren op hetrogene grids
te vergemakkelijken. Dit soort grids bestaan vaak uit verschillende
hardware, besturingssystemen en middleware software en het is vaak
lastig om hier grid applicaties voor te programmeren. De API is ontwikkeld
als een aanspreekpunt voor de grid, onafhankelijk van de onderliggende
hard- en software. Tevens houdt de API de programmeur weg bij de onderliggende
details, die per platform zeer kunnen verschillen, en laat de programmeur
zich bezighouden met het programmeren van een hogere abstractie niveau
voor zijn applicatie.

Om de API te kunnen gebruiken moet de functionaliteit beschreven door
SAGA ge\"implementeerd worden door andere software. De algemene term
voor deze software is de Referentie Implementatie. Momenteel bestaan
drie verschillende referentie implementaties die gemaakt zijn in de
programmeertalen Java en C++. Dit houdt globaal in de talen C++ en
Java redelijk eenvoudig is om een applicatie te programmeren en de
achterliggende referentie implementatie en daarmee het grid te gebruiken.
Deze Master these beschrijft de inspanning om daar een derde taal
aan toe te voegen, namelijk Python. Python word op dit moment al deels
ondersteund door de C++ referentie implementatie maar er is geen algemene
specificatie, ofwel een 'language binding', beschikbaar die de SAGA
functionaliteit beschrijft in een Python specifieke manier, onafhankelijk
van de onderliggende referentie implementatie. Tijdens mijn Master
project heb ik een Python language binding voor SAGA gespecificeerd
en daarnaast ge\"implementeerd bovenop de Java referentie implementatie.
Deze implementatie zou in theorie moeten werken op elke Java referentie
implementatie. 

Deze these is onderverdeeld in verschillende delen. Eerst zal ik uitleggen
wat SAGA is, waar het vandaan komt en hoe het geimplementeerd is.
Ik zal doorgaan met een beschrijving van Python en een specifieke
implementatie van Python genaamd Jython, gevolgd door de specificatie
van de language binding en zijn implementatie. Ik zal besluiten met
het test gedeelte, de discussie, het vooruitzicht en de conclusie.


\section{SAGA}


\subsection{SAGA}

As mentioned in the introduction, SAGA stands for Simple API for Grid
Applications. SAGA came as an idea in a time when multiple middleware
projects and applications groups were looking for higher-level programming
abstractions and the simplification of programming for the grid \cite{SAGA}.
A SAGA research group (SAGA-RG) was founded within the Global Grid
Forum (GGF), which later merged into the Open Grid Forum (OGF). The
aim of the group has been to identify a set of basic grid operations
and derive a simple consistent API, which eases the development of
applications that make use of grid technologies. 


\subsection{Use Cases}

To poll the needs of users, the research group sent out a call for
use cases. In these use cases users described many subjects such as
their application area, the desired look and feel of the API and resource,
performance, security considerations. The majority of use cases which
were returned came from scientific users \cite{UseCases}, which probably
biased SAGA in the analysis of the use cases towards scientific applications.
In this analysis, the research group focused on the identification
of the SAGA API scope, on the level of abstraction wanted and needed
by the application programmers. Non-functional requirements and requirements
from other projects, such as GAT \cite{GAT} and CoG \cite{CoG} were
also considered.


\subsection{The API}

With 24 use cases available, the requirements from the users could
be distilled \cite{ReqAnalysis}. A design team was formed to use
these requirements to design and develop the API. A few general design
issues were considered and agree upon.

\begin{itemize}
\item The API would be designed and developed in a object-oriented manner
using a language-neutral representation.
\item Asynchronicity is prefered to be handled by a polling mechanism rather
than a subscribe/listen mechanism to make implementations in non-multithreaded
environment easier.
\item Grid subsystems should be specified independent from each other to
allow independent development and implementation of parts of the API.
\item Sessions and Security should be an essential part of SAGA since applications
often run accros administrative domains and security boundaries.
\item Data Management like remote file access and replica catalogs are also
an important part of grid applications and should therefore also be
in SAGA.
\item Remote jobs and asynchronous operations are a common requirement for
grid applications and must be supported in the API.
\item SAGA should support interprocess communication as a stream concept,
similar to BSD sockets.
\end{itemize}
Ultimately, the purpose of SAGA is to provide an simple API that can
be used with much less effort compared to the vanilla interfaces of
existing grid middleware. A guiding principle for achieving this simplicity
is the 80/20 rule: serve 80\% of the use cases with 20\% of the effort
needed for serving 100 \% of all possible requirements and to provide
a standardized, common interface across various grid middleware systems
and their versions. 

After determining the requirements, a so-called SAGA Strawman API
was developed to accomodate the requirements and after some iterations
SAGA was released in January 2008 \cite{GFD.90}. SAGA is described
in a document called \emph{{}``A Simple API for Grid Applications
(SAGA)''} or GFD.90. GFD.90 specifies the core components of SAGA
has formed the basis of specification of the Python language-binding,
which will be explained in section \ref{sec:Specification}, and the
reference implementations. It is aimed at implementors of the API
and not directly at end-users. The end-users can use the documentation
and specific language-bindings given by the implementors of the reference
implentations. The document holds much information about the complete
SAGA project but what is interesting for building a language-binding
is stated in section 3 and 4 of GFD.90. These sections consist of
a number of interface and class specifications, which are divided
in multiple packages. 

The packages in Look \& Feel part consist of:

%
\begin{table}
\begin{tabular}{|c|c|c|c|}
\hline 
Error&
Object&
URL&
Buffer\tabularnewline
\hline
\hline 
Session&
Context&
Permission&
Attributes\tabularnewline
\hline 
&
Task&
Monitoring&
\tabularnewline
\hline
\end{tabular}


\caption{\label{tab: LF}Look and Feel packages}
\end{table}


\begin{tabular}{cccc}
Errors&
Object&
Url&
Buffer\tabularnewline
Session&
Context &
Permission &
Attributes \tabularnewline
&
Task&
\multicolumn{1}{c}{Monitoring}&
\tabularnewline
\end{tabular}



The API packages consist of:

\begin{tabular}{ccc}
Job&
Namespace&
File\tabularnewline
Replica&
Streams&
RPC\tabularnewline
\end{tabular}


\subsection{OMII-UK}

\cite{OMII-UK} 


\subsection{OGF}


\subsection{Previous Work}


\subsection{Java SAGA Reference Implementation}


\subsection{C++ SAGA Reference Implementation}


\subsection{Adaptors - JavaGAT}


\subsection{Adaptors}


\section{Python}


\subsection{Explain Python}


\subsection{Special Syntax Features}


\subsubsection{Named Parameters}


\subsubsection{Overloading}


\subsubsection{Dynamic Typing}


\subsubsection{Extending Python}


\subsection{Jython}


\subsection{Difference Jython - Python}


\section{Specification\label{sec:Specification}}


\subsection{All Modules Specified with Design Decisions}

\newpage{}


\section{Implementation}


\subsection{Previous work // Solutions Tested}


\subsection{Delegate Object}


\subsection{Convert Exception}


\subsection{Tasks}


\subsection{Inheritance}


\subsection{Get\_id()}


\subsection{All Modules Specified with Design Decisions}


\section{Testing}


\subsection{Test Environment}


\subsection{scripts}


\subsection{Bugs}


\subsection{Repository}

\newpage{}


\section{Discussion}


\section{Future Work}


\subsection{Synchronize specification with LSU}


\subsection{Extending the API}

Extention packages


\subsection{Special Python operators}


\subsection{Updating Implementation with new Reference Implementations}


\section{List of Frequently Used Terms}

\begin{itemize}
\item API: Application Programming Interface. A set of variables, methods
and classes that is offered by an operating system or software library
to support requests made by computer programs. 
\item Grid: A collection of interconnected computers consisting of different
hardware, placed in different locations and belonging to different
organizations. 
\item Grid Aware: Applications which are grid aware are designed to run
on a grid and use the possibilities of the grid, such as distributing
workload between available nodes in the grid. 
\item Language Binding: An API in a specific programming language which
gives access to a library or service
\item Reference Implementation: Software which implements the functionality
described by SAGA. New applications can link to this software and
call methods described in SAGA to use the grid. 
\item SAGA: Simple API for Grid Applications 
\end{itemize}
\newpage{}


\section{Conclusion}


\section{Bibliography}

\bibliographystyle{plain}
\addcontentsline{toc}{section}{\refname}\bibliography{Thesis}



\section{Acknowledgments}


\section{Appendix A: Installing and Running JySaga}


\section{Appendix B: Proof of a Higgs-boson}
\end{document}
