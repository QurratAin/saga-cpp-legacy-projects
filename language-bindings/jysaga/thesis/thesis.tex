%% LyX 1.5.3 created this file.  For more info, see http://www.lyx.org/.
%% Do not edit unless you really know what you are doing.
\documentclass[11pt,english,a4paper,onecolumn]{article}
\usepackage[latin9]{inputenc}

\makeatletter
\usepackage{babel}
\makeatother

\begin{document}

\title{{\Large Specification and Implementation of a Python to SAGA Language
Binding}{\normalsize }\\
{\normalsize Computer Science Master Thesis}{\Large{} }}


\author{{\normalsize P.F.A. van Zoolingen}\\
{\normalsize 1284657, pzn400@few.vu.nl}}

\maketitle

\section{Abstract}


\subsection{Abstract English}

This thesis describes the effort to create a Python language binding
for SAGA, the Simple Api for Grid Applications, and its efforts to
implement the language binding on top of the Java reference implementation.
Using this functionality, Python programmers can use SAGA to program
grid aware applications and to shield themselves from all the details
which come with grids. This adds to the adopting of SAGA in a world
of with many different APIs and middleware layers.


\subsection{Abstract Nederlands}


\subsection{List of Frequently Used Terms}

\begin{itemize}
\item API: Application Programming Interface. A set of variables, methods
and classes that is offered by an operating system or software library
to support requests made by computer programs.
\item Grid: A collection of interconnected computers consisting of different
hardware, placed in different locations and belonging to different
organizations.
\item Grid Aware: Applications which are grid aware are designed to run
on a grid and use the possibilities of the grid, such as distributing
workload between available nodes in the grid.
\item Reference Implementation: Software which implements the functionality
described by SAGA. New applications can link to this software and
call methods described in SAGA to use the grid.
\item SAGA: Simple API for Grid Applications
\end{itemize}

\section{Introduction}

SAGA stands for Simple API for Grid Applications and was developed
to offer users a simple tool to program applications for of heterogeneous
grids. These grids often consist of different types of hardware, operating
systems and middleware software and are hard to program. This API
is developed to be independent of any underlying hardware or software
and to shield the user from all the details and to let him focus on
programming grid aware applications.

To use this API, the functionality described by SAGA has to be implemented
by another piece of software. This is called the Saga reference implementation.
At this point there are two different reference implementations which
are programmed in the programming languages Java and C++. Which in
a general sense means that only Java and C++ applications can easily
use SAGA to access the grid in an easy way. This thesis describes
the efforts to add another language to that list, namely Python. Python
is partially supported by the C++ reference implementation, but there
is no specific Python language binding available which describes the
SAGA functionality in Python specific way independent of the chosen
reference implementation. During the course of my master project I
have specified the Python language binding and have implemented the
language binding for the Java reference implementation. 

This thesis is divided into different pieces. First I will describe
and explain what SAGA is, where it comes from and how it is implemented.
Then I will continue with a description of Python and a special implementation
of Python called Jython. Followed by the specification of the language
binding and its implementation. After that, the testing, discussion,
future work and the conclusion will be described.


\section{SAGA}


\subsection{SAGA}


\subsection{GFD.90}


\subsection{Use Cases}


\subsection{OMII}


\subsection{OGF}


\subsection{Previous Work}


\subsection{SAGA Reference Implementations}


\subsection{Java SAGA Reference Implementation}


\subsection{C++ SAGA Reference Implementation}


\subsection{Adapters - JavaGAT}


\subsection{Adapters}


\section{Python}


\subsection{Explain Python}


\subsection{Special Syntax Features}


\subsubsection{Named Parameters}


\subsubsection{Overloading}


\subsubsection{Dynamic Typing}


\subsubsection{Extending Python}


\subsection{Jython}


\subsection{Difference Jython - Python}


\section{Specification}


\subsection{All Modules Specified with Design Decisions}

\newpage{}


\section{Implementation}


\subsection{Previous work // Solutions Tested}


\subsection{Delegate Object}


\subsection{Convert Exception}


\subsection{Tasks}


\subsection{Inheritance}


\subsection{Get\_id()}


\subsection{All Modules Specified with Design Decisions}


\section{Testing}


\section{Test Environment}


\subsection{scripts}


\subsection{Bugs}


\subsection{Repository}

\pagebreak[0]


\section{Discussion}


\section{Future Work}


\subsection{Synchronize specification with LSU}


\subsection{Extending the API}


\subsection{Special Python operators}


\subsection{Updating Implementation with new Reference Implementations}

\pagebreak[0]


\section{Conclusion}

\pagebreak[0]


\section{Bibliography}


\section{Acknowledgments}

\pagebreak[0] 


\section{Appendix A: Source Code}


\section{Appendix B: Proof of a Higgs-boson}
\end{document}
