\documentclass[a4paper,10pt]{article}
\usepackage[utf8]{inputenc}
\usepackage{graphicx}
\usepackage{url}
\usepackage{float}
\usepackage{times}
\usepackage{multirow}
\usepackage{listings}
\usepackage{times}
\usepackage{paralist}
\usepackage{epsfig}
\usepackage{subfigure}
\usepackage[hypertex]{hyperref}
\usepackage{subfigure}
\usepackage{color}
\usepackage{xspace}

%\documentclass{rspublic}

\usepackage{ifpdf}

\newcommand{\I}[1]{\textit{#1}}
\newcommand{\B}[1]{\textbf{#1}}
\newcommand{\BI}[1]{\textbf{\textit{#1}}}
\newcommand{\T}[1]{\texttt{#1}}

\newcommand{\sagaspec}{\textit{SAGA}\xspace}
\newcommand{\sagaimpl}{\textit{SAGA}\xspace}

\newcommand{\spec}{\sagaspec}
\newcommand{\impl}{\sagaimpl}

\setlength\topmargin{0in}
\setlength\headheight{0in}
\setlength\headsep{0in}
\setlength\textheight{9.5in}
\setlength\textwidth{6.5in}
\setlength\oddsidemargin{0in}
\setlength\evensidemargin{0in}
\setlength\parindent{0.1in}
\setlength\parskip{0.25em}


\ifpdf
 \DeclareGraphicsExtensions{.pdf, .jpg}
\else
 \DeclareGraphicsExtensions{.eps, .ps}
\fi

\newcommand{\note}[1]{ {\textcolor{red} { ***NOTE: #1 }}}

\newif\ifdraft
\drafttrue

\ifdraft
\newcommand{\amnote}[1]{   {\textcolor{magenta} { ***Andre:    #1 }}}
\newcommand{\jhanote}[1]{  {\textcolor{red}     { ***Shantenu: #1 }}}
\else
\newcommand{\amnote}[1]{}
\newcommand{\jhanote}[1]{}
\fi

\begin{document}

 \title{ \Large \vspace{-3.5em} Mephisto Documentation }
 
 \author{ SAGA-Group}
 \date{05-04-2011}
 \maketitle
 

% \jhanote{Remember in addition to serving as an abstract, this will
%   serve as a summary of what will go to the 3 editors of the journals
%   that we are considering publishing a full paper in. Thus some more
%   information/discussion on what the underlying problem and context
%   will be about.}


% \jhanote{Once we have defined / introduced SAGA, we should probably
%   have 3 subsections -- interface, library and adatptors/backends?}

\section*{Introduction}
Mephisto is a perl script that supports SAGA (Simplified API for Grid Applications) installations. SAGA is an API that provides the basic functionality required to build distributed applications, tools and frameworks so as to be independent of the details of the underlying infrastructure ~\cite{}. More information on SAGA can be found on the website:
\begin{verbatim}
  http://www.saga.cct.lsu.edu/
\end{verbatim}
SAGA installations require to install various dependency packages. Mephisto is a tool that automates the installation of SAGA, installing all its dependencies specified for SAGA and SAGA adaptors too.  The version number of all the packages and the type of packages can be specified by the user at the command line. The description of all the command line options available are described below in later sections.
\section*{Command line options}
\section*{Example Command line options explained}
\subsection*{1. How to test a SAGA Core Pre-Release with a specific Boost version (for developers)}
\section*{Example of a Screen Shot after execution}
%\subsection*{In depth code analysis}
\section*{Possible errors} 



% 
% \bibliographystyle{IEEEtran} 
% \bibliography{saga}


\end{document}

