\documentclass[a4paper,10pt]{article}
\usepackage[utf8]{inputenc}
\usepackage{graphicx}
\usepackage{url}
\usepackage{float}
\usepackage{times}
\usepackage{multirow}
\usepackage{listings}
\usepackage{times}
\usepackage{paralist}
\usepackage{epsfig}
\usepackage{subfigure}
\usepackage[hypertex]{hyperref}
\usepackage{subfigure}
\usepackage{color}
\usepackage{xspace}

%\documentclass{rspublic}

\usepackage{ifpdf}

\newcommand{\I}[1]{\textit{#1}}
\newcommand{\B}[1]{\textbf{#1}}
\newcommand{\BI}[1]{\textbf{\textit{#1}}}
\newcommand{\T}[1]{\texttt{#1}}

\newcommand{\sagaspec}{\textit{SAGA}\xspace}
\newcommand{\sagaimpl}{\textit{SAGA}\xspace}

\newcommand{\spec}{\sagaspec}
\newcommand{\impl}{\sagaimpl}

\setlength\topmargin{0in}
\setlength\headheight{0in}
\setlength\headsep{0in}
\setlength\textheight{9.5in}
\setlength\textwidth{6.5in}
\setlength\oddsidemargin{0in}
\setlength\evensidemargin{0in}
\setlength\parindent{0.1in}
\setlength\parskip{0.25em}


\ifpdf
 \DeclareGraphicsExtensions{.pdf, .jpg}
\else
 \DeclareGraphicsExtensions{.eps, .ps}
\fi

\newcommand{\note}[1]{ {\textcolor{red} { ***NOTE: #1 }}}

\newif\ifdraft
\drafttrue

\ifdraft
\newcommand{\amnote}[1]{   {\textcolor{magenta} { ***Andre:    #1 }}}
\newcommand{\jhanote}[1]{  {\textcolor{red}     { ***Shantenu: #1 }}}
\else
\newcommand{\amnote}[1]{}
\newcommand{\jhanote}[1]{}
\fi

\begin{document}

 \title{ \Large \vspace{-3.5em} Mephisto Documentation }
 
 \author{ SAGA-Group}
 \date{05-04-2011}
 \maketitle
 

% \jhanote{Remember in addition to serving as an abstract, this will
%   serve as a summary of what will go to the 3 editors of the journals
%   that we are considering publishing a full paper in. Thus some more
%   information/discussion on what the underlying problem and context
%   will be about.}


% \jhanote{Once we have defined / introduced SAGA, we should probably
%   have 3 subsections -- interface, library and adatptors/backends?}

\section*{Introduction}
Mephisto is a perl script that supports SAGA (Simplified API for Grid Applications) 
installations. SAGA is an API that provides the basic functionality required to build 
distributed applications, tools and frameworks so as to be independent of the details 
of the underlying infrastructure ~\cite{}. More information on SAGA can be found on 
the website:
\begin{verbatim}
  http://www.saga.cct.lsu.edu/
\end{verbatim}
SAGA installations require to install various dependency packages. 
Mephisto is a tool that automates the installation of SAGA, installing 
all its dependencies specified for SAGA and SAGA adaptors too.  
The version number of all the packages and the type of packages can 
be specified by the user at the command line. The description 
of all the command line options available are described 
in the next section.\\*
Example of the command line statement  
\begin{verbatim} perl mephisto install --target-dir=/home/user/saga --tmp-dir=/home/user/tmp-saga \end{verbatim}
Other examples with different command line options are explained in later section. \\*
\section*{Command line options}
This section discusses all the various options available at 
users disclosure. By default if no options are specified, mephisto 
pulls up the packages listed on
\begin{verbatim} 
http://static.saga.cct.lsu.edu/mephisto/repository/latest/INDEX. 
\end{verbatim}
Expanded version of the Command to run mephisto
\begin{verbatim} 
perl mephisto install/test --target-dir=/path/to/folder --tmp-dir=/path/to/folder 
--with-saga-version= --with-boost-version= --with-globus-version= 
--repository= --with-adaptors= 
\end{verbatim} 
\emph{install} \\*
 This can be used to install a working copy of SAGA and all its 
 dependencies specified. \\*
\emph{test} \\*
 This option not only installs SAGA but also runs the command 
 'make check' to check the installation. It creates a log file displaying 
 the test results. \\*
\emph{target-dir} \\*
 This option is used to specify the path you need to install SAGA and all 
 packages libraries. By default, it creates a folder at tmp/meph\_inst.\\*
\emph{tmp-dir} \\*
 This option is used to specify the path for a temporary folder to 
 download all the packages and create log files. By default, it 
 creates a folder at /tmp/meph\_tmp\\*
\emph{repository} \\*
 Can be used to choose different repositories to install SAGA from. 
 Two repositories available as of now are 'latest' (default) and svn\_trunk. 
 Recommended svn\_trunk for the most recent SAGA version. \\*
\emph{with-saga-version}\\*
 Choose saga versions available.\\*
\emph{with-boost-version}\\*
Choose boost versions available $>$=1.40.0\\*
\emph{with-globus-version} \\*
 Choose globus version available $>$=4.0. This option would automatically 
 invoke installations of X509, globus adaptors too. \\*
\emph{with-adaptors}\\*
 Mention any adaptors to install that would invoke globus installations along with 
 saga-adaptors too. \\*

\section*{Example Command line options}
\emph{NOTE: All commands are explained individually here. They can be used interchangeably.
Choice has to be made been 'install' and 'test'}
\subsection*{- Basic Default command}
\begin{verbatim}
perl mephisto install 
\end{verbatim}
This would install SAGA. All the default options will be loaded as no
option is chosen. As of now the default options are (can change):
Repository: http://static.saga.cct.lsu.edu/mephisto/repository/latest/
SAGA= 1.5.2;
Boost= 1.40.0;
Python=2.6.2;
Post-gre-sql=8.4.1;
Sqlite=3.6.18;
SAGA-PYTHON;
SAGA-ADAPTORS-X509;
SAGA-ADAPTORS-SSH.
All the packages will be stored in temporary folders created by the script.
\subsection*{- Command with paths to folders}
\begin{verbatim}
perl mephisto install --target-dir=/path/to/any/desired directory/ 
--tmp-dir=/path/to/temp/folder 
\end{verbatim}
This will install all packages in the folder specified. 
\subsection*{- Install SAGA from svn\_trunk repository}
\begin{verbatim}
perl mephisto install --target-dir=/path/to/any/desired directory/ 
--tmp-dir=/path/to/temp/folder --repository=svn\_trunk
\end{verbatim}
Choose snv\_trunk repository to install SAGA. This will have the most
recent and tested working SAGA version. 
\subsection*{- Install SAGA will all saga, globus, boost versions specified}
\begin{verbatim}
perl mephisto install --target-dir=/path/to/any/desired directory/ 
--tmp-dir=/path/to/temp/folder --with-saga-version=1.5.3 
--with-boost-version=1.44.0 --with-globus-version=5.0.2
(or)
perl mephisto install --target-dir=/path/to/any/desired directory/ 
--tmp-dir=/path/to/temp/folder --with-saga=1.5.3 
--with-boost=1.44 --with-globus=5.0.2
\end{verbatim}
To install SAGA with different boost, SAGA and globus versions.
Given here are two ways of defining the versions. Using all the three 
is not compulsory and any combination can be used. Any version 
not mentioned will infer default options. 
\subsection*{- Globus installations}
\begin{verbatim}
perl mephisto install --target-dir=/path/to/any/desired directory/ 
--tmp-dir=/path/to/temp/folder --with-globus-version=5.0.2
\end{verbatim}
Specifying any globus installations will also invoke globus adaptors
installations as required and necessary. 
\subsection*{- Install SAGA with Globus adaptors}
\begin{verbatim}
perl mephisto install --target-dir=/path/to/any/desired directory/ 
--tmp-dir=/path/to/temp/folder --with-adaptors=globus
\end{verbatim}
To install globus adaptors. Automatically invokes globus installations which 
is required for adaptors. Also automates x509 adaptors too. 
\subsection*{- How to install a SAGA Core Pre-Release with a specific Boost version (for developers)}
\begin{verbatim}
perl mephisto install --target-dir=/path/to/any/desired directory/ 
--tmp-dir=/path/to/temp/folder --with-saga=1.5.3-pre
\end{verbatim}
For developers, this would install the pre-release version from the given 
link shared in the email with boost and all other dependencies creating log 
files necessary for debugging. 
\subsection*{- Install and proceed to TEST the installations}
\begin{verbatim}
perl mephisto test --target-dir=/path/to/any/desired directory/ 
--tmp-dir=/path/to/temp/folder --with-saga-version=1.5.3 
--with-boost-version=1.44.0 --with-globus-version=5.0.2
\end{verbatim}
Choose 'test' option to first install SAGA and its dependencies and continue
to perform 'make check' operation and create a log file. 
\section*{Example of a Screen Shot after execution}
\section*{Possible errors} 



% 
% \bibliographystyle{IEEEtran} 
% \bibliography{saga}


\end{document}

