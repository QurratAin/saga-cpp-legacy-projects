\documentclass[a4paper,10pt]{article}
\usepackage[utf8]{inputenc}
\usepackage{graphicx}
\usepackage{url}
\usepackage{float}
\usepackage{times}
\usepackage{multirow}
\usepackage{listings}
\usepackage{times}
\usepackage{paralist}
\usepackage{epsfig}
\usepackage{subfigure}
\usepackage[hypertex]{hyperref}
\usepackage{subfigure}
\usepackage{color}
\usepackage{xspace}
\usepackage{multicol}
%\documentclass{rspublic}

\usepackage{ifpdf}

\newcommand{\I}[1]{\textit{#1}}
\newcommand{\B}[1]{\textbf{#1}}
\newcommand{\BI}[1]{\textbf{\textit{#1}}}
\newcommand{\T}[1]{\texttt{#1}}

\newcommand{\sagaspec}{\textit{SAGA}\xspace}
\newcommand{\sagaimpl}{\textit{SAGA}\xspace}

\newcommand{\spec}{\sagaspec}
\newcommand{\impl}{\sagaimpl}

\setlength\topmargin{0in}
\setlength\headheight{0in}
\setlength\headsep{0in}
\setlength\textheight{9.5in}
\setlength\textwidth{6.5in}
\setlength\oddsidemargin{0in}
\setlength\evensidemargin{0in}
\setlength\parindent{0.1in}
\setlength\parskip{0.25em}


\ifpdf
 \DeclareGraphicsExtensions{.pdf, .jpg,.jpeg}
\else
 \DeclareGraphicsExtensions{.eps, .ps}
\fi

\newcommand{\note}[1]{ {\textcolor{red} { ***NOTE: #1 }}}

\newif\ifdraft
\drafttrue

\ifdraft
\newcommand{\amnote}[1]{   {\textcolor{magenta} { ***Andre:    #1 }}}
\newcommand{\jhanote}[1]{  {\textcolor{red}     { ***Shantenu: #1 }}}
\else
\newcommand{\amnote}[1]{}
\newcommand{\jhanote}[1]{}
\fi

\begin{document}

 \title{ \Large \vspace{-3.5em} A perl script for automated API bundle installations (SAGA)}
 
 \author{\normalsize SAGA-Group}
 \date{\normalsize 06-13-2011}
 \maketitle
 

% \jhanote{Remember in addition to serving as an abstract, this will
%   serve as a summary of what will go to the 3 editors of the journals
%   that we are considering publishing a full paper in. Thus some more
%   information/discussion on what the underlying problem and context
%   will be about.}


% \jhanote{Once we have defined / introduced SAGA, we should probably
%   have 3 subsections -- interface, library and adatptors/backends?}

% outline for paper
% quality control of software packages on various testing environment before deploying it on production environment - 
%not manual but a continuous integrated framework for that - NMI funded by NSF dealt with deployment? - the first step to 
%building such deployment tool - a continuous integration framework - buildbot was used - which needed external manual
%installations of its dependencies thus - mephisto for SAGA was developed which automatically built packages - further envisioning
%to be delveoped into continuous integration with buildbot and later developed further into a SAGA deployment tool. 
%Present paper details the working of mephisto that could be generalized for automated building. 

%Failed attempt to write introduction
%SAGA is an API package for building grid applications, tools and framework. Similar to some of the traditional API packages, SAGA library set 
%depends on various other packages. It is very important for these dependency packages to be installed accurately on any working environments.
%Some of the applications using SAGA requires SAGA installations to be completed in multiple environments which are sometimes geographically separated 
%or can be virtual. Manual installations on every resource available can be very tedious. Testing SAGA for bug detection and fixing
%requires installations to be automated and notifications sent to developers which would speed up the bug fixing process and release. SAGA team
%employed an automated test bed tool named Buildbot which triggers compatibility test with each revision made to the SAGA library and
%also send notifications after each build. Thus the testing phase was partially taken care of since it still needed all the dependency
%packages to be installed in the system running the test. To be able to continuously integrate testing and efficiently deploy SAGA on any machine 
%geographically apart contributes best to the efficiently manage SAGA and its evolution. This led to writing a simple perl script which when integrated with 
%buildbot would initiate installations and building in succession. This perl script in a combination with Buildbot  can later be developed to be 
%used as a software deployment tool on not only testing environments but also production environments i.e., resources using SAGA. The script was
%called 'mephisto.pl' and section 2 of this paper will give more insight into its functionalities. The paper will proceed to try to explain the existing 
%methods of software deployment and ways of integrating the developed script in section 3.1. Section 3.2 would envision possible modifications that can be done to 
%the existing script to make it versatile and then end with a conclusion. 


% explore: the basic need for faster or non-redundant 
\begin{multicols}{2}
\section*{\normalsize \textbf{Abstract}}
\textbf{Developing, testing and debugging software packages before making them available on the production environment should be well 
tested on various working platforms. Packages to be released in future, or an update made to the previous version, are rigorously
made bug-free and versatile to the changing environments and dependency packages. These noted steps involved in software deployment cycle of 
any software package can often be complex and is made short through automated build systems, and continuous integrated frameworks~\cite{wiki-soft-dep}. 
The best case however is to have a software deployment tool that automates every single step of deployment - developing, bug tracking and distribution. 
Present paper deals with a simple perl script (mephisto.pl) that automates build process of SAGA (Simple API for Grid Applications) along
with its dependencies. It is light-weight, flexible, and portable that can be 
generalized to any software development process. The paper presented goes through the existing problem faced and mephisto 
as its possible solution. The paper includes its future possibilities of including mephisto into a continuous integration framework and serve as a software deployment 
tool for SAGA. The paper tries to compare some of the existing software deployment tools such as NMI-GRIDS testing, NMI build 
and test lab, and automated building tools such as Buildbot to explain their standing with respect to the script being developed~\cite{wiki-buildsofts}. 
}
\section*{\normalsize 1. Introduction}
Developing software packages require immediate bug fixes and tools for faster deployment. Like many other software packages, SAGA 
was required to be installed on different platforms to carry out version compatibility tests of its dependency packages, bug fixes and running applications. 
In the beginning manual installations were employed to get a working copy of SAGA on different machines. This was time consuming 
since installations required its dependencies to be taken care of, which differed with every working platform. Also any updates made to SAGA 
or its dependencies should be tested on different platforms but installations had to be implemented prior to any testing. Tracking and reporting 
bugs from testing-results was also important before deploying bug free SAGA version on production environment or for applications
to be carried out. This cyclic process required repeated installations and manually installing them every time were to slow the development
cycle. Also extra care was to be taken to make sure the development community were kept informed of the different bugs suspected,
their logs, changes already made and tested. This entire process required a central administrator to co-ordinate all the work that lacked
a distributed framework essential for a smooth and exponential development process. Hence the problem was to develop a tool that automated 
installations, tracked bugs and maintained a log record for the developers which would in turn facilitate parallel independent research work into
developing SAGA. This is practiced in many software development process for faster development and to stay stable with its dependency 
packages on all platforms alike~\cite{nmi}. Thus as an initial step to build a complete deployment tool for SAGA, 'mephisto.pl', a perl script was developed to 
fill in for automated installations or building SAGA automatically. 
A simple command line option would install SAGA and all its dependencies irrespective of the underlying computer resource used. 
In future it would be integrated with Buildbot, an automated software testing tool. The paper will proceed to explain more about mephisto 
and its working in Section 3 following some of the existing deployment tools and possible discussion in Section 2. 
\section*{\normalsize 2. Existing methods}
There are wide variety of software deployment tools available in the market both commercial and open source. Deployment tools available for 
grid computing software packages are constrained to a few. CMS and ADEM are among the few deployment tools employed for grid software packages at 
CERN~\cite{cms,adem}. 
A few other tools are aimed just at automating development and building process together termed as building systems. Building systems can be categorized into 
on-demand, scheduled, and triggered automation~\cite{wiki-build-auto}. NMI scripts, and buildbot are to name a few. 
\subsection*{\normalsize 2.1 NMI scripts and NMI build and test labs} 
NMI-GRID scripts, an NSF funded project was aimed at developing a software testing structure that automated building and testing of software packages. 
The primary objective of the project was to automate the entire process without the control being centered on a single local administrator. Any 
commits done to the subversion repositories by project developers would automatically trigger temporary building process on different working
platforms while logging each result either failure or success. Multiple requests for testing are queued before executing them at a time. All the results are 
stored which are visible for the entire developers community alike to be reviewed later~\cite{nmi-grids,nmi}. Several other NMI building and testing labs 
available for scientific community are maintained by UW madison, CERN, INFN, and TeraGrid. They provide both infrastructural and software support 
necessary for development and building process~\cite{nmi-build}.
\subsection*{\normalsize 2.2 Buildbot}
Buildbot is one such other way of triggered automation employed by SAGA developers group. The architecture consists of a build master that 
triggers a series of builds and testing upon each commits done to subversion or cvs or any such repositories or directories. The testing process 
are to be carried out on its slave machines working on different platforms. All the notifications after each built are sent to all developers community 
either by mailing or maintaining a web space. All the governing options can be configured prior to any of the developing process. This is an open source 
project targeted to a wide range of software packages~\cite{buildbot}.
\section*{\normalsize 3. Working of mephisto}
Example of the command line statement: 
\begin{verbatim} perl mephisto install 
--target-dir=/home/user/saga
 --tmp-dir=/home/user/tmp-saga \end{verbatim} 
\subsection*{\normalsize 3.1 Motivation and reason}
Buildbot was already used as an automated tool for solving SAGA building problems. But it still required to install all the dependencies of SAGA 
on different machines where (buildbot slaves) SAGA testing was performed. This operation could not be scaled very quickly to 
newer machines or platforms to be added further. Thus it required a tool that downloaded all its dependencies temporarily on the disk space and then
continued building process. This was necessary as different platforms provided different dependency challenges. It was required to test 
different versions of the packages at users discretion for complete testing which buildbot failed to provide. Thus mephisto
was developed such that it could download all its dependency packages with the specified versions using a single command line option
provided by the user and proceed to build SAGA. 
\subsection*{\normalsize 3.2 Commands for execution}
Given are all the various options available at users disclosure. Mephisto pulls up the packages information from a pre-defined repository. 
By default if no option is specified, mephisto pulls up the packages listed on repository at
\begin{verbatim} 
http://static.saga.cct.lsu.edu/
mephisto/repository/latest/INDEX. 
\end{verbatim}
Expanded version of the Command to run mephisto
\begin{verbatim} 
perl mephisto install/test 
--target-dir=/path/to/folder 
--tmp-dir=/path/to/folder 
--with-saga-version= 
--with-boost-version= 
--with-globus-version= 
--repository= --with-adaptors= 
--with-python-version= 
--with-postgresql-version=
\end{verbatim} 
\emph{install} \\*
 This can be used to install a working copy of SAGA and all its 
 dependencies specified. \\*
\emph{test} \\*
 This option not only installs SAGA but also runs the command 
 'make check' to check the installation. It creates a log file displaying 
 the test results. \\*
\emph{target-dir} \\*
 This option is used to specify the path you need to install SAGA and all 
 packages libraries. By default, it creates a folder at tmp/meph\_inst.\\*
\emph{tmp-dir} \\*
 This option is used to specify the path for a temporary folder to 
 download all the packages and create log files. If not mentioned, it 
 creates a folder at /tmp/meph\_tmp\\*
\emph{repository} \\*
 Can be used to choose different repositories to install SAGA from. 
 Two repositories available as of now are 'latest' (default) and svn\_trunk. 
 Recommended svn\_trunk for the most recent SAGA version. \\*
\emph{with-saga-version}\\*
 Choose saga versions available.\\*
\emph{with-boost-version}\\*
Choose boost versions available $>$=1.40.0\\*
\emph{with-globus-version} \\*
 Choose globus version available $>$=4.0. This option would automatically 
 invoke installations of X509, globus adaptors too. \\*
\emph{with-adaptors}\\*
 Mention any adaptors to install that would invoke globus installations along with 
 saga-adaptors too. \\*
 \emph{with-python-version}\\*
 Choose from the list of python versions available. Must be 
 $>$=2.6.2 \\*
 \emph{with-postgresql-version}\\*
 Choose from postgresql versions available. Should be $>$=8.4.1
 
\subsection*{\normalsize 3.3 Repositories for Mephisto} 
Mephisto uses two different repositories for SAGA packages which have been
pre-defined by the SAGA group, namely: 
\begin{verbatim}
(i) http://static.saga.cct.lsu.edu/
mephisto/repository/latest/
\end{verbatim}
\begin{verbatim}
(ii) http://static.saga.cct.lsu.edu/
mephisto/repository/svn_trunk
\end{verbatim}
When the user doesn't define any repository, the default 
repository taken by mephisto will be 
\emph {http://static.saga.cct.lsu.edu/mephisto/repository/latest/}. 
\subsection*{\normalsize 3.4 Future of mephisto}
This simple script could thus provide a working copy of SAGA on all the different working platforms alike 
with command line options specified. It could also be used to choose the type of version available. This would 
be included with buildbot scripts to form a continuous integrated testing tool such that the script should be
deployed on various working platforms and later send the notifications of the testing results. This would be completely
independent of the underlying working environment~\cite{wiki-ci}. The future would be to develop a way to deploy SAGA 
on required machines upon obtaining a completely reviewed bug free SAGA version when triggered manually for deployment.
\subsection*{\normalsize 3.5 Availability}
A copy of mephisto.pl can be obtained from \emph{http://faust.cct.lsu.edu/trac/mephisto}. Other details and documentation 
can be found on the same website. An installed version of perl is the only necessary component on your system to run mephisto.pl
\section*{\normalsize 4. Conclusion}
The paper presented a solution to automate building process for SAGA with its various dependency packages as the 
first step in the process of developing an entire distributed deployment structure for SAGA. The script can very easily be modified 
to apply for automated building process of any software packages. It is flexible, and portable. The perl script can also be easily integrated to 
any automated testing software tool to become a continuous testing platform for faster software development.

%
%\end{multicols}
%\section*{\normalsize Introduction}
%% talk about prior installations/solutions & disadv; reason for writing this script; structure and command line tools; 
%% some advantages; adjustments can be made; conclusions.
%% ques 1: bundle? or API specifications? or API package? or Libraries set? 
%% ques 2: SAGA on each and every platform required?
%Like any other software or API packages, SAGA depends on other back end software packages. A collection of such libraries
%or packages should be installed and their path identifiable prior installing SAGA which is checked automatically 
%during the installation process. Some of the distributed applications or programs written using SAGA involving two or more distributed
%resources also require SAGA to be installed in those resources. Thus making the installation procedure lengthy and tedious for a user to cope.
%Also various research operations being done on SAGA requires its installations to be done in a single step process. For this purpose, 
%a simple perl script was developed which would automate the entire installations right from its dependencies to its installations and also
%check SAGA installations upon users command. It is also designed to give the user the control to choose the apt version of the packages 
%being installed. \\*
%Future improvement were thought to include mephisto in a continuous testing framework such that it could be used to automatically test SAGA's 
%compatibilities with different versions of the backend packages. Thus mephisto was provided with options that could take in any version number 
%of any package. Buildbot was the tool used to continuously testing after each committed version of SAGA but it still required to install SAGA's dependencies 
%separately on each machine it ran the test. Thus mephisto was thought to eliminate that necessity. Although mephisto was not completely embedded
%into buildbot to provide continuous testing framework, it could successfully make installations easy and testing become possible in a single command run. 
%The paper next will go through the detailed structure of mephisto and its command line tools followed by some of the advantages achieved, 
%future modifications to it and conclusion at the end. 

\bibliographystyle{IEEEtran} 
\bibliography{mephisto}
\end{multicols}

\end{document}

