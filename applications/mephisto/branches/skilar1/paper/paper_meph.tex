\documentclass[a4paper,10pt]{article}
\usepackage[utf8]{inputenc}
\usepackage{graphicx}
\usepackage{url}
\usepackage{float}
\usepackage{times}
\usepackage{multirow}
\usepackage{listings}
\usepackage{times}
\usepackage{paralist}
\usepackage{epsfig}
\usepackage{subfigure}
\usepackage[hypertex]{hyperref}
\usepackage{subfigure}
\usepackage{color}
\usepackage{xspace}
\usepackage{multicol}
%\documentclass{rspublic}

\usepackage{ifpdf}

\newcommand{\I}[1]{\textit{#1}}
\newcommand{\B}[1]{\textbf{#1}}
\newcommand{\BI}[1]{\textbf{\textit{#1}}}
\newcommand{\T}[1]{\texttt{#1}}

\newcommand{\sagaspec}{\textit{SAGA}\xspace}
\newcommand{\sagaimpl}{\textit{SAGA}\xspace}

\newcommand{\spec}{\sagaspec}
\newcommand{\impl}{\sagaimpl}

\setlength\topmargin{0in}
\setlength\headheight{0in}
\setlength\headsep{0in}
\setlength\textheight{9.5in}
\setlength\textwidth{6.5in}
\setlength\oddsidemargin{0in}
\setlength\evensidemargin{0in}
\setlength\parindent{0.1in}
\setlength\parskip{0.25em}


\ifpdf
 \DeclareGraphicsExtensions{.pdf, .jpg,.jpeg}
\else
 \DeclareGraphicsExtensions{.eps, .ps}
\fi

\newcommand{\note}[1]{ {\textcolor{red} { ***NOTE: #1 }}}

\newif\ifdraft
\drafttrue

\ifdraft
\newcommand{\amnote}[1]{   {\textcolor{magenta} { ***Andre:    #1 }}}
\newcommand{\jhanote}[1]{  {\textcolor{red}     { ***Shantenu: #1 }}}
\else
\newcommand{\amnote}[1]{}
\newcommand{\jhanote}[1]{}
\fi

\begin{document}

 \title{ \Large \vspace{-3.5em} A perl script for automated API bundle installations (SAGA)}
 
 \author{\normalsize SAGA-Group}
 \date{\normalsize 06-13-2011}
 \maketitle
 

% \jhanote{Remember in addition to serving as an abstract, this will
%   serve as a summary of what will go to the 3 editors of the journals
%   that we are considering publishing a full paper in. Thus some more
%   information/discussion on what the underlying problem and context
%   will be about.}


% \jhanote{Once we have defined / introduced SAGA, we should probably
%   have 3 subsections -- interface, library and adatptors/backends?}

% outline for paper
% quality control of software packages on various testing environment before deploying it on production environment - 
%not manual but a continuous integrated framework for that - NMI funded by NSF dealt with deployment? - the first step to 
%building such deployment tool - a continuous integration framework - buildbot was used - which needed external manual
%installations of its dependencies thus - mephisto for SAGA was developed which automatically built packages - further envisioning
%to be delveoped into continuous integration with buildbot and later developed further into a SAGA deployment tool. 
%Present paper details the working of mephisto that could be generalized for automated building. 

%Failed attempt to write introduction
%SAGA is an API package for building grid applications, tools and framework. Similar to some of the traditional API packages, SAGA library set 
%depends on various other packages. It is very important for these dependency packages to be installed accurately on any working environments.
%Some of the applications using SAGA requires SAGA installations to be completed in multiple environments which are sometimes geographically separated 
%or can be virtual. Manual installations on every resource available can be very tedious. Testing SAGA for bug detection and fixing
%requires installations to be automated and notifications sent to developers which would speed up the bug fixing process and release. SAGA team
%employed an automated test bed tool named Buildbot which triggers compatibility test with each revision made to the SAGA library and
%also send notifications after each build. Thus the testing phase was partially taken care of since it still needed all the dependency
%packages to be installed in the system running the test. To be able to continuously integrate testing and efficiently deploy SAGA on any machine 
%geographically apart contributes best to the efficiently manage SAGA and its evolution. This led to writing a simple perl script which when integrated with 
%buildbot would initiate installations and building in succession. This perl script in a combination with Buildbot  can later be developed to be 
%used as a software deployment tool on not only testing environments but also production environments i.e., resources using SAGA. The script was
%called 'mephisto.pl' and section 2 of this paper will give more insight into its functionalities. The paper will proceed to try to explain the existing 
%methods of software deployment and ways of integrating the developed script in section 3.1. Section 3.2 would envision possible modifications that can be done to 
%the existing script to make it versatile and then end with a conclusion. 
\begin{multicols}{2}
\section*{\normalsize \textbf{Abstract}}
\textbf{Developing, testing and debugging of software packages before making them available for the production environment should be well 
tested on various testing platforms. Packages to be released in future, in this case an update made for the previous version, again are
made bug-free and versatile to the changing environments and dependent packages. Above noted steps involved in software deployment of 
any software package can often be complex and is made short through automated build systems, and continuous integration frameworks. 
The best case however is to have a software deployment tool that automates every single step of deployment - developing, bug tracking and distribution. 
Present paper deals with a simple perl script (mephisto.pl) that automates build process of SAGA (Simple API for Grid Applications), 
a tool that can be generalized for any developing software package. The paper would also go through the working model of mephisto and 
discuss its future work of featuring as a continuous integration tool and later serve as a software deployment tool for SAGA. 
The paper will also try to explore some of the existing software deployment tools such as NMI-GRIDS testing, NMI build and test lab, 
and automated building tools such as Buildbot to try to explain its standing with respect to the script being developed. 
}
\section*{\normalsize 1. Introduction}
Developing software packages requires immediate bug fixes and tools for faster deployment. Like many other software packages, SAGA 
was required to be installed on different platforms to carry out version compatibility tests of its dependency packages, bug fixes and running applications . 
In the beginning manual installations was employed to get a working copy of SAGA on different machines. This was time consuming 
since installations required its dependencies to be taken care of, which deferred with every working platform. Also any version update made to SAGA 
or its dependencies should be tested on different platform but installations of the newer versions were necessary prior testing. Tracking and reporting 
bugs from testing for software betterment was important before deploying bug free SAGA version on production environment or for applications
to be carried out. This cyclic process required repeated installations and manually installing them every time would slower the development
cycle. Also extra care was to be taken to make sure the development community were on the same page with respect to different bugs suspected,
their logs, change already made and tested. This required a central administrator for co-ordinating all the work, which lacked
a distributed framework essential for a smooth and exponential development process. Hence the problem was to develop a tool that automated 
installations, bug tracking and maintain a log record for the developers which would in turn facilitate parallel independent research work into
developing SAGA. This is practiced in many software development process for faster development and to be compatible with the dependency 
packages updates on all platform alike. Thus 'mephisto.pl', a perl script was developed to fill in for automated installations.  
A simple command line option would install SAGA and all its dependencies irrespective of the underlying computer resource used. 
This was later hoped to be integrated with Buildbot, an automated software testing tool, which would then automate both development and testing. 
The paper will proceed to explain more about mephisto and its working in Section 2 followed by some of the existing deployment tools and possible 
alterations to mephisto in Section 3. 
\section*{\normalsize 2. Working of mephisto}
Mephisto is a tool that is aimed at  automating the building process of SAGA, installing all its dependencies specified for SAGA and SAGA adaptors too.  
The version number of all the packages and the type of packages can be specified by the user at the command line. The description of all the command 
line options available are described in the next section.\\*
Example of the command line statement: 
\begin{verbatim} perl mephisto install --target-dir=/home/user/saga --tmp-dir=/home/user/tmp-saga \end{verbatim} 
This will be followed by few random examples of mephisto usage and screenshots taken during the
process. Refer to the last section for common errors found and suggestions to counter them\\*
\subsection*{\normalsize 2.1 Download mephisto.pl and other information}
A copy of mephisto.pl can be obtained from \emph{http://faust.cct.lsu.edu/trac/mephisto}. 
Other details and documentation can be found on the same website. 
An installed version of perl would be necessary on your system to run mephisto.pl 
\subsection*{\normalsize 2.2 Command line options}
This section discusses all the various options available at 
users disclosure. By default if no option is specified, mephisto 
pulls up the packages listed on the repositories listed at
\begin{verbatim} 
http://static.saga.cct.lsu.edu/
mephisto/repository/latest/INDEX. 
\end{verbatim}
Expanded version of the Command to run mephisto
\begin{verbatim} 
perl mephisto install/test 
--target-dir=/path/to/folder 
--tmp-dir=/path/to/folder 
--with-saga-version= 
--with-boost-version= 
--with-globus-version= 
--repository= --with-adaptors= 
--with-python-version= 
--with-postgresql-version=
\end{verbatim} 
\emph{install} \\*
 This can be used to install a working copy of SAGA and all its 
 dependencies specified. \\*
\emph{test} \\*
 This option not only installs SAGA but also runs the command 
 'make check' to check the installation. It creates a log file displaying 
 the test results. \\*
\emph{target-dir} \\*
 This option is used to specify the path you need to install SAGA and all 
 packages libraries. By default, it creates a folder at tmp/meph\_inst.\\*
\emph{tmp-dir} \\*
 This option is used to specify the path for a temporary folder to 
 download all the packages and create log files. If not mentioned, it 
 creates a folder at /tmp/meph\_tmp\\*
\emph{repository} \\*
 Can be used to choose different repositories to install SAGA from. 
 Two repositories available as of now are 'latest' (default) and svn\_trunk. 
 Recommended svn\_trunk for the most recent SAGA version. \\*
\emph{with-saga-version}\\*
 Choose saga versions available.\\*
\emph{with-boost-version}\\*
Choose boost versions available $>$=1.40.0\\*
\emph{with-globus-version} \\*
 Choose globus version available $>$=4.0. This option would automatically 
 invoke installations of X509, globus adaptors too. \\*
\emph{with-adaptors}\\*
 Mention any adaptors to install that would invoke globus installations along with 
 saga-adaptors too. \\*
 \emph{with-python-version}\\*
 Choose from the list of python versions available. Must be 
 $>$=2.6.2 \\*
 \emph{with-postgresql-version}\\*
 Choose from postgresql versions available. Should be $>$=8.4.1
\subsection*{\normalsize 2.3 Repositories for Mephisto} 
Mephisto uses two different repositories for SAGA packages which have been
pre-defined by the SAGA group, namely: 
\begin{verbatim}
(i) http://static.saga.cct.lsu.edu/
mephisto/repository/latest/
\end{verbatim}
\begin{verbatim}
(ii) http://static.saga.cct.lsu.edu/mephisto/
repository/svn_trunk
\end{verbatim}
When the user doesn't define any repository, the default 
repository taken by mephisto will be \emph {http://static.saga.cct.lsu.edu/
mephisto/repository/latest/}. \\*
The list of packages pre-defined for (i) are: \\*
SAGA= 1.5.2; \\*
Boost= 1.40.0; \\*
Python=2.6.2; \\*
Post-gre-sql=8.4 .1 ; \\*
Sqlite=3.6.18;   \\*
SAGA-PYTHON; \\*
SAGA-ADAPTORS-X509; \\*
SAGA-ADAPTORS-SSH.  \\*\\*
The list of packages pre-defined for (ii) are: \\*
SAGA= 1.5.4 (updated as of now); \\*
Boost= 1.40.0; \\*
Python=2.6.2; \\*
Post-gre-sql=8.4 .1 ; \\*
Sqlite=3.6.18;   \\*
SAGA-PYTHON. \\*\\*
The suggested repository to be utilized by the user will be \emph{svn\_trunk} 
that is updated with the latest version of SAGA as and when
new version is released with bug fixes. Example to switch between 
repositories follows in the next section. 
\subsection*{\normalsize 2.4 Example Command line options}
\emph{NOTE: All examples here are tackled one at a time. They can be used interchangeably.
Choice has to be made between 'install' and 'test'}
\subsection*{\normalsize 2.4.1 Basic Default command}
\begin{verbatim}
perl mephisto install 
\end{verbatim}
This would install SAGA. All the default options will be loaded as no
option was chosen here. As of now the default options are as foretold 
from the repository given below. As no directory was mentioned all the 
packages will be stored in a temporary folder created by the script.
%\begin{verbatim}
http://static.saga.cct.lsu.edu/mephisto/repository/latest/
%\end{verbatim}
\subsection*{\normalsize 2.4.2 Command with paths to folders}
\begin{verbatim}
perl mephisto install 
--target-dir=/path/to/any/destination/ 
--tmp-dir=/path/to/temp/folder 
\end{verbatim}
This will install all packages in the folder specified. 
\subsection*{\normalsize 2.4.3 Install SAGA from svn\_trunk repository}
\begin{verbatim}
perl mephisto install 
--target-dir=/path/to/any/destination/ 
--tmp-dir=/path/to/temp/folder 
--repository=svn_trunk
\end{verbatim}
Choose snv\_trunk repository to install SAGA. This will have the most
recent and tested working SAGA version. 
\subsection*{\normalsize 2.4.4 Install SAGA will all saga, globus, boost versions specified}
\begin{verbatim}
perl mephisto install 
--target-dir=/path/to/any/destination/ 
--tmp-dir=/path/to/temp/folder 
--with-saga-version=1.5.3 
--with-boost-version=1.44.0 
--with-globus-version=5.0.2
(or)
perl mephisto install 
--target-dir=/path/to/any/destination/ 
--tmp-dir=/path/to/temp/folder 
--with-saga=1.5.3 
--with-boost=1.44 
--with-globus=5.0.2
\end{verbatim}
To install SAGA with different boost, SAGA and globus versions.
Given here are two ways of defining the versions. Using all the three 
is not compulsory and any combination can be used. Any version 
not mentioned will infer default options. 
\subsection*{\normalsize 2.4.5 Globus installations}
\begin{verbatim}
perl mephisto install 
--target-dir=/path/to/any/destination/ 
--tmp-dir=/path/to/temp/folder 
--with-globus-version=5.0.2
\end{verbatim}
Specifying any globus installations will also invoke globus adaptors
installations which are required and necessary with globus.
\subsection*{\normalsize 2.4.6 Install SAGA with Globus adaptors}
\begin{verbatim}
perl mephisto install 
--target-dir=/path/to/any/destination/ 
--tmp-dir=/path/to/temp/folder 
--with-adaptors=globus
\end{verbatim}
To install globus adaptors. Automatically invokes globus installations which 
is required for adaptors. Also automates x509 adaptors installations too.
This is similar to what we have in the previous example but different 
command line options have been given for easy usage.  
\subsection*{\normalsize 2.4.7 How to install a SAGA Core Pre-Release with a specific Boost version (for developers)}
\begin{verbatim}
perl mephisto install 
--target-dir=/path/to/any/destination/ 
--tmp-dir=/path/to/temp/folder 
--with-saga=1.5.3-pre
\end{verbatim}
For developers, this would install the pre-release version from the given 
link shared in the email with boost and all other dependencies creating log 
files necessary for debugging. 
\subsection*{\normalsize 2.4.8 Install with Python versions}
\begin{verbatim}
perl mephisto test 
--target-dir=/path/to/any/dest/ 
--tmp-dir=/path/to/temp/folder 
--with-python-version=2.7.1
\end{verbatim}
Install with any python version $>$=2.6.2. Please note that you may 
be thrown an error and break from the build for versions incompatibility.
\subsection*{\normalsize 2.4.9 Install with postgresql version}
\begin{verbatim}
perl mephisto test 
--target-dir=/path/to/any/destination/ 
--tmp-dir=/path/to/temp/folder 
--with-postgresql-version=8.4.2
\end{verbatim}
Similar to what was described earlier for python versions right above.
\subsection*{\normalsize 2.4.10 Install and proceed to TEST the installations}
\begin{verbatim}
perl mephisto test 
--target-dir=/path/to/any/destination/ 
--tmp-dir=/path/to/temp/folder 
--with-saga-version=1.5.3 
--with-boost-version=1.44.0 
--with-globus-version=5.0.2
\end{verbatim}
Choose 'test' option to first install SAGA and its dependencies and then have it
continuously perform 'make check' operation and create a log file in the end.
\section*{\normalsize 3. Present methods and possible developments}
buildbot, NMI scripts like -> mephisto
\section*{\normalsize 4. Conclusion}

\end{multicols}
%\section*{\normalsize Introduction}
%% talk about prior installations/solutions & disadv; reason for writing this script; structure and command line tools; 
%% some advantages; adjustments can be made; conclusions.
%% ques 1: bundle? or API specifications? or API package? or Libraries set? 
%% ques 2: SAGA on each and every platform required?
%Like any other software or API packages, SAGA depends on other back end software packages. A collection of such libraries
%or packages should be installed and their path identifiable prior installing SAGA which is checked automatically 
%during the installation process. Some of the distributed applications or programs written using SAGA involving two or more distributed
%resources also require SAGA to be installed in those resources. Thus making the installation procedure lengthy and tedious for a user to cope.
%Also various research operations being done on SAGA requires its installations to be done in a single step process. For this purpose, 
%a simple perl script was developed which would automate the entire installations right from its dependencies to its installations and also
%check SAGA installations upon users command. It is also designed to give the user the control to choose the apt version of the packages 
%being installed. \\*
%Future improvement were thought to include mephisto in a continuous testing framework such that it could be used to automatically test SAGA's 
%compatibilities with different versions of the backend packages. Thus mephisto was provided with options that could take in any version number 
%of any package. Buildbot was the tool used to continuously testing after each committed version of SAGA but it still required to install SAGA's dependencies 
%separately on each machine it ran the test. Thus mephisto was thought to eliminate that necessity. Although mephisto was not completely embedded
%into buildbot to provide continuous testing framework, it could successfully make installations easy and testing become possible in a single command run. 
%The paper next will go through the detailed structure of mephisto and its command line tools followed by some of the advantages achieved, 
%future modifications to it and conclusion at the end. 

% \bibliographystyle{IEEEtran} 
%\bibliography{mephisto}


\end{document}

