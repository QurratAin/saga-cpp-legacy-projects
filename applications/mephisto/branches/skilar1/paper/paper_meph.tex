\documentclass[a4paper,10pt]{article}
\usepackage[utf8]{inputenc}
\usepackage{graphicx}
\usepackage{url}
\usepackage{float}
\usepackage{times}
\usepackage{multirow}
\usepackage{listings}
\usepackage{times}
\usepackage{paralist}
\usepackage{epsfig}
\usepackage{subfigure}
\usepackage[hypertex]{hyperref}
\usepackage{subfigure}
\usepackage{color}
\usepackage{xspace}

%\documentclass{rspublic}

\usepackage{ifpdf}

\newcommand{\I}[1]{\textit{#1}}
\newcommand{\B}[1]{\textbf{#1}}
\newcommand{\BI}[1]{\textbf{\textit{#1}}}
\newcommand{\T}[1]{\texttt{#1}}

\newcommand{\sagaspec}{\textit{SAGA}\xspace}
\newcommand{\sagaimpl}{\textit{SAGA}\xspace}

\newcommand{\spec}{\sagaspec}
\newcommand{\impl}{\sagaimpl}

\setlength\topmargin{0in}
\setlength\headheight{0in}
\setlength\headsep{0in}
\setlength\textheight{9.5in}
\setlength\textwidth{6.5in}
\setlength\oddsidemargin{0in}
\setlength\evensidemargin{0in}
\setlength\parindent{0.1in}
\setlength\parskip{0.25em}


\ifpdf
 \DeclareGraphicsExtensions{.pdf, .jpg,.jpeg}
\else
 \DeclareGraphicsExtensions{.eps, .ps}
\fi

\newcommand{\note}[1]{ {\textcolor{red} { ***NOTE: #1 }}}

\newif\ifdraft
\drafttrue

\ifdraft
\newcommand{\amnote}[1]{   {\textcolor{magenta} { ***Andre:    #1 }}}
\newcommand{\jhanote}[1]{  {\textcolor{red}     { ***Shantenu: #1 }}}
\else
\newcommand{\amnote}[1]{}
\newcommand{\jhanote}[1]{}
\fi

\begin{document}

 \title{ \Large \vspace{-3.5em} A perl script for automated API bundle installations (SAGA)}
 
 \author{\normalsize SAGA-Group}
 \date{\normalsize 06-13-2011}
 \maketitle
 

% \jhanote{Remember in addition to serving as an abstract, this will
%   serve as a summary of what will go to the 3 editors of the journals
%   that we are considering publishing a full paper in. Thus some more
%   information/discussion on what the underlying problem and context
%   will be about.}


% \jhanote{Once we have defined / introduced SAGA, we should probably
%   have 3 subsections -- interface, library and adatptors/backends?}

% outline for paper
% quality control of software packages on various testing environment before deploying it on production environment - 
%not manual but a continuous integrated framework for that - NMI funded by NSF dealt with deployment? - the first step to 
%building such deployment tool - a continuous integration framework - buildbot was used - which needed external manual
%installations of its dependencies thus - mephisto for SAGA was developed which automatically built packages - further envisioning
%to be delveoped into continuous integration with buildbot and later developed further into a SAGA deployment tool. 
%Present paper details the working of mephisto that could be generalized for automated building. 

\section*{\normalsize \textbf{Abstract}}
\textbf{Developing, testing and debugging of software packages before making them available for the production environment should be well 
tested on various testing platforms.  Packages to be released in future, in this case an update made for the previous version, again are
made bug-free and versatile to the changing environments and dependent packages. Above noted steps involved in software deployment of 
any software package often can be complex and is made short through automated build systems, and continuous integration frameworks. 
The best case however is to have a software deployment tool that automates every single step of deployment - developing, bug tracking and distribution. 
Present paper deals with a simple perl script (mephisto.pl) that automates build process of SAGA (Simple API for Grid Applications), 
a tool that can be generalized for any developing software package. The paper will also go through the working model of mephisto and 
discuss its future work to be able to develop it into a continuous integration tool and later serve as a software deployment tool for SAGA. 
The paper will also try to explore some of the existing software deployment tools such as NMI-GRIDS testing, NMI build and test lab, 
and automated building tools such as Buildbot to try to explain its standing with respect to the script being developed. 
}

%\section*{\normalsize Introduction}
%% talk about prior installations/solutions & disadv; reason for writing this script; structure and command line tools; 
%% some advantages; adjustments can be made; conclusions.
%% ques 1: bundle? or API specifications? or API package? or Libraries set? 
%% ques 2: SAGA on each and every platform required?
%Like any other software or API packages, SAGA depends on other back end software packages. A collection of such libraries
%or packages should be installed and their path identifiable prior installing SAGA which is checked automatically 
%during the installation process. Some of the distributed applications or programs written using SAGA involving two or more distributed
%resources also require SAGA to be installed in those resources. Thus making the installation procedure lengthy and tedious for a user to cope.
%Also various research operations being done on SAGA requires its installations to be done in a single step process. For this purpose, 
%a simple perl script was developed which would automate the entire installations right from its dependencies to its installations and also
%check SAGA installations upon users command. It is also designed to give the user the control to choose the apt version of the packages 
%being installed. \\*
%Future improvement were thought to include mephisto in a continuous testing framework such that it could be used to automatically test SAGA's 
%compatibilities with different versions of the backend packages. Thus mephisto was provided with options that could take in any version number 
%of any package. Buildbot was the tool used to continuously testing after each committed version of SAGA but it still required to install SAGA's dependencies 
%separately on each machine it ran the test. Thus mephisto was thought to eliminate that necessity. Although mephisto was not completely embedded
%into buildbot to provide continuous testing framework, it could successfully make installations easy and testing become possible in a single command run. 
%The paper next will go through the detailed structure of mephisto and its command line tools followed by some of the advantages achieved, 
%future modifications to it and conclusion at the end. 

% \bibliographystyle{IEEEtran} 
%\bibliography{mephisto}


\end{document}

