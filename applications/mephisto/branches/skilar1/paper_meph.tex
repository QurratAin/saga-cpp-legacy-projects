\documentclass[a4paper,10pt]{article}
\usepackage[utf8]{inputenc}
\usepackage{graphicx}
\usepackage{url}
\usepackage{float}
\usepackage{times}
\usepackage{multirow}
\usepackage{listings}
\usepackage{times}
\usepackage{paralist}
\usepackage{epsfig}
\usepackage{subfigure}
\usepackage[hypertex]{hyperref}
\usepackage{subfigure}
\usepackage{color}
\usepackage{xspace}

%\documentclass{rspublic}

\usepackage{ifpdf}

\newcommand{\I}[1]{\textit{#1}}
\newcommand{\B}[1]{\textbf{#1}}
\newcommand{\BI}[1]{\textbf{\textit{#1}}}
\newcommand{\T}[1]{\texttt{#1}}

\newcommand{\sagaspec}{\textit{SAGA}\xspace}
\newcommand{\sagaimpl}{\textit{SAGA}\xspace}

\newcommand{\spec}{\sagaspec}
\newcommand{\impl}{\sagaimpl}

\setlength\topmargin{0in}
\setlength\headheight{0in}
\setlength\headsep{0in}
\setlength\textheight{9.5in}
\setlength\textwidth{6.5in}
\setlength\oddsidemargin{0in}
\setlength\evensidemargin{0in}
\setlength\parindent{0.1in}
\setlength\parskip{0.25em}


\ifpdf
 \DeclareGraphicsExtensions{.pdf, .jpg,.jpeg}
\else
 \DeclareGraphicsExtensions{.eps, .ps}
\fi

\newcommand{\note}[1]{ {\textcolor{red} { ***NOTE: #1 }}}

\newif\ifdraft
\drafttrue

\ifdraft
\newcommand{\amnote}[1]{   {\textcolor{magenta} { ***Andre:    #1 }}}
\newcommand{\jhanote}[1]{  {\textcolor{red}     { ***Shantenu: #1 }}}
\else
\newcommand{\amnote}[1]{}
\newcommand{\jhanote}[1]{}
\fi

\begin{document}

 \title{ \Large \vspace{-3.5em} A perl script for automated API budle installations (SAGA)}
 
 \author{\normalsize SAGA-Group}
 \date{\normalsize 06-13-2011}
 \maketitle
 

% \jhanote{Remember in addition to serving as an abstract, this will
%   serve as a summary of what will go to the 3 editors of the journals
%   that we are considering publishing a full paper in. Thus some more
%   information/discussion on what the underlying problem and context
%   will be about.}


% \jhanote{Once we have defined / introduced SAGA, we should probably
%   have 3 subsections -- interface, library and adatptors/backends?}

\section*{\normalsize Abstract}
SAGA is an API used to build distributed applications, abstractions, tools and frameworks. 
It comes with a package of specifications which can be utilized for writing distributed software programs. 
In understanding SAGA, like any other API packages, installations form the very first basic step that leads 
to exploring its various capabilities. Most often than not packages as such require to have one or more than 
one dependency packages installed and SAGA is no exception to it. Manual installations of such packages 
and keeping track of them may become cumbersome for any user alike and any automated installation framework 
may very well become handy and ease the installation procedure for everyone. This basically enables any user 
to skip each and every step of the installation process. The paper presented here discusses a perl script called 'mephisto' 
that automates SAGA installations developed by the SAGA group in this case. A Main advantage seen to this script is that it is
flexible in its usage and can be generalized to any software package installations. The paper will proceed to explain more 
about mephisto and its functionalities, solutions existing prior to it and the additional advantages it brings in.
% \bibliographystyle{IEEEtran} 
%\bibliography{mephisto}


\end{document}

