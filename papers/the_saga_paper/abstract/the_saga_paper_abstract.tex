\documentclass[a4paper,10pt]{article}
\usepackage[utf8]{inputenc}
\usepackage{graphicx}
\usepackage{url}
\usepackage{float}
\usepackage{times}
\usepackage{multirow}
\usepackage{listings}
\usepackage{times}
\usepackage{paralist}
\usepackage{epsfig}
\usepackage{subfigure}
\usepackage[hypertex]{hyperref}
\usepackage{subfigure}
\usepackage{color}
\usepackage{xspace}

%\documentclass{rspublic}

\usepackage{ifpdf}

\newcommand{\I}[1]{\textit{#1}}
\newcommand{\B}[1]{\textbf{#1}}
\newcommand{\BI}[1]{\textbf{\textit{#1}}}
\newcommand{\T}[1]{\texttt{#1}}

\newcommand{\sagaspec}{\textit{SAGA}\xspace}
\newcommand{\sagaimpl}{\textit{SAGA}\xspace}

\newcommand{\spec}{\sagaspec}
\newcommand{\impl}{\sagaimpl}

\setlength\topmargin{0in}
\setlength\headheight{0in}
\setlength\headsep{0in}
\setlength\textheight{9.5in}
\setlength\textwidth{6.5in}
\setlength\oddsidemargin{0in}
\setlength\evensidemargin{0in}
\setlength\parindent{0.1in}
\setlength\parskip{0.25em}


\ifpdf
 \DeclareGraphicsExtensions{.pdf, .jpg}
\else
 \DeclareGraphicsExtensions{.eps, .ps}
\fi

\newcommand{\note}[1]{ {\textcolor{red} { ***NOTE: #1 }}}

\newif\ifdraft
\drafttrue

\ifdraft
\newcommand{\amnote}[1]{   {\textcolor{magenta} { ***Andre:    #1 }}}
\newcommand{\jhanote}[1]{  {\textcolor{red}     { ***Shantenu: #1 }}}
\else
\newcommand{\amnote}[1]{}
\newcommand{\jhanote}[1]{}
\fi

\begin{document}

 \title{ \Large \vspace{-3.5em} SAGA: Towards a Comprehensive Distributed Programming System }
 
 \author{Shantenu Jha$^{1,2,3}$, Hartmut Kaiser$^{1}$, Andre Merzky$^{1}$, Ole Weidner$^{1}$ \\
   \small{\emph{$^{1}$Center for Computation \& Technology, Louisiana State University, USA}}\\
   \small{\emph{$^{2}$Department of Computer Science, Louisiana State University, USA}}\\
   \small{\emph{$^{3}$e-Science Institute, University of Edinburgh, UK}}
 }
 \date{}
 \maketitle
 

% \jhanote{Remember in addition to serving as an abstract, this will
%   serve as a summary of what will go to the 3 editors of the journals
%   that we are considering publishing a full paper in. Thus some more
%   information/discussion on what the underlying problem and context
%   will be about.}


% \jhanote{Once we have defined / introduced SAGA, we should probably
%   have 3 subsections -- interface, library and adatptors/backends?}

\subsection*{Introduction}
\vspace{-0.6em}

 Large-scale distributed systems and applications have been the primary workhorse 
 for the computational sciences for more than two decades now. Originally rooted
 in the 1960s operating system research~\cite{519301}, distributed computing 
 quickly became one of the most popular and exciting area of both, theoretical and
 applied computer science research and climaxed, at least in terms of popularity,
 with the conceptualization and implementation of the World Wide Web in the early 1990s. 
 
 
 Today, the fundamental concepts of how to compose portable, reliable, fault-tolerant 
 and scalable distributed systems are very well understood and an important
 part of any modern computer science and informatics curriculum. Yet, especially 
 the scientific community still struggles to develop and deploy scientific 
 applications that implement these very characteristics. Why? Wouldn't one expect 
 distributed computing to have matured progressively from the theoretical concepts
 to powerful abstractions? From distributed mutual exclusion techniques to to
 communication patterns, to powerful and transparent application frameworks? 


 \bibliographystyle{IEEEtran} 
 \bibliography{the_saga_paper_abstract,saga_ogf}


\end{document}

