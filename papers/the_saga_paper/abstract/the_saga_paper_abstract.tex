\documentclass[a4paper,10pt]{article}
\usepackage[utf8]{inputenc}
\usepackage{graphicx}
\usepackage{url}
\usepackage{float}
\usepackage{times}
\usepackage{multirow}
\usepackage{listings}
\usepackage{times}
\usepackage{paralist}
\usepackage{epsfig}
\usepackage{subfigure}
\usepackage[hypertex]{hyperref}
\usepackage{subfigure}
\usepackage{color}
\usepackage{xspace}

%\documentclass{rspublic}

\usepackage{ifpdf}

\newcommand{\I}[1]{\textit{#1}}
\newcommand{\B}[1]{\textbf{#1}}
\newcommand{\BI}[1]{\textbf{\textit{#1}}}
\newcommand{\T}[1]{\texttt{#1}}

\newcommand{\sagaspec}{\textit{SAGA}\xspace}
\newcommand{\sagaimpl}{\textit{SAGA}\xspace}

\newcommand{\spec}{\sagaspec}
\newcommand{\impl}{\sagaimpl}

\setlength\topmargin{0in}
\setlength\headheight{0in}
\setlength\headsep{0in}
\setlength\textheight{9.5in}
\setlength\textwidth{6.5in}
\setlength\oddsidemargin{0in}
\setlength\evensidemargin{0in}
\setlength\parindent{0.1in}
\setlength\parskip{0.25em}


\ifpdf
 \DeclareGraphicsExtensions{.pdf, .jpg}
\else
 \DeclareGraphicsExtensions{.eps, .ps}
\fi

\newcommand{\note}[1]{ {\textcolor{red} { ***NOTE: #1 }}}

\newif\ifdraft
\drafttrue

\ifdraft
\newcommand{\amnote}[1]{   {\textcolor{magenta} { ***Andre:    #1 }}}
\newcommand{\jhanote}[1]{  {\textcolor{red}     { ***Shantenu: #1 }}}
\else
\newcommand{\amnote}[1]{}
\newcommand{\jhanote}[1]{}
\fi

\begin{document}

 \title{ \large \vspace{-3.5em} SAGA: Towards a Comprehensive Programming System for Distributed Scientific Applications }
 
 \author{\normalsize Ole Weidner$^{1}$, Shantenu Jha$^{1,2,3}$, Andre Merzky$^{1}$, Hartmut Kaiser$^{1}$ \\
   \small{\emph{$^{1}$Center for Computation \& Technology, Louisiana State University, USA}}\\
   \small{\emph{$^{2}$Department of Computer Science, Louisiana State University, USA}}\\
   \small{\emph{$^{3}$e-Science Institute, University of Edinburgh, UK}}
 }
 \date{}
 \maketitle
 

% \jhanote{Remember in addition to serving as an abstract, this will
%   serve as a summary of what will go to the 3 editors of the journals
%   that we are considering publishing a full paper in. Thus some more
%   information/discussion on what the underlying problem and context
%   will be about.}


% \jhanote{Once we have defined / introduced SAGA, we should probably
%   have 3 subsections -- interface, library and adatptors/backends?}

\subsection*{Abstract}
%\vspace{-0.6em}

Large-scale distributed systems playing an increasingly vital roll in numerous research and development
projects across many different scientific and engineering domains. Yet, we experience a non-equilibrium between the fast
growth of distributed infrastructure and the scientific applications that can exploit the full
potential of these systems. 
In this article we give and in-depth introduction to the Simple API for Grid Applications (SAGA), both
as an emerging interface standards as well as a comprehensive programming system as an effective way to
develop distributed scientific applications, tools and frameworks. We discuss how SAGA's standardized
interface, modular runtime architecture and growing list of supported distributed systems middleware
have been used to develop real-life systems that are capable of overcoming current infrastructure 
barriers and that give us valuable insights on how to architect next-generation distributed applications.

\subsection*{Introduction}
%\vspace{-0.6em}

 Originally rooted in the 1960s operating system research, distributed computing 
 quickly became a popular and diverse area of research in both, theoretical and applied 
 computer science. Based on the fundamental principle of communicating autonomous processes, 
 the study of distributed systems and applications systematically explored and extended the 
 fields of algorithms, programming models and languages and rapidly grew to a full stack of 
 patterns, concepts and methodologies (see e.g. \cite{519301}) which fostered the development 
 of more and more complex systems and applications. This development arguably climaxed in the  
 conceptualization and implementation of the Internet and World Wide Web~\cite{Berners-lee92world-wideweb} 
 in the early 1990s as the foundation and prototype for all modern large-scale, globally-distributed 
 systems. Quickly adopted by the scientific community, large-scale distributed systems have 
 been the primary workhorse for the computational sciences for more than two decades now 
 with many of today's systems consisting of a vast sets of globally interconnected resources 
 and services, forged to provide researchers with dedicated high-bandwidth communication 
 channels, hundreds of thousands of CPU cores and petabytes of storage.
 
 Yet there is a noticeable non-equilibrium between the progress that is made in providing
 even larger distributed systems and the distributed applications that are capable of
 using the full potential of these systems. Although the fundamental concepts of how to 
 compose portable, reliable, scalable distributed applications are very well understood and an 
 important part of any modern computer science and informatics curriculum, 
 the scientific community still struggles to develop and deploy large-scale distributed applications.
 In order to better understand the current dilemma, it is important to highlight an emerging paradigm 
 shift that affects they way large-scale distributed systems are being used: while in the the past 
 systems were distributed, but mostly self-contained, like for example campus grids or \textit{@Home}   
 infrastructures, todays system landscape emphasizes on global distributed \textit{meta-systems}. 
 National and international efforts and collaborations are trying to bring together these formerly
 self-contained distributed systems to provide a global scientific community with an unprecedented 
 network of resources. Administrative and technical borders are torn down to allow applications 
 to tap into this pool and application scientists are studying the characteristics of these new 
 meta-systems with the goal to apply already known as well as to develop novel concepts and 
 patterns that can support a new generation of \textit{embarrassingly-distributed} applications.
 
 But what appears to be simple or at least as a straight-forward research agenda at a first glance, 
 quickly turns into a Herculean Challenge on the technical level. A look at the recent history of 
 large-scale distributed computing and the status quo of the technologies provides one possible
 explanation: although
 the majority of the community has been agreeing on what a large-scale distributed system should
 be early on, the implementation landscape couldn't look more diverse. Traditionally 
 driven by strong political and personal agendas, system implementations in the 1990s and 2000s 
 quickly diverged into a plethora of massive, more and more sophisticated middleware stacks.
 And although the ... which lead to zero cross-site-portability of application code.
 
 
{\bf [...] }
 
 
 
 With the \textit{Simple API for Grid Applications} (SAGA) we present a programming interface and
 runtime system for large-scale distributed systems that has been evolved from a mature, community 
 driven standard. 
 
 
 
 \section{The API Standard}

 \subsection{Unified Syntax and Semantics}
 
 \subsection{Functional Packages}

 \subsection{Community Uptake}
 
 \subsection{Implementations}

 \section{The C++ Reference Implementation}
 
 \subsection{Design Philosophy}

 \subsection{Core Components}

 \subsubsection{Application Programming Interface}

 \subsubsection{Runtime System}

 \subsection{Middleware Adaptors}

 \subsection{Performance Aspects}

 \subsubsection{Core Components}

 \subsubsection{Middleware Adaptors}

 \section{Applications}

 \subsection{Frameworks}


 \section{Acknowledgements}
 
 
 

 \bibliographystyle{IEEEtran} 
 \bibliography{the_saga_paper_abstract,saga_ogf}


\end{document}

