\section{The SAGA Runtime System}

\newcommand{\thesagaimpl}{\textcolor{red}{WhateverWeWantToCallIt} }

\newcommand{\forhartmut}[1]{{\textcolor{magenta}{\textbf{For Hartmut:} #1 }}}
\newcommand{\forole}[1]{{\textcolor{green}{\textbf{For Ole:} #1 }}}
\newcommand{\foroleandhartmut}[1]{{\textcolor{blue}{\textbf{For Ole \& Hartmut:} #1 }}}

  This section describes the design and implementation of a dynamic runtime
  system that implements the SAGA interface described in the previous section.
  \thesagaimpl is a production-grade system that combines the concepts of SAGA
  with a versatile runtime system and plug-in mechanism in order to provide a
  practical foundation for the development of distributed applications.

  %%-------------------------------------------------------------------------%%
  \subsection{Design Assumptions} 
  \forole{What are the underlying assumptions that we made when we started to
  implement saga? Lay out landscape of production cyber-infrastructure! }


  %%-------------------------------------------------------------------------%%
  \subsection{Design Objectives}
  \forole{ Capabilities that must be supported - in terms of performance,
  capabilities, scalability, interoperability, ...}

  %%-------------------------------------------------------------------------%%
  \subsection{Design Philosophy}
  \forhartmut{ PIMPL, etc., How it is Split up Late Binding, Why Boost?}

  %%-------------------------------------------------------------------------%%
  \subsection{Architecture}
  
  \forole{This is were the BIG PICTURE goes and where we introduce concepts and
  terminologies, like "adaptor" "core", etc..}


  %%%%---------------------------------------------------------------------%%%%
  \subsubsection{User Interface Layer (API)}
  \foroleandhartmut{What is this and how does it work?}
 
 
  %%%%---------------------------------------------------------------------%%%%
  \subsubsection{Adaptor Interface Layer (CPI)}
  \foroleandhartmut{Needs rewrite!}\\
  \forhartmut{instancedata, adaptordata, ... also good for cpidoc.saga.. ;-)}\\
  The CPI acts as both,
  a logical and a physical barrier between the SAGA Core and the Adaptors:
  logical because of a well defined programming interface and physical because
  it enforces the encapsulation of adaptor code in separate shared libraries.
  The CPI reflects the same class hierarchy and method names as the API. This
  is intuitive, since the adaptors are supposed to translate API calls
  one-to-one, to the underlying middleware systems. However, at the CPI level,
  the interface has been enriched with additional data structures that are not
  visible at the API level, as well as with a set of convenience functions and
  macros that can help the adaptor developer reduce the implementation overhead
  for parsing configuration files, logging, error handling and other repetitive
  and redundant tasks. Figure~\ref{fig:cpi-detail} provides deeper insight into
  SAGA's architecture at the CPI level and how adaptors interface with it. Even
  though this specific example focuses on the \texttt{job} package, it is
  representative for any other functional package in SAGA.
 
 
  %%%%---------------------------------------------------------------------%%%%
  \subsubsection{Core Runtime System}
  \foroleandhartmut{Adaptor selection, logging, mechanisms, error handling,... }


  %%%%---------------------------------------------------------------------%%%%
  \subsubsection{Python API Wrapper}
  \forole{What is this and how does it work?}\\
  \forhartmut{How and why do we use boost::python?}
  

  %%-------------------------------------------------------------------------%%
  \subsection{Middleware Adaptors (Plug-Ins)}
  \forole{Needs rewrite!}\\
   One of the most important design porperties in SAGA is its vertical
   extensibility at the so-called CPI (Capability Provider Interface) level.
   The CPI defines the boundary between the end-user \jhanote{at some point in
   the paper, relatively upstream, we'll have to define who in our model of an
   end-user is?} API and runtime (\textit{SAGA Core}) and the middleware
   bindings (\textit{SAGA Adaptors)}.  It allows system integrators to develop
   SAGA bindings for any underlying system without having to touch the SAGA
   Core implementation.  This concept fosters community involvement by
   providing a plug-in mechanism at the \jhanote{slightly awkward construct the
   next one:} system-integrator-level, but it also greatly contributes to the
   overall stability and robustness of the SAGA Core implementation by
   completely decoupling API and runtime layer from platform-specific
   implementation details. The effectiveness of the adaptor plug-in concept is
   certainly reflected in the growing list of middleware adaptors that have
   been developed to support a broad spectrum of distributed technologies. The
   following sections will introduce some of the concepts and details of the
   CPI as well as try to give a short, yet comprehensive overview of
   distributed computing technologies and systems supported by SAGA through
   middleware adaptors.

 
  %%%%---------------------------------------------------------------------%%%%
  \subsubsection{Adaptors for Task Execution and Management} 
  The various job adaptors obviously \textbf{O.W}


  %%%%---------------------------------------------------------------------%%%%
  \subsubsection{Adaptors for Data Access and Management} 
  \forole{Needs rewrite!}\\
  In the wake of a new important class of distributed applications that focus
  on complex - often sensor-driven - data-intensive scientific workflows, data
  management and access capabilities are becoming incresingly important for
  application developers. Although SAGA does not provide a high-level
  abstraction for \textit{data} per se, it provides several functional packages
  that allow and simplify low-level data access. These package APIs can easily
  be used to develop a data abstraction framework on-top-of SAGA. The following
  list gives an overview of functional packages and adaptors that support data
  access and management:
    
    \begin{itemize}
    
    \item \textbf{File Package} - \onote{Short Description}
    
    \begin{itemize}
    \item Globus GridFTP File Adaptor
    \item HDFS File Adaptor
    \item Local File Adaptor
    \item SSH File Adaptor
    
    \end{itemize}
    
    \item \textbf{Advert Package} - \onote{Short Description}
    
    \begin{itemize}
    \item PostgreSQL Advert Adaptor
    \item SQLite3 Advert Adaptor
    \end{itemize}
    
    \item \textbf{Replica Package} - \onote{Short Description}
    
    \begin{itemize}
    \item PostgreSQL/SQLite3 Replica Adaptor
    \item Globus RLS Replica Adaptor
    \end{itemize}
    
    \item \textbf{Stream Package}
    
    \begin{itemize}
    \item TCP Socket Stream Adaptor
    \end{itemize}
    
    \end{itemize}
 

  %%%%---------------------------------------------------------------------%%%%
   \subsubsection{More Adaptors}
   \forole{Whatever else is worth mentioning: streams, sd, ...}


  %%-------------------------------------------------------------------------%%
  \subsection{Measurements and Performance}
  \forole{T.B.D. (Look at old Globus adaptor benchmarks, etc..)}
  
  %%-------------------------------------------------------------------------%%
  \subsection{Development Process}
  \onote{The more I think about it the more I'm convinced that this should NOT
  go into the RUNTIME SYSTEM ARCHITECTURE section. It should be discussed if
  this should go into an architecture/systems paper at all}
  
  Software architecture and development process and how it is designed to
  encourage and facilitate community contribution
    
  \jhanote{Ideally this subsection will describe how SAGA does or does not
  address many/all elements of error handling in a distributed environment}

  \subsection{Regression Testing} \jhanote{We need a subsection on regression
  testing. Can't have a paper on production grade software without some
  description of testing process, and possibly also release/packaging (though
  far less important)}
  
  \subsection{Package and Delivery} (AM)
