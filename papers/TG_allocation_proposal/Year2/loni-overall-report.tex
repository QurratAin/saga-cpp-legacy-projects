\documentclass[a4paper,11pt]{article}

%\usepackage[utf8]{inputenc}
\usepackage{graphicx}
\usepackage{url}
\usepackage{float}
\usepackage{times}
\usepackage{multirow}
\usepackage{listings}
\usepackage{times}
\usepackage{paralist}
\usepackage{epsfig}
\usepackage{subfigure}
\usepackage[hypertex]{hyperref}
\usepackage{subfigure}
\usepackage{color}
\usepackage{ifpdf}
\usepackage{wrapfig}

\usepackage{texdraw}
\usepackage{epsf}
\usepackage{array}
\usepackage{cite}
\usepackage{enumitem}
\usepackage{verbatim}
\usepackage{setspace}
\sloppy
\usepackage{geometry}



\newcommand{\I}[1]{\textit{#1}}
\newcommand{\B}[1]{\textbf{#1}}
\newcommand{\BI}[1]{\textbf{\textit{#1}}}
\newcommand{\T}[1]{\texttt{#1}}
\newcommand{\dctf}{dC$_{25}$ }
\newcommand{\dctfnsp}{dC$_{25}$}
\newcommand{\atf}{A$_{25}$ }
\newcommand{\dco}{dC$_{1}$ }
\newcommand{\atfnsp}{A$_{25}$}
\newcommand{\dconsp}{dC$_{1}$}
\newcommand{\aonsp}{A$_{1}$}
\newcommand{\ao}{A$_{1}$ }
\newcommand{\ato}{A$_{1}$ }
\newcommand{\ahl}{$\alpha$HL }
\newcommand{\ahlnsp}{$\alpha$HL}
\newcommand{\prim}{$^{\prime}$ }
\newcommand{\primnsp}{$^{\prime}$}


\pdfpagewidth 8.5in
\pdfpageheight 11in 

\setlength\topmargin{0in}
\setlength\headheight{0in}
\setlength\headsep{0in}
\setlength\textheight{9in}
\setlength\textwidth{6.5in}
\setlength\oddsidemargin{0in}
\setlength\evensidemargin{0in}
\setlength\parindent{0.1in}
\setlength\parskip{0.25em}

\ifpdf
 \DeclareGraphicsExtensions{.pdf, .jpg, .png}
\else
 \DeclareGraphicsExtensions{.eps, .ps}
\fi

\newcommand{\up}{\vspace*{-1em}}
\newcommand{\upp}{\vspace*{-0.5em}}
\newcommand{\tc}{$T_c$ }
\newcommand{\ttc} {TTC}
\newcommand{\tcc}{TCC}

\newif\ifdraft
\drafttrue
\ifdraft
\newcommand{\amnote}[1]{ {\textcolor{magenta} { ***AM: #1c }}}
\newcommand{\jhanote}[1]{ {\textcolor{red} { ***SJ: #1 }}}
\newcommand{\michaelnote}[1]{ {\textcolor{blue} { ***MM: #1 }}}
\newcommand{\yyenote}[1]{ {\textcolor{green} { ***YYE: #1 }}}
\else
\newcommand{\amnote}[1]{}
\newcommand{\jhanote}[1]{}
\newcommand{\michaelnote}[1]{ {\textcolor{blue} { ***MM: #1 }}}
\newcommand{\yyenote}[1]{ {}}
\fi



\begin{document}

\title{\Large Large-Scale Atomistic MD Simulations and associated
  Cyberinfrastructure using LONI (and XSEDE) Resources}

\date{}

%\maketitle
 
\section*{Summary: Large-Scale Atomistic MD Simulations and associated Cyberinfrastructure using LONI (and XSEDE) Resources}

\section{Understanding non-coding Functional RNAs: 
Folding Dynamics and Binding mechanism of Riboswitches}

\subsection{Project Progress}
In order to fully understand the coupling between the ligand binding and the folding of riboswitch RNAs, we adopt all atomic simulations to explore this linkage.  We have been engaged in seeking physical insights into novel ideas that elucidate the interplay between the Aptamer domain and the Expression domain; this has not been explored in previous studies.  In particular, we are using very long time-scales ($>$ 100ns) all-atom multiple MD simulations/trajectories, to seek an understanding of the role of the branch migration during a dynamical transition toward the OFF state of S-adenosyl Methione (SAM) binding riboswitch (SAM-I riboswitch).  Based upon interesting phenomenon that has been observed in trajectories generated on Ranger, we have developed new analysis to monitor the trajectory on the fly. Additionally, these data also provide us the basis of choosing a system to be submitted for tens of us (microsecond) time scale on Anton, a machine specially designed for running MD simulations. This work was carried out by Wei Huang a final year PhD (graduation Dec 2011) student co-supervised by the PI. 

We need to extend this to cover multiple distinct starting configuration; the need for multiple trajectories arises because we have to simulate various initial structures that are sampled by 3D modeling and thus cover the conformational space appropriately.  In addition, these multiple all-atom MD simulations should be conducted with SAM-bound and SAM-free states, since the role of SAM could be understood by comparing trajectories of two different states.

Initial work in this project has been accepted as a Special Issue of
Concurrency and Computing: Practice and Experience (CCPE) for Emerging
Methods for the Life Sciences~\cite{ecmls_ccpe10}.  Several other
publications are in very advanced stages of publication (see
\url{http://cct.lsu.edu/~sjha/select_publications/in-prep/}).

The next round of simulations, should provide conclusive evidence of a potential role of the SAM in facilitating P1 helix formation over the AT helix formation, which will eventually clarify whether the proposed branch migration mechanism as a major switching pathway between two alternative secondary structures of SAM-I riboswitch.

As part of this stage of the project we developed a submission tool based on SAGA, which amongst other things allowed us to coordinate simulation runs on machines on both the XSEDE (Ranger) and the EU Distributed European Infrastructure for Supercomputing Applications (DEISA/PRACE)~\cite{DEISA-PRACE}. The submission tool has been completed as part of our previous allocation: TG-MCB090174 and has been extended to include a gateway system: DARE~\cite{dare-tg11} and infrastructure to support submission through Unicore, Genesis II (see \url{http://www.saga-project.org/interop-demos}) and Cloud systems~\cite{pstar11}.


\section{Atomistic Simulations of Physiological Systems}

The long term scientific objective of our project is to develop molecular dynamics simulations of medically relevant enzymes into a tool for clinicians to use in determining the cocktail of drugs to administer to an HIV-infected individual. This work is supported by grants under EU FP7 and FP6 via the VPH-NOE (EU FP7-ICT-2007-5.3 223920), Contra Cancrum (EU FP7-ICT-2007-5.3 223979), p-Medicine (EU FP7-ICT-2009-6 270089) and CHAIN (EU FP7 HEALTH-2007-2.3.2-7) projects. For such applications, reproducible accuracy at the level which can rank drug efficacies, and rapidity of acquisition of results (for clinical relevance) are all essential. This takes the application of bio-MD techniques into an entirely new domain. This project is divided into two distinct sub--projects: (i) Patient specific HIV therapy and (ii) Predicting the affinity of the EGFR kinase domain for drug inhibitors of lung cancer.

\subsection{Subproject: Towards Patient Specific HIV Therapy}

\subsubsection{Subproject Progress}
The long term scientific objective of our project is to develop molecular dynamics simulations of HIV-1 Pol enzymes into a tool for clinicians to use in determining the cocktail of drugs to administer to an HIV infected individual. We have recently completed a study which applies our free energy calculation protocol~\cite{Ref2} to a patient derived HIV-1 protease (PR) sequence and the drug lopinavir, identified as producing ambiguous resistance rankings from currently used clinical decision support tools~\cite{Ref3}. This study has suggested a potential new mechanism for drug resistance. We have completed studies of the protonation state of the protease catalytic dyad when bound to all FDA approved HIV-1 PR inhibitors. This is a prerequisite for the application of our protocol to these drugs. With a further allocation we would look to extend this work to compare wildtype and known resistant mutants for each drug.

We have also extended our protocol to investigate the binding of drugs to HIV-1 RT. In this system we have identified the impact of large scale protein motions, distant from the binding site, on the binding free energy~\cite{Ref4}.



In addition to this progress on the originally proposed simulations, we also extended our previous work evaluating the binding affinity of HIV-1 protease mutants to the inhibitor lopinavir. As part of a previous collaboration in the EU Virolab project (EU FP7 223131) a comparative drug ranking methodology was used to compare drug resistance rankings produced by the Stanford HIVdb, ANRS and RegaDB clinical decision support systems. The methodology was used to identify a patient sequence for which the three rival online tools produced differing resistance rankings. This process identified mutations at only three positions (L10I, A71IV and L90M) which influenced the resistance level assigned to the sequence. We have simulated not only the full patient sequences but also systems containing the constituent mutations (a total of 12 sequence variants were simulated).  Inserting any combination of the identified mutations into the wildtype sequence produced no impact on the binding affinity of the protease for lopinavir. In contrast when the mutations were inserted into the background sequence present in the patient derived sequence resistance was observed. Our simulations also identified changes in the relative conformation of the two beta sheets that form the protease dimer interface which suggest an explanation of the relative frequency of different amino acids observed in patients at residue 71. 

Simulations of the HIV-1 reverse transcriptase bound to the inhibitor efavirenz (EFZ) have also been performed. The use of an ensemble approach has revealed that previous single trajectory results which allowed the discrimination of wildtype, K103N and L100I/K103N were fortuitous. Consequently we performed simulations of only the wildtype, K103N, L100I and L100I/K103N sequences rather than the more extensive range of variants proposed. Our results suggest that we need to adapt our protocol here to both include more replicas and perhaps to simulate the apo enzyme as well as the drug bound form.  Simulating the apo form should allow us to evaluate the energetic cost of binding pocket formation for each sequence. Simulation conducted as part of this study have shown that experimental differences in binding affinity between EFZ and another drug (nevirapine, NVP) can consistently be reproduced.




\section{ Standards-based Interoperable, Extensible and Scalable Cyberinfrastructure for XSEDE}
In conjunction with the scientific questions we are addressing, we are also involved in the development of a distributed runtime execution system (DARE) that uses the Simple API for Grid Applications (SAGA: http://saga.cct.lsu.edu/)
and higher-level abstractions such as the SAGA-based Pilot-Job (BigJob).

DARE forms the basis of an entire class of Gateways that effectively utilize XSEDE, OSG, LONI and EGI (European Grid Resources) resources. See \url{http://dare.cct.lsu.edu}, for a range of application types.

SAGA-BigJob -- an interoperable Pilot-Jobhas allowed us to run and coordinate hundreds, if not thousands of large-scale ensembles across resources, both on XSEDE and the EU DEISA network, as part of the NSF-HPCOPS funded Interoperability Project~\cite{tg-vph-interop}.  

Significant progress has recently been achieved in enabling the use of SAGA and SAGA-based DARE capabilities to interoperate between different resources -- whether they be different resources on XSEDE or two different grids.  A key goal of an extended allocation would be to further develop the standards-based cyberinfrastructure for XSEDE and assess performance using real scientific workloads, and make it available for the broader \& larger community of biomolecular simulators.

%\bibliographystyle{IEEEtran}

\bibliographystyle{unsrt}
\bibliography{saga,Supp,jha_loni_alloc_jul01,ucl_trac,yye00}

\end{document}
