
\textbf{Professor B.M. Boghosian}'s research interests center around the mathematical and computational modelling and analysis of nonlinear dynamical systems, with particular emphasis on hydrodynamics, multiphase and amphiphilic fluid flow, and kinetic theory. Professor Boghosian and Professor P.V. Coveney of University College London (UCL) have previously jointly held NSF PACI (ASC030006P) and NRAC grants (MCA04N014) for supercomputing time at the Pittsburgh Supercomputing Center (which underpinned the US involvement in the TeraGyroid project). Since 2000,  Professor Boghosian has been a faculty member at Tufts University, where he holds the position of Professor in the Department of Mathematics and is currently Chair (2006-2010), and Adjunct Professor in the Department of Computer Science.  Professor Boghosian was an EPSRC Visiting Fellow within the RealityGrid project at University College London between 2002-2005. Professor Boghosian is a Fellow of the American Physical Society. He is a member of the editorial boards of \emph{Physica A}, of \emph{Computers in Science and 
Engineering} and of the \emph{International Journal of Modern Physics C: Physics and Computers}. He is the recipient of an NSF cyberinfrastructure award for the acquisition of a state-of-the-art scientific visualization facility.
He is also a recipient of the TeraGrid 5K Club award for the HYPO4D code at TeraGrid'08.

\textbf{Professor P.V. Coveney} holds a Chair in Physical Chemistry and is Director of the Centre for Computational Science (CCS) within the Department of Chemistry at UCL. He holds an Honorary Professorship in Computer Science, also at UCL. He is the Principal Investigator of the \$12M EU FP7 Virtual Physiological Human Network of Excellence. He led the UK/US TeraGyroid (RealityGrid grant GR/R67699) and SPICE (NSF Grant Numbers CA SCI-0525308 and CSA SCI-0438712 and EPSRC Grant EP/D500028/1) projects funded by EPSRC and NSF, and is the recipient of an HPC Challenge Award at Supercomputing 2003, an International Supercomputing Conference Award in 2004, an inaugral HPC Analytics Challenge Award at Supercomputing 2005 and an International Supercomputing Conference Life Sciences Award 2006. He is the recipient of a Transformational Science Challenge award as well as a recipient of two TeraGrid 5K Club awards for the codes HYPO4D and LB3D at TeraGrid'08. Professor Coveney is currently Chairman of the UK Collaborative Computational Projects (CCP) Steering Panel. He is the editor of the first single volume publication on scientific grid computing, published by \emph{Philosophical Transactions of the Royal Society A} (2005). He is a founding editor (with Peter Sloot and Jack Dongarra) of the Elsevier \emph{Journal of Computational Science} which will appear for the first time in 2010. Coveney and Boghosian have collaborated extensively on lattice-Boltzmann and molecular dynamics research for more than fifteen years and have co-authored twenty-five papers. Coveney and Jha have collaborated over seven years in the field of molecular dynamics, high performance, distributed \& grid computing and have co-authored 12 papers in these areas.

\textbf{Professor Shantenu Jha} is the Director of Cyberinfrastructure Development at the Center for Computation and Technology (CCT), and a Research Professor in Computer Science at Louisiana State University (LSU). He is also a theme-leader at the e-Science Institute, Edinburgh and a Visiting Scientist at CCS, UCL. His research interests lie at the triple point of Computational Science, Cyberinfrastructure Development and Computer Science.  His Research is supported by a broad range of sources, including UK-EPSRC, US-NSF and NIH as well with SUN and Google (GSOC). He is a co-PI on a \$2.6M NSF award that enables LSU/LONI participation as a TeraGrid Resource Provider. Shantenu leads the SAGA project, and leads Work Package 4 of the NSF Funded Cybertools Project (http://www.cybertools.org) (NSF Award; Total Value \$12M). Professor Jha is also the co-PI of LSU/LONI's HPCOPS NSF award, “Joining the TeraGrid”, member of the Scientific Board of the LONI Institute. 
Professor Jha is also the PI of multiple Louisiana Board of Regents award and holds an LSU  Faculty Award. Integration of SAGA with applications is part of Cybertools and the PI also holds multiple peer-reviewed awards for the development and integration of SAGA. The Interoperability Project is currently funded by an NSF HPCOPS award, and is being executed by Professor Jha.

\textbf{Dr Andrew Sherman}
Dr. Sherman provides HPC support via the Yale Provost's Office for
faculty and students, and is himself an expert on linear algebra
methods in high performance computing. The present TeraGrid TRAC
proposal contains two areas in which we will be collaborating with
colleagues at Yale. These include biomedical research areas concerned
with neurovascular modelling and stimulation, using HemeLB and patient
specific HIV therapy based on rapid and accurate determination of
protease-inhibitor binding affinities; and also in the determination
of unstable periodic orbits based on the HYPO4D package. In both these
areas, Dr. Sherman will lead and co-ordinate the participation of
other Yale colleagues. He will contribute to our effort to improve the
pre-conditioning of the variational relaxation procedure in order to
enhance numerical convergence on candidate UPOs.  On 8 October 2009,
the Presidents of Yale \& UCL signed a five-year collaboration
agreement designed to provide for a deepening alliance§ between the
two universities. In its initial phases, currently underway, this will
focus on advancing biomedical research and healthcare on a global
scale, through joint scientific research, clinical and educational
effort. As the alliance widens, it is anticipated that more academic
areas will become part of this formal agreement.

One central area of immediate and longer term concern is sharing of
computational resources and facilities. The UCL-Yale consortium aims
to conduct international science through sharing of resources on both
sides of the Atlantic. The current proposal will help us to build
joint computational effort in relevant research areas, with reciprocal
resources being made available in UK and EU via DEISA and related
resources. In addition, as part of their alliance, Yale and UCL are
exploring ways to link each others' sites via an international high
speed network.

\begin{comment}
Yale \& UCL are planning to link each others' sites via an
international high speed network in 2010, funding for which will come
from the two institutions, UK Government and EU (as part of the ELIXIR
project which will develop sustainable data infrastructures across
Europe). JA.NET (the UK's education and research network,
http://www.ja.net) will provide the 10Gb/s dedicated connectivity via
their LightPath (http://www.ja.net/services/lightpath) link
procurements to DANTE (DANTE Delivery of Advanced Network Technology
to Europe; http://www.dante.net), crossing under the Atlantic to
Internet2 and hence on to Yale \& TeraGrid.
\end{comment}

\subsection{Associated Technical Developments, Collaborations \& Other High
Performance Computing Resources}

We plan to release the next version of our grid middleware Application Hosting Environment (AHE) in the second half of 2010 \cite{zasada2009,coveney2007}, during the proposed TRAC allocation.  The AHE provides simple graphical and command line interfaces to run applications on resources provided by grid computing infrastructures in addition to local campus-based clusters, while hiding from the user details of the underlying middleware in use by the distinct resource providers.  We are engaged in integrating AHE with a metascheduling framework, based on a computational mechanism design, which is mediated by software agents, to efficiently allocate work between a set of resources based on cost minimization and run time optimization.  The patient-specific clinical simulation work proposed in this TRAC will be supported by our plans to integrate support for SPRUCE (a system developed at Argonne National Labs for urgent computing by allowing users to run emergency jobs) within the AHE client interface, so that jobs submitted with SPRUCE tokens use the mechanisms provided by SPRUCE to preempt the current workload on a machine. We also plan to introduce a mechanism to host workflows in the AHE as virtualized applications, composed by orchestrating the execution of other AHE hosted applications. The purpose of workflow management systems is to automate common time consuming tasks that the scientist carries out when performing \emph{in silico} studies. With the planned developments, we can interact with workflows involving simulation pre-processing, launch and post-processing via the AHE in the same way as with a single application. This work is being done in collaboration with developers of the GSEngine workflow management tool within the European Union's (EU) Virtual Physiological Human (VPH) project.

In addition to the requested TeraGrid allocation, we currently have an allocation of 4.04 million Allocation Units (AUs) on the UK's Cray XT4 supercomputing resource associated with the EPSRC grant (EP/F00521/1) entitled ``Large scale lattice-Boltzmann simulation of liquid crystals'' for materials science research. Within the EU VPH Virtual Community project, we annually receive an allocation of 2.0M CPU hours on the EU's DEISA grid for our work in the bio-medical sciences domain.  (This computational allocation has been awarded to Coveney (PI); award number EU FP7-ICT-2007-5.3 223920 funded by the European Union with a start date of 06/01/2008 and expiration date of 11/30/2012).  We have been successful in the DEISA Extreme Computing Initiative (DECI) calls successively for the past several years, most recently gaining 700,000 CPU hours for our computational science activities.  (The most recent DECI computational allocation was awarded to Coveney (PI); award number EU RI-222919 funded by the European Union with a start date of 01/01/2009 and expiration date of 01/31/2010).  Use of the AHE and software from the SAGA project has allowed our research team to inter-operate between high performance computing resources in the US, UK and the EU.  Additionally, within our grid middleware development activities, we are working on
a new mechanism, termed as Audited Credential Delegation, to allow controlled access to a single grid certificate for a group of users, authenticated by their local institutional security credentials, facilitated by AHE. This approach is being developed to address issues related to the scalability of the current X.509 certificate-based authentication model, which relies on each individual user obtaining and managing his/her own grid certificates. This project is being funded by the UK EPSRC grant (EP/D051754/1) entitled ``User-Friendly Authentication and Authorisation for Grid Environments''.

\begin{comment}	
\emph{We need to ensure we address point raised last time, by citing 
reciprocal activities in UK/EU, and equivalnet utilization of DEISA REOSOURCEes.}
We have an allocation of X million AUs on UK's Cray XT4 supercomputer associated
with the UK EPSRC Grant(LSLBLC NO.) which
will support the projects within the bio-materials science domain. We plan
to perform complementary activities on EU's DEISA grid 
in the biomedicine domain, where the work is
supported by EU FP6 and FP7 funding and the new UCL-Yale initiative. 
We have been successful in the DEISA Extreme Computing Initiative (DECI)
calls successively for the past two years in gaining access to some of the
biggest supercomputers in the EU. Use of SAGA and AHE 3.0 will allow us to
seamlessly inter-operate between high performance computing resources in the
US and EU.
\end{comment} 
%dan katz cyber infrastructure. \newline\newline
%transatlantic lightpaths bbrsc nih \newline\newline
%ucl-yale initiative  
