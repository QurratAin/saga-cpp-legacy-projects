Previously we have used TeraGrid resources to conduct a binding affinity study of the protease inhibitor saquinavir with G48V and L90M containing mutatants which showed good agreement with experiment \cite{Stoica2008}. % uring SuperComputing 2007 we used
Using Lonestar (TG allocation number TG-ASC070019T) to expand this work by running 10ns-long simulations on 5 different HIV protease systems bound to 5 different FDA-approved drugs: amprenavir, indinavir, lopinavir, ritonavir, and saquinavir. We compared our binding energy results for each of the 5 protease genotypes in the MDR set to experimental values published in \cite{Ohtaka2003}. Our results showed varying correlations to experimental trends, with the 5 ritonavir systems showing good correlation to experimental whilst the 5 saquinavir systems showed poor correlation. In light of these results further work was done both on Lonestar (TG allocation numbers TG-ASC070019T and TG-DMR070014N) and more recently Ranger (TG allocation numbers A-rfUser1 and TG-ASC090009) to determine whether repeating the study as an ensemble of MD models would improve correlations. The results showed significantly improved correlations and have led us to refine our simulation protocol and reparameterise a number of inhibitors. An article focussed primarily on our work on the lopinavir bound system has recently been accepted for publication \cite{Sadiq2010}.

In addition to this we have performed a study into the comparative dynamics of different liganded forms of reverse transcriptase, concluding that the binding of the inhibitors efavirenz and nevirapine alter the motions of the catalytically important fingers domain of RT and that these motions are uncorrelated to fluctuations in the drug binding energy (as yet unpublished). We also performed preliminary investigations into the efficacy of the ensemble methodology for calculating binding affinities on NVP bound systems, again concluding that the convergence properties are improved over single trajectory. These results also allowed us to differentiate binding affinity of the L100I and L100I/K103N sequences from the wildtype using simulations of the ligand bound enzyme alone which indicates that they do not significantly affect the creation of the binding pocket. All this work was performed using Ranger. 

These simulations were orchestrated using our automated simulation tool, the Binding Affinity Calculator (BAC), which makes use of AHE \cite{coveney2007,zasada2009} to deploy and run simulations and then retrieve data. Two papers describing this tool and its potential use as a clinical support tool have been published\cite{Sadiq2008, Sadiq2008a}. The BAC allows us to distribute the simulation and analysis sections of our workflow over a number of geographically disparate locations. 
% The mutation and hydration of crystal structure based models takes place on local resources at UCL, using the AMBER software suite, while the actual simulations can be performed in NAMD either on TeraGrid or UKNational Grid Service (NGS) computers and/or EU DEISA grid. The final analysis of the resultant trajectories, on local recources and the NGS or both, is performed using the AMBER MD suite of codes. 
In the last year, BAC has been extended to allow the simulation of the epidermal growth factor receptor (EGFR) and the molecular dynamics code GROMACS \cite{Hess2008}.  This new functionality has facilitated a study of the binding of three inhibitors (AEE788, AFN941 and getfitinib) bound to EGFR, performed on Ranger.

In the area of bio-minerals our previous simulations have utilized periodic boundaries on the clay sheets, allowing a simulation cell to represent an infinite clay platelet \cite{JPCC_2007,Thyveetil,Thyveetil_JACS, Soft_Matter1, Ratcliffe2009, Anderson2010}. These large-scale, fully atomistic simulations, approached the size of a physically realistic platelet. From this, we were able to calculate mesoscopic and macroscopic properties directly from molecular dynamics simulations in the absence of finite-size effects of both clay nanocomposites\cite{JPCC_2007,Thyveetil, Soft_Matter1} and bio-composites~\cite{Thyveetil_JACS}.  Using resources from TG-ASC070019T we extended these simulations to calculate the mechanical response of poly(ethylene oxide) polymer-clay nanocomposites, separating the response into contributions from the polymer and clay mineral layer~\cite{Soft_Matter1}. This separation technique allowed us to determine how the clay-polymer elastic properties change with distance from the clay surface. This is the first time the effect of mineral layers on the elastic modulus of polymeric materials in the vicinity of a mineral surface has been calculated. This result is a vital first step to understanding the enhancement mechanism of nanocomposites and the role of the very large surface area of the clay mineral layer on the surrounding medium.  We have also examined the mechanisms by which clay mineral layers buckle in clay-polymer nanocomposites under compressive stress. We find that a clay sheet remains stable in a flat state until a critical compressive strain is reached, at which point it buckles, and regains its uncompressed area. Over this buckling transition, the Poisson ratio of the clay sheets turns negative, a property which has been predicted for 2-dimensional sheets ~\cite{Soft_Matter2}. This is the first time such behaviour has been seen in a molecular simulation of mineral layers, the large scale allowing us to probe large buckling wavelengths that are inhibited in smaller scale simulation.
% To simulate a realistically sized clay platelet required very large scale simulations, made possible using the resources available via the LRAC allocation.