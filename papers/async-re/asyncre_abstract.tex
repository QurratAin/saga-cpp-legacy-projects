\documentclass[a4paper,10pt]{article}
\usepackage[utf8]{inputenc}
\usepackage{graphicx}
\usepackage{url}
\usepackage{float}
\usepackage{times}
\usepackage{multirow}
\usepackage{listings}
\usepackage{times}
\usepackage{paralist}
\usepackage{epsfig}
\usepackage{subfigure}
\usepackage[hypertex]{hyperref}
\usepackage{subfigure}
\usepackage{color}

%\documentclass{rspublic}

\usepackage{ifpdf}

\newcommand{\I}[1]{\textit{#1}}
\newcommand{\B}[1]{\textbf{#1}}
\newcommand{\BI}[1]{\textbf{\textit{#1}}}
\newcommand{\T}[1]{\texttt{#1}}

\setlength\topmargin{0in}
\setlength\headheight{0in}
\setlength\headsep{0in}
\setlength\textheight{9in}
\setlength\textwidth{6.5in}
\setlength\oddsidemargin{0in}
\setlength\evensidemargin{0in}
\setlength\parindent{0.1in}
\setlength\parskip{0.25em}


\ifpdf
 \DeclareGraphicsExtensions{.pdf, .jpg}
\else
 \DeclareGraphicsExtensions{.eps, .ps}
\fi

\newcommand{\note}[1]{ {\textcolor{red} { ***NOTE: #1 }}}

\begin{document}

\title{\LARGE Running Asynchronous Replica-Exchange Simulations Across Heterogeneous Distributed Infrastructures}
 
\author{Shantenu Jha$^{1,2,3}$, Abhinav Thota$^{1}$, Andre Luckow$^{1}$ \\
   \small{\emph{$^{1}$Center for Computation \& Technology, Louisiana State University, USA}}\\
   \small{\emph{$^{2}$Department of Computer Science, Louisiana State University, USA}}\\
   \small{\emph{$^{3}$e-Science Institute, University of Edinburgh, UK}}
   }
 
\maketitle
 
Writing applications that are able to orchestrate heterogeneous
resources across virtual organizations (VO) is a complex task.  Several 
classes of applications, which are well suited for loosely
coupled Grids, exist; arguably the best known and most commonly used ones are
task farming applications. Despite the simple nature, many problems
can be solved with such a model, e.\,g.\ parameter studies found in
many sciences or Monte Carlo simulations. More complicated, but possibly more 
interesting than {\it pleasingly  distributed} applications, is the class of applications that are
essentially loosely-coupled, but with a small level of coupling
between the tasks.  A class of algorithms that belongs to this category are
\emph{Replica Exchange Molecular Dynamics (REMD)}~\cite{hansmann,Sugita:1999rm} simulations. 
 
 
- 1 para intro to replica-exchange, and how it is traditionally done (ie case I)
 
- 1 para limitation on traditional replica exchange

- Introduce asynchrononous Replica Exchange --  1 para on case II and case
III (algorithmically)

- Describe how we implement Case II and Case III (you can use figures)
  using SAGA and the advantages
  
- unfortunately we dont have results, so we will say, (i) we establish the
 ability to scale-out (distributed and exa-scale)  across different
 infrastructure (ii) compare the Async versus sync formulation at
 unprecedented scales (iii) compare different implementations  of
 the Async version
 
 
 \bibliographystyle{IEEEtran} 
 \bibliography{literature}


\end{document}

