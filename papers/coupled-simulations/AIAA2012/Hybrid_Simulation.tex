% Typical processing for PostScript (PS) output:
%
%  latex advanced_example
%  bibtex advanced_example  (bibliography)
%  makeindex -s nomencl.ist -o advanced_example.gls advanced_example.glo
%                            (nomenclature)
%  latex advanced_example   (repeat as needed to resolve references)
%
%  xdvi advanced_example    (onscreen draft display)
%  dvips advanced_example   (postscript)
%  gv advanced_example.ps   (onscreen display)
%  lpr advanced_example.ps  (hardcopy)
%
% With the above, only Encapsulated PostScript (EPS) images can be used.
%
% Typical processing for Portable Document Format (PDF) output:
%
%  pdflatex advanced_example
%  bibtex advanced_example    (bibliography)
%  makeindex -s nomencl.ist -o advanced_example.gls advanced_example.glo
%                              (nomenclature)
%  pdflatex advanced_example  (repeat as needed to resolve references)
%
%  acroread advanced_example.pdf  (onscreen display)
%
% If you have EPS figures, you will need to use the epstopdf script
% to convert them to PDF because PDF is a limmited subset of EPS.
% pdflatex accepts a variety of other image formats such as JPG, TIFF,
% PNG, and so forth -- check the documentation for your version.
%
% If you do *not* specify suffixes when using the graphicx package's
% \includegraphics command, latex and pdflatex will automatically select
% the appropriate figure format from those available.  This allows you
% to produce PS and PDF output from the same LaTeX source file.
%
% To generate a large format (e.g., 11"x17") PostScript copy for editing
% purposes, use
%
%  dvips -x 1467 -O -0.65in,0.85in -t tabloid advanced_example
%
% For further details and support, read the Users Manual, aiaa.pdf.



\documentclass[]{aiaa-tc}% insert '[draft]' option to show overfull boxes

 \usepackage{varioref}%  smart page, figure, table, and equation referencing
 \usepackage{wrapfig}%   wrap figures/tables in text (i.e., Di Vinci style)
 \usepackage{threeparttable}% tables with footnotes
 \usepackage{dcolumn}%   decimal-aligned tabular math columns
  \newcolumntype{d}{D{.}{.}{-1}}
 \usepackage{nomencl}%   nomenclature generation via makeindex
  \makeglossary
 \usepackage{subfigure}% subcaptions for subfigures
 \usepackage{subfigmat}% matrices of similar subfigures, aka small mulitples
 \usepackage{fancyvrb}%  extended verbatim environments
  \fvset{fontsize=\footnotesize,xleftmargin=2em}
 \usepackage{lettrine}%  dropped capital letter at beginning of paragraph
 \usepackage[dvips]{dropping}% alternative dropped capital package
 \usepackage[colorlinks]{hyperref}%  hyperlinks [must be loaded after dropping]


%%%%% Title %%%%%
 \title{Multi-species Fluid Flow Simulations using a Hybrid Computational Fluid Dynamics - Molecular Dynamics Appraoch}
 \skonote{Or "Polyatomic Lagrangian Dynamics Modelling for a Hybrid Computational Fluid Dynamics - Molecular Dynamics Appraoch"}
%%%%% End Title %%%%%


%%%%% Author %%%%%
 \author{
  Nayong Kim\thanks{A Post-doctoral Researcher; Center for Computation \& Technology,
  Louisiana State University, Baton Rouge, LA 70803, USA; Non-AIAA Member}\\
  {\normalsize\itshape
   Louisiana State University, Baton Rouge, LA 70803, USA}\\
  \and
  Soon-Heum Ko\thanks{A Computational Scientist; National Supercomputing Centre,
  Link\"{o}ping University, Link\"{o}ping, 581 83 Sweden; Non-AIAA Member}\\
  {\normalsize\itshape
   Link\"{o}ping University, Link\"{o}ping, 581 83, Sweden}\\
  \and
  Shantenu Jha\thanks{An Assistant Professor; Department of Electrical and Computer Engineering,
  94 Brett Road, Piscataway, NJ 08854, USA; Non-AIAA Member}\\
  {\normalsize\itshape
   Rutgers University, Piscataway, NJ 08854, USA}\\
  \and
  Dorel Moldovan\thanks{An Associate Professor; Department of Mechanical Engineering,
  Louisiana State University, Baton Rouge, LA 70803, USA; Non-AIAA Member}
    \ and
  Dimitris E. Nikitopoulos\thanks{A Professor; Department of Mechanical Engineering,
  Louisiana State University, Baton Rouge, LA 70803, USA; AIAA Member}\\
  {\normalsize\itshape
   Louisiana State University, Baton Rouge, LA 70803, USA}\\
 }
%%%%% End Author %%%%%


 % Data used by 'handcarry' option
 \AIAApapernumber{YEAR-NUMBER}
 \AIAAconference{Conference Name, Date, and Location}
 \AIAAcopyright{\AIAAcopyrightD{YEAR}}

 % Define commands to assure consistent treatment throughout document
 \newcommand{\eqnref}[1]{(\ref{#1})}
 \newcommand{\class}[1]{\texttt{#1}}
 \newcommand{\package}[1]{\texttt{#1}}
 \newcommand{\file}[1]{\texttt{#1}}
 \newcommand{\BibTeX}{\textsc{Bib}\TeX}


%%%%% Note Configuration %%%%%
\newcommand{\jhanote}[1]{ {\textcolor{red} { ***Jha: #1 }}}
\newcommand{\Nkimnote}[1]{ {\textcolor{blue} { ***NKim: #1 }}}
\newcommand{\skonote}[1]{ {\textcolor{green} { ***Jeff: #1 }}}
\newcommand{\menote}[1]{ {\textcolor{purple} { ***Comment from ME: #1 }}} 
%%%%% End Note Configuration %%%%%


\begin{document}

\maketitle


%%%%% Abstract %%%%%
\begin{abstract}
\skonote{Before going further, check the AIAA membership on co-authors: basically Dimitris is most likely to own the authorship}
The constrained Lagrangian dynamics modelling in the hybrid computational fluid dynamics (CFD) - molecular dynamics (MD) approach is improved for the simulation of multi-species polyatomic fluid.
Microscopic mean velocity term on the classical Lagrangian dynamics equation is replaced by the division of mean linear momentum and mean mass to account for multi-species fluid system.
Also, the equation is applied on molecules instead of individual atom, to preserve the linear momentum between continuum and particle domain without encountering the unfaborable numerical break-down of molecular structure.
We verify our hybrid CFD-MD simulation package by analyzing a nano-scale transient Couette flow of a single monatomic fluid.
We will evaluate the multi-species polyatomic Lagrangian dynamics modelling by analyzing two different fluid models: the mixture of two monatomic fluids and a polyatomic molecular fluid under the short-range potential.
These two applications will describe the effect of particle-level mass variation on the macroscopic flow evolution.
\end{abstract}
%%%%% End Abstract %%%%%


%\printglossary %creates nomenclature section produced by MakeIndex


%%%%% Introduction %%%%%
\section{Introduction}
\label{sec:intro}

\lettrine[nindent=0pt]{T}{he} hybrid computational fluid dynamics (CFD) - molecular dynamics (MD) approach is getting more attraction as a potential answer in accurately describing the nano-scale flow phenomena in the range of an acceptable computational cost.

%%%%% End Introduction %%%%%


%%%%% Hybrid Schema %%%%%
\section{Hybrid CFD-MD Appraoch for Multi-species Flow}
\label{sec:hybrid}

\subsection{The Hybrid CFD-MD Approach}
\label{sec:hybrid_design}

Contents

\subsection{Constrained Lagrangian Dynamics for Multi-species Flow Simulation}
\label{sec:hybrid_multispecies}

Contents

%%%%% End Hybrid Schema %%%%%


%%%%% Numerical Schemes %%%%%
\section{Development of a Hybrid CFD-MD Simulation Package}
\label{sec:numerics}

\subsection{Continuum Incompressible Flow Solver}
\label{sec:numerics_cfd}

Contents

\subsection{Particle Dynamics Solver}
\label{sec:numerics_md}

Contents

\subsection{Incorporation of Hybrid Schemes}
\label{sec:numerics_hybrid}

Contents: how to design and setup domain - hybrid parameters; noise reduction - multiple replica sampling; how to exchange properties and match - exchanging interface + conversion of properties

%%%%% End Numerical Schemes %%%%%


%%%%% Numerical Solutions %%%%%
\section{Numerical Results}
\label{sec:result}

\subsection{Domain Construction and Validation}
\label{sec:result_val}

Contents

\subsection{Multi-species Flow Simulation}
\label{sec:result_multi}

Contents
%%%%% End Numerical Solutions %%%%%


%%%%% Conclusion and Future Works %%%%%
\section{Conclusion and Future Works}
\label{sec:conclusion}

Contents
%%%%% End Conclusion and Future Works %%%%%


%%%%% Acknowledgement %%%%%
\section*{Acknowledgement}

Content

%%%%% End Acknowledgement %%%%%


% produces the bibliography section when processed by BibTeX
\bibliography{bibtex_database}
\bibliographystyle{aiaa}

\end{document}

% - Release $Name:  $ -
