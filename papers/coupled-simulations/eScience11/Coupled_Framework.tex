%  $Description: Author guidelines and sample document in LaTeX 2.09$
%
%  $Author: ienne $
%  $Date: 1995/09/15 15:20:59 $
%  $Revision: 1.4 $
%
%\documentclass[times, 10pt,twocolumn]{article}
%\documentclass[conference,final]{IEEEtran}
\documentclass[10pt,conference]{IEEEtran}
\usepackage{latex8}
\usepackage{times}

% Users' option
\usepackage{amssymb}
\usepackage{amsmath}
\usepackage{graphicx}
\usepackage{epstopdf}
\usepackage{color}
\topmargin=0.01in
\usepackage{multirow}
\usepackage{booktabs}
\newif\ifdraft
\drafttrue

\renewcommand{\multirowsetup}{\centering}
\renewcommand{\arraystretch}{1.2}
\def\nyc{\centering}

\ifdraft
\newcommand{\fixme}[1]{ { \bf{ ***FIXME: #1 }} }
\newcommand{\jhanote}[1]{ {\textcolor{red} { ***Jha: #1 }}}
\newcommand{\Nkimnote}[1]{ {\textcolor{green} { ***Nkim: #1 }}}
\newcommand{\skonote}[1]{ {\textcolor{blue} { ***Jeff: #1 }}}
\else
\newcommand{\jhanote}[1]{}
\newcommand{\Nkimnote}[1]{}
\newcommand{\fixme}[1]{}
\newcommand{\skonote}[1]{}
\fi
% End of users' option

%\documentstyle[times,art10,twocolumn,latex8]{article}

%-------------------------------------------------------------------------
% take the % away on next line to produce the final camera-ready version
\pagestyle{empty}

\newcommand{\up}{\vspace*{-1em}}
\newcommand{\upp}{\vspace*{-0.5em}}
\newcommand{\ts}{$T_{s}$}


%-------------------------------------------------------------------------
\title{A Framework for Coupled Multi-physics Simulations \skonote{some charming title?? 
emphasizing ''it can cover various kinds of applications in different requirement, 
main components can be replaced by other similar softwares, etc.''} }

\author{
 ~\\[-2em]
 Soon-Heum Ko$^{1,2}$, Nayong Kim$^{2}$, Shantenu Jha$^{3,2}$\\
 \small{\emph{$^{1}$National Supercomputing Centre, Lin\"{o}ping University, Lin\"{o}ping, Sweden}}\\
 \small{\emph{$^{2}$Center for Computation \& Technology, Louisiana State University, USA}}\\
 \small{\emph{$^{3}$Dept. of Elec. and Comp. Eng., Rutgers University, Piscataway, New Jersey, USA}}\\
% \small{\emph{$^{*}$Contact Author}}\\
}

%\thispagestyle{empty}

\begin{document}

\maketitle

\begin{abstract}
we design and develop a multi-physics framework for coupled simulations
in which scientific components (codes) are logically separated.
Designed framework provides the interface between scientific components,
(compilation system for system architectures,)
a runtime environment for scheduling of coupled tasks, and
data management/code versioning support.
The framework is built by developing the data exchange interface 
between coupled codes, (providing the wrapping script to users' Makefiles,)
adopting a BigJob abstraction, and using PetaShare service.
We apply this framework into the hybrid computational fluid dynamics -
particle dynamics application to demonstrate its capability.
\end{abstract}
\up\up

%-------------------------------------------------------------------------
\section{Introduction and Motivation}
On developing multi-physics application codes, two types of production
system can be present. One is to modularize all components and bind into
a single executable, the other is to make multiple standalone codes and 
provide the interface between individual executable. The former is good for
scheduling on most batch queue system and effectively using allocated
resources, while some restrictions on data structure or standardization
can be present on writing the code and the binary can be heavier to contain
unused functions. The other gives much freedom in writing the individual 
component but can also cause computational headache.




Two types of 
What has been done?
What is needed?

% -------------------------------------------------------------------------
\section{Design of a Coupled Simulation Framework}

\subsection{Requirements}

\subsection{Design}

% -------------------------------------------------------------------------
\section{Implementation}
\subsection{Interface between Coupled Tasks}

\subsection{Compilation System}

\subsection{Runtime Environment}

\subsection{Data Management and Archiving}

% -------------------------------------------------------------------------
\section{Numerical Experiments}

% -------------------------------------------------------------------------
\section{Further Achievements}

%-------------------------------------------------------------------------
\section{Conclusions}

%-------------------------------------------------------------------------
\section*{Acknowledgment}

%-------------------------------------------------------------------------
\nocite{ex1,ex2}
\bibliographystyle{latex8}
%\bibliographystyle{IEEEtran}
\bibliography{saga_tg08}


\end{document}


