%  $Description: Author guidelines and sample document in LaTeX 2.09$
%
%  $Author: ienne $
%  $Date: 1995/09/15 15:20:59 $
%  $Revision: 1.4 $
%
%\documentclass[times, 10pt,twocolumn]{article}
%\documentclass[conference,final]{IEEEtran}
\documentclass[10pt,conference]{IEEEtran}
\usepackage{latex8}
\usepackage{times}

% Users' option
\usepackage{amssymb}
\usepackage{amsmath}
\usepackage{graphicx}
\usepackage{epstopdf}
\usepackage{color}
\topmargin=0.01in
\usepackage{multirow}
\usepackage{booktabs}
\newif\ifdraft
\drafttrue

\renewcommand{\multirowsetup}{\centering}
\renewcommand{\arraystretch}{1.2}
\def\nyc{\centering}

\ifdraft
\newcommand{\fixme}[1]{ { \bf{ ***FIXME: #1 }} }
\newcommand{\jhanote}[1]{ {\textcolor{red} { ***Jha: #1 }}}
\newcommand{\Nkimnote}[1]{ {\textcolor{green} { ***Nkim: #1 }}}
\newcommand{\skonote}[1]{ {\textcolor{blue} { ***Jeff: #1 }}}
\else
\newcommand{\jhanote}[1]{}
\newcommand{\Nkimnote}[1]{}
\newcommand{\fixme}[1]{}
\newcommand{\skonote}[1]{}
\fi
% End of users' option

%\documentstyle[times,art10,twocolumn,latex8]{article}

%-------------------------------------------------------------------------
% take the % away on next line to produce the final camera-ready version
\pagestyle{empty}

\newcommand{\up}{\vspace*{-1em}}
\newcommand{\upp}{\vspace*{-0.5em}}
\newcommand{\ts}{$T_{s}$}


%-------------------------------------------------------------------------
\title{A Framework for Coupled Multi-physics Simulations \skonote{some charming title?? 
emphasizing ''it can cover various kinds of applications in different requirement, 
main components can be replaced by other similar softwares, etc.''} }

\author{
 ~\\[-2em]
 Soon-Heum Ko$^{1,2}$, Nayong Kim$^{2}$, Shantenu Jha$^{3,2}$\\
 \small{\emph{$^{1}$National Supercomputing Centre, Lin\"{o}ping University, Lin\"{o}ping, Sweden}}\\
 \small{\emph{$^{2}$Center for Computation \& Technology, Louisiana State University, USA}}\\
 \small{\emph{$^{3}$Dept. of Elec. and Comp. Eng., Rutgers University, Piscataway, New Jersey, USA}}\\
% \small{\emph{$^{*}$Contact Author}}\\
}

%\thispagestyle{empty}

\begin{document}

\maketitle

\begin{abstract}
we design and develop a multi-physics framework for coupled simulations
in which scientific components (codes) are logically separated.
Designed framework provides the interface between scientific components,
(compilation system for system architectures,)
a runtime environment for scheduling of coupled tasks, and
data management/code versioning support.
The framework is built by developing the data exchange interface 
between coupled codes, (providing the wrapping script to users' Makefiles,)
adopting a BigJob abstraction, and using the PetaShare service.
We apply this framework into the hybrid computational fluid dynamics -
particle dynamics application to demonstrate its capability.
\end{abstract}
\up\up

%-------------------------------------------------------------------------
\section{Introduction and Motivation}

{\it Framework} is defined as ''an abstraction in which software 
providing generic functionality can be selectively changed 
by user code, thus providing application specific software.''~\cite{Framework}
According to this definition, a coupled multi-component framework
shall provide the following functionalities: (1) the framework
provides the interface between users' application codes, (2)
embedded softwares should be adaptive to users' implementations 
in diverse formulations, and (3) they should be removable/replaceable
by users' preferences.

Two types of formulations can be present on a coupled multi-component 
framework. One is to modularize all components and bind into
a single executable, and the other is to make multiple standalone softwares
and provide the interface between individual executable. 
The former is good for scheduling on most batch queue system and 
effectively using allocated resources, while some restrictions 
on data structure or standard grammar may apply on users' codes.
Also, the binary can be heavier to contain unused or non-optimized functions
for specific targets. The other gives much freedom in writing 
the individual component but can cause computational headache in scheduling
components in the remote production system.
We consider that binding into the single binary is recommendable if
one simulation package is apparently heavier than the other softwares or
all elements share most memory allocations. It is preferred to provide
the interface and the scheduling function if individual component request
independent memory spaces.
%This way is desirable if one simulation package is apparently lighter than the other or multiples of components are working together for a single application target, or all elements share a very similar data structure with variables or all elements can be scheduled in a sequential/tandem order on the simulation process. Cactus framework is one such example.

A runtime environment for a coupled multi-physics application
~\cite{CCGrid_Hybrid} has been developed previously. In this work,
two coupled yet logically distributed scientific softwares have been
effectively scheduled under the single batch queue allocation, 
by the use of a BigJob~\cite{saga_royalsoc} with the load balancing
capability incorporated. It demonstrated that a Pilot-job formulation
can ease the scheduling of a coupled application whose components are
hardly packaged into a single binary. However this runtime environment
lacks the reusability because it only provides the scheduling functionality
between already-coupled distributed application codes.

In this work, we design and develop a coupled multi-component framework.
Along with the runtime environment between logically separated components,
we also provide the standard interface between these components and
take care of the data management/versioning. The structure of
a coupled simulation framework is introduced in Sec.~\ref{sec:design}.
Specific softwares implemented in this framework are described in
Sec.~\ref{sec:implementation}. A coupled multi-physics application and
an experiment-computation integrated research procedure are presented
in Sec.~\ref{sec:experiment}. Recommendation for future work and 
conclusions are presented in Sec.~\ref{sec:futureworks} and
Sec.~\ref{sec:conclusion}.



% -------------------------------------------------------------------------
\section{Design of a Coupled Simulation Framework}
\label{sec:design}

\subsection{Requirements}
Multiple components in a coupled application is hard to be packaged 
into a single binary if 
(1) each component uses very different computational kernels 
or 
(2) a manual procedure is involved in one of components.
A hybrid computational fluid dynamics (CFD) - molecular dynamics (MD) 
simulation is an example whose data structures are completely different
so that it is fairly hard to incorporate into a single binary.
Another situation is occasionally observed when there is an iterative
feedback between a numerical simulation and the experimental measurement.
Most probably, the physical experiment involves the human labor
of changing specimen/mockup after the numerical analysis, 
which cannot be digitalized.

In both cases, scheduling the operation is of the most importance.


The interface which provides the data stream between multiple components 
is very important.



- Easy interface for coupling
- Data management / versioning
- Scheduling and runtime

\subsection{Design}
- Library form (lighter), none-resource usage
- Distribution, data sharing between local development system and production system + make system for compiling in different architectures
- Co-scheduling and load balancing

% -------------------------------------------------------------------------
\section{Implementation}
\label{sec:implementation}

\subsection{Interface between Coupled Tasks}

\subsection{Compilation System}

\subsection{Runtime Environment}

\subsection{Data Management and Archiving}

% -------------------------------------------------------------------------
\section{Numerical Experiments}
\label{sec:experiment}

% -------------------------------------------------------------------------
\section{Further Achievements}
\label{sec:futureworks}

%-------------------------------------------------------------------------
\section{Conclusions}
\label{sec:conclusion}

%-------------------------------------------------------------------------
\section*{Acknowledgment}

%-------------------------------------------------------------------------
%\nocite{ex1,ex2}
\bibliographystyle{latex8}
%\bibliographystyle{IEEEtran}
\bibliography{saga_tg08}


\end{document}


