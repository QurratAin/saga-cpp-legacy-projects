%  $Description: Author guidelines and sample document in LaTeX 2.09$
%
%  $Author: ienne $
%  $Date: 1995/09/15 15:20:59 $
%  $Revision: 1.4 $
%
%\documentclass[times, 10pt,twocolumn]{article}
%\documentclass[conference,final]{IEEEtran}
\documentclass[10pt,conference]{IEEEtran}
\usepackage{latex8}
\usepackage{times}

% Users' option
\usepackage{amssymb}
\usepackage{amsmath}
\usepackage{graphicx}
\usepackage{epstopdf}
\usepackage{color}
\topmargin=0.01in
\usepackage{multirow}
\usepackage{booktabs}
\newif\ifdraft
\drafttrue

\renewcommand{\multirowsetup}{\centering}
\renewcommand{\arraystretch}{1.2}
\def\nyc{\centering}

\ifdraft
\newcommand{\fixme}[1]{ { \bf{ ***FIXME: #1 }} }
\newcommand{\jhanote}[1]{ {\textcolor{red} { ***Jha: #1 }}}
\newcommand{\Nkimnote}[1]{ {\textcolor{green} { ***Nkim: #1 }}}
\newcommand{\skonote}[1]{ {\textcolor{blue} { ***Jeff: #1 }}}
\else
\newcommand{\jhanote}[1]{}
\newcommand{\Nkimnote}[1]{}
\newcommand{\fixme}[1]{}
\newcommand{\skonote}[1]{}
\fi
% End of users' option

%\documentstyle[times,art10,twocolumn,latex8]{article}

%-------------------------------------------------------------------------
% take the % away on next line to produce the final camera-ready version
\pagestyle{empty}

\newcommand{\up}{\vspace*{-1em}}
\newcommand{\upp}{\vspace*{-0.5em}}
\newcommand{\ts}{$T_{s}$}


%-------------------------------------------------------------------------
\title{Runtime Environment for High Performance Computational Fluid Dynamics Applications based on SAGA}

\author{
 ~\\[-2em]
 Soon-Heum Ko$^{1,2}$, Nayong Kim$^{2}$, Shantenu Jha$^{3,2}$\\
 \small{\emph{$^{1}$National Supercomputing Centre, Lin\"{o}ping University, Lin\"{o}ping, Sweden}}\\
 \small{\emph{$^{2}$Center for Computation \& Technology, Louisiana State University, USA}}\\
 \small{\emph{$^{3}$Dept. of Elec. and Comp. Eng., Rutgers University, Piscataway, New Jersey, USA}}\\
% \small{\emph{$^{*}$Contact Author}}\\
}

%\thispagestyle{empty}

\begin{document}

\maketitle

\begin{abstract}
We propose a runtime environment for high-performance computational
fluid dynamics (CFD) applications in supercomputing resource pool.
We direct to ease technical difficulties in submitting/scheduling
specific CFD applications on traditional batch queues through the use
of a pilot-job implementation. We address three kinds of CFD applications
which suffer from the difficulty in scheduling: a single large-scale
long-term job, multiple jobs for parametric study and coupled multi-component
jobs. A BigJob framework (needs reference!!!!!) on the basis of SAGA 
(Simple API for Grid Applications) is introduced to solve above difficulties,
in which the technical basis for effectively scheduling multiple jobs~\cite{SAGA_Thota}
or redistributing resources for coupled applications~\cite{CCGrid_Hybrid} 
has been already explored. So we focus on implementing a framework for running
a long-term job by the single launch from users' perspective. The idea is to
submit a number of BigJobs in tandem and migrate to the latest allocation
until the completion of a simulation, within the extent of administrating
policies of supercomputing centers and generally agreeable morals. Along with 
this technical development, we also design an interface to conveniently submit
those types of CFD jobs. We present our use cases in this paper, one which is
fixed in size and whose simulation time exceeds the maximal wall time of a single allocation 
(unsteady flapping motion of an insect), the other which contains multiple moderate jobs
in size and time whose time-for-completion might exceed one job allocation 
(fluid-structure interaction for aerodynamic shape optimization), and
still the other which is composed of multiple coupled jobs (coupled simulation of
aerodynamics and combustion).
\end{abstract}
\up\up

%-------------------------------------------------------------------------
\section{Introduction and Motivation}

% -------------------------------------------------------------------------
\section{SAGA and SAGA-based Frameworks for Large-Scale and Distributed Computation}

% -------------------------------------------------------------------------
\section{Scheduling a Single Long-term Job under the Supercomputing System}

% -------------------------------------------------------------------------
\section{Designing the Interface for CFD Applications on Supercomputers}

% -------------------------------------------------------------------------
\section{Three Use Cases}

%-------------------------------------------------------------------------
\section{Conclusions}

%-------------------------------------------------------------------------
\section*{Acknowledgment}

%-------------------------------------------------------------------------
\nocite{ex1,ex2}
\bibliographystyle{latex8}
%\bibliographystyle{IEEEtran}
\bibliography{saga_tg08}


\end{document}


