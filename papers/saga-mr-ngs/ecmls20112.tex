% THIS IS SIGPROC-SP.TEX - VERSION 3.1
% WORKS WITH V3.2SP OF ACM_PROC_ARTICLE-SP.CLS
% APRIL 2009
%
% It is an example file showing how to use the 'acm_proc_article-sp.cls' V3.2SP
% LaTeX2e document class file for Conference Proceedings submissions.
% ----------------------------------------------------------------------------------------------------------------
% This .tex file (and associated .cls V3.2SP) *DOES NOT* produce:
%       1) The Permission Statement
%       2) The Conference (location) Info information
%       3) The Copyright Line with ACM data
%       4) Page numbering
% ---------------------------------------------------------------------------------------------------------------
% It is an example which *does* use the .bib file (from which the .bbl file
% is produced).
% REMEMBER HOWEVER: After having produced the .bbl file,
% and prior to final submission,
% you need to 'insert'  your .bbl file into your source .tex file so as to provide
% ONE 'self-contained' source file.
%
% Questions regarding SIGS should be sent to
% Adrienne Griscti ---> griscti@acm.org
%
% Questions/suggestions regarding the guidelines, .tex and .cls files, etc. to
% Gerald Murray ---> murray@hq.acm.org
%
% For tracking purposes - this is V3.1SP - APRIL 2009

%\documentclass{acm_proc_article-sp}
\documentclass{sig-alternate}
\usepackage[numbers, sort, compress]{natbib}
\usepackage{graphics}
\usepackage{graphicx}
\usepackage{epstopdf}
\usepackage{color}
\usepackage{hyperref}
\usepackage{pdfsync}
\usepackage{mdwlist}

\begin{document}

\conferenceinfo{ECMLS'12,} {}
\CopyrightYear{2012}
\crdata{978-1-4503-0702-4/11/06}
\clubpenalty=10000
\widowpenalty = 10000


%\title{A Sample {\ttlit ACM} SIG Proceedings Paper in LaTeX
%Format\titlenote{(Does NOT produce the permission block, copyright information nor page numbering). For use with ACM\_PROC\_ARTICLE-SP.CLS. Supported by ACM.}}

\newif\ifdraft
%\drafttrue                                                                                                   

\ifdraft
% \newcommand{\reviewer}[1]{ {\textcolor{blue}    { ***Reviewer:     #1 }}}
 \newcommand{\jkimnote}[1]{{\textcolor{green}   { ***Joohyun:   #1 }}}
 \newcommand{\jhanote}[1]{  {\textcolor{red}     { ***SJ: #1 }}}
  \newcommand{\pmnote}[1]{  {\textcolor{red}     { ***Pradeep: #1 }}}
 \newcommand{\todo}[1]{  {\textcolor{red}     { ***TODO: #1 }}}
 \newcommand{\fix}[1]{  {\textcolor{red}     { ***FIX: #1 }}}
 \newcommand{\reviewer}[1]{}
\else
 \newcommand{\reviewer}[1]{}
 \newcommand{\jkimnote}[1]{}
 \newcommand{\pmnote}[1]{}
 \newcommand{\jhanote}[1]{}
 \newcommand{\todo}[1]{  {\textcolor{red}     { ***TODO: #1 }}}
 \newcommand{\fix}[1]{}                                                                                     
\fi

\title{Next-Generation Sequencing Reads Alignment using SAGA-MapReduce}

\numberofauthors{3} %  in this sample file, there are a *total*
% of EIGHT authors. SIX appear on the 'first-page' (for formatting
% reasons) and the remaining two appear in the \additionalauthors section.
%
\author{
% You can go ahead and credit any number of authors here,
% e.g. one 'row of three' or two rows (consisting of one row of three
% and a second row of one, two or three).
%
% The command \alignauthor (no curly braces needed) should
% precede each author name, affiliation/snail-mail address and
% e-mail address. Additionally, tag each line of
% affiliation/address with \affaddr, and tag the
% e-mail address with \email.
%
\alignauthor Pradeep Kumar Mantha\\
       \affaddr{Center for Computation and Technology}\\
       \affaddr{Louisiana State University}\\
       \affaddr{216 Johnston}\\
       \affaddr{Baton Rouge, LA}
       \email{pradeepm66@gmail.com}
\alignauthor Andre Luckow\\
       \affaddr{Center for Computation and Technology}\\
       \affaddr{Louisiana State University}\\
       \affaddr{216 Johnston}\\
       \affaddr{Baton Rouge, LA}
       \email{yours are here}       
\alignauthor Joohyun Kim\titlenote{Author for correspondence}\\
       \affaddr{Center for Computation and Technology}\\
       \affaddr{Louisiana State University}\\
       \affaddr{216 Johnston}\\
       \affaddr{Baton Rouge, LA} \\
       \email{jhkim@cct.lsu.edu}
\alignauthor Shantenu Jha\titlenote{Author for correspondence}\\
      \affaddr{Center for Computation and Technology}\\
     \affaddr{Louisiana State University}\\
      \affaddr{214 Johnston}\\
      \affaddr{Baton Rouge, LA}
     \email{sjha@cct.lsu.edu}
}
% There's nothing stopping you putting the seventh, eighth, etc.
% author on the opening page (as the 'third row') but we ask,
% for aesthetic reasons that you place these 'additional authors'
% in the \additional authors block, viz.
%\additionalauthors{Additional authors: John Smith (The Th{\o}rv{\"a}ld Group,
%email: {\texttt{jsmith@affiliation.org}}) and Julius P.~Kumquat
%(The Kumquat Consortium, email: {\texttt{jpkumquat@consortium.net}}).}
\date{25 Feb. 2012}
% Just remember to make sure that the TOTAL number of authors
% is the number that will appear on the first page PLUS the
% number that will appear in the \additionalauthors section.

\maketitle
\begin{abstract} 

In this work, we present the development of a Next-Generation Sequencing reads alignment tool using PilotJob (PJ)-based MapReduce.  Our implementation of the MapReduce programming model is based on previous works, SAGA-MapReduce, using  SAGA (www.saga-project.org), in particularly focusing on the capability of developing the dynamic application which represents applications executed in distributed resources via effective and dynamic task and data management.  

Computational challenges with data intensive computing associated with high-throughput DNA sequencing technologies such as Next-Generation Sequencing platforms are recently of great interests in bioinformatics, computational biology, and biocomputing in general.  The novel challenges associated with are rather complicated because that an integrative solution leveraging algorithmic advances, computational implementations, and infrastructure developments altogether is required for a biologist with sequencing data sets in his/her hand.  In response to such challenges, in this work we demonstrate how PJ-based MapReduce has a capability for executing a task of NGS data analysis effectively with recognizing aspects associated with algorithms, implementations, and infrastructure.

Our goal with this work is to develop an implementation of a promising programming model for distributed data processing, MapReduce with a SAGA-based approach for a computational task, reads alignment and duplicate removal, to assess the performance of this tool in distributed computing environments, and to discuss the potential of this approach for a broad range of NGS data analytics.  Also, we compare our implementation with other approaches that is based on conventional MapReduce and Hardoop-based approaches. 


 
\end{abstract}

\category{D.1.3}{Software}{Concurrent Programming}{ Distributed
  programming/parallel programming} \category{J.3}{Computer
  Applications}{Bioinformatics, Mapping}


% A category with the (minimum) three required fields
%\category{H.4}{Information Systems Applications}{Miscellaneous} %Acategory including the fourth, optional field follows...
%\category{D.2.8}{Software Engineering}{Metrics}[complexity measures,performance measures]

\terms{Design, Experimentation, Performance}

 \keywords{Genome Sequence Alignment, BWA, Human Genome, RNA-Seq,
  MapReduce, Distributed Computing, Simple API for Grid
  Applications (SAGA)}

%\keywords{ACM proceedings, \LaTeX, text tagging} % NOT required for Proceedings 
%\keywords{RNA conformation energy landscape, Runtime Environment, SAM-I riboswitch,
% S gene of Bovine Corona Viral Genome} % NOT required for Proceedings

\section{INTRODUCTION} 



\section{SAGA-MapReduce for Next-Generation Sequencing Reads Alignment : Background}
\pmnote{PJ based mapreduce involves advert coordination- whereas HDFS filesytem involves TCP/IP layer for communication between data nodes; and job nodes use RPC to communicate between each other.

Hadoop/Normal MR cannot be scaled to multiple clusters( unless and until clusters have some common global filesystem ,, which is rare and involves latnecy issues)... .. PMR can be scaled to mulitple clusters and no need of having common filesystem..
}

\section{Implementation and Characteristics}

 \begin{table}
 \small
 \begin{tabular}{|c|c|c|c|c|} 
 \hline 
   &  SAGA- & SEAL & Crossbow & CloudBurst \\ 
   & MapReduce &  &  \\\hline
 Key  & PilotJob   &  Hardoop  &  Hardoop & Hardoop \\ 
 Aspect &   & (HDFS)  &  (HDFS) & (HDFS) \\ \hline
Target & Alignment/ & Alignment/ & Alignment/ & Alignment \\
 Tasks       & Multiple & Duplicate & SNP & \\ 
        & Analyses &  Removal & &  \\ \hline  
Distri-  & Fully   & Hard  & Hard & Hard \\
buted   & interoperable &  &   & \\ 
Resources &  &  &  & \\ \hline

Strength & 1. Multiple  &  &  &  \\ 
& Resource &  &  &  \\
&  2. Decoupling & &  &  \\ 
& of Map/Reduce &  &  &  \\ \hline
Weakness &  & 1. HDFS & 1. HDFS & 1. HDFS \\ \hline

\hline
\end{tabular}
\caption{Comparison of MapReduce Implementations utilized for Next-Generation Sequencing Tools}
  
  \label{table:mr-comparison} 
\end{table}



\section{Performance}


\section{Conclusion and Future Work}


% \begin{table}
% \small
% \begin{tabular}{|c|c|c|c|c|c|} 
% \hline 
%Case & Read File & Threads   &  \# of & BigJob Size   &   $T_C$   \\
%   & Size& per Task & Tasks  & Cores(Nodes)  & \\
%   \hline
%g1 & 0.209 GB & 2 &   40 &  80(10) & 3966 s \\
%g2 & 0.435 GB & 2 &  20 & 40(5) & 8031 s\\ \hline
%g3  & 0.209 GB& 2 & 40  & 12(3) & 25807 s \\
%g4 & 0.435 GB& 2 & 20  & 12(3) & 23872 s  \\ \hline
%\hline
%g5 & 0.209 GB& 2& 40 & 80(20) & 1111 s \\
%g6&0.435 GB&2& 20 & 40(10)&2096 s\\
%\hline
%\end{tabular}
%\caption{Performance comparison for different parallel configurations
%  using SAGA-BigJob on a HPC-Grid (LONI). One BigJob is submitted with
%  the number of sub-jobs, where each sub-job is a BFAST task.  The
%  total (read) data size is the read-file size multiplied by the
%  number of tasks (which is equal to the number of read-files); which
%  is a constant for g1-g6.  Cases g1, g2 and g3,g4 and g5,g6 are
%  conducted on QB, Painter and Eric respectively. The cases g1, g2,
%  g3, g4 are use 40 index files of a Human Chromosome 21.  Note that
%  g5 and g6 are the results with 10 index files; g6 is specifically
%  carried out to provide a direct comparison to c5 (on a cloud
%  resource as in the following Table~\ref{table:cloud-VM}) }
%  
%  \label{table:bigjob-loni} 
%\end{table}



\section*{Acknowledgement}
This document was developed with support from the National Science
Foundation (NSF) under Grant No.  0910812 to Indiana University for
``FutureGrid: An Experimental, High-Performance Grid Test-bed.''  We
also acknowledge Ole Weidner and Le Yan for useful performance related
discussions, Diana Holmes for sharing her experience with mapping
using BFAST, and Jong-Hyun Ham for allowing us to use B. Glumae genome
sequences.  Computing resources were made possible via NSF TRAC award
TG-MCB090174 and LONI resources.  The project described was partially
supported by Grant Number P20RR016456 from the NIH National Center For
Research Resources.

\bibliographystyle{abbrv} 
\bibliography{ecmls12}
\bibligography{saga}


\end{document}

Any opinions, ndings, and conclusions or recommendations expressed in
this material are those of the author(s) and do not necessarily
reflect the views.
