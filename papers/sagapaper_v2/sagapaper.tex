\documentclass[a4paper,10pt]{article}

\usepackage[utf8]{inputenc}
\usepackage{color}
\usepackage{epsfig}
\usepackage{float}
\usepackage{graphicx}
\usepackage{hyperref}
\usepackage{ifpdf}
\usepackage{listings}
\usepackage{multirow}
\usepackage{paralist}
\usepackage{subfigure}
\usepackage{times}
\usepackage{url}
\usepackage{xspace}
\usepackage{booktabs}

%\setlength\topmargin{0in}
%\setlength\headheight{0in}
%\setlength\headsep{0in}
%\setlength\textheight{9.5in}
%\setlength\textwidth{6.5in}
%\setlength\oddsidemargin{0in}
%\setlength\evensidemargin{0in}
%\setlength\parindent{0.1in}
%\setlength\parskip{0.25em}

\ifpdf
 \DeclareGraphicsExtensions{.pdf, .jpg}
\else
 \DeclareGraphicsExtensions{.eps, .ps}
\fi

\newcommand{\note}[1]{ {\textcolor{red} { ***NOTE: #1 }}}

\newif\ifdraft
\drafttrue

\ifdraft
\newcommand{\todo}[1]{     {\textcolor{red}  { ***TODO      #1 }}}
\newcommand{\amnote}[1]{   {\textcolor{gree} { ***Andre:    #1 }}}
\newcommand{\jhanote}[1]{  {\textcolor{red}  { ***Shantenu: #1 }}}
\newcommand{\onote}[1]{    {\textcolor{blue} { ***Ole:      #1 }}}
\else
\newcommand{\todo}[1]{}
\newcommand{\amnote}[1]{}
\newcommand{\jhanote}[1]{}
\newcommand{\onote}[1]{}
\fi

\newcommand{\spec}{\sagaspec}
\newcommand{\impl}{\sagaimpl}
\newcommand{\lf}{Look-\&-Feel}


\begin{document}

\title{ \large \vspace{-3.5em} SAGA: An Access Layer to Distributed CI
  and Providing Abstractions for Distributed Applications}


\author{\normalsize Shantenu Jha$^{1}$, Andre Merzky$^{2}$, Ole
  Weidner$^{2}$, \\ \small{\emph{$^{1}$Rutgers University, NJ, 08804,
      USA}}\\ \small{\emph{$^{2}$Center for Computation \& Technology,
      Louisiana State University, USA}}\\ } \date{}
 \maketitle

\abstract{The abstract.}
 
\section{Introduction}\todo{SJ}
 \label{intro}

\begin{verbatim}
   - grand intellectual challenges of designing DCIs
     - how to design infrastructures to bridge gap
     - what is role of abstractions/interfaces in general, and for
       SAGA
       - interface necessary, but sufficient?
       - should be sufficient, if fully functional (contract)
       - but what are the challenges of providing a functional interface?
       - at the end deep integration!
       - from there follows design, limitations
         (possibly goes partially into conclusions)

    - roles for SAGA: 
      - shield from environment
      - need to be able to reason about distribution without
      considering those env details
\end{verbatim}

Distributed applications utilize multiple resources, or are capable of
utilizing them. One could argue that any application that would
benefit from increased peak performance, throughput, or reduced time
to solution, by using multiple compute and distributed data resources,
can be classified as a distributed application, or a candidate to be
formulated as such.

% Although \ci technologies have matured considerably over the past
% few years, applications that can effectively utilize these
% technologies are far from ubiquitous.

The ability to develop or formulate scientific applications as
distributed applications,
% Advances in distributed applications have 
has simply not kept pace with advances in other aspects of distributed
cyberinfrastructure (CI). This is true whether measured by the number
of novel applications developed to exploit distributed infrastructure,
or even by something simpler, such as the number of applications that
can use multiple resources to reduce the time-to-solution of
scientific problems.  There are at least three complex
reasons~\cite{dpa_grid09}, which defy over-simplification. We believe
the main issue is simply that developing large-scale distributed
applications is fundamentally a difficult process~\cite{dpa-paper},
made more difficult due to the scarcity of high-level programming
abstractions and interfaces that bridge the divide between the needs
of distributed applications and the capabilities offered by middleware
and system-level interfaces~\cite{cloud-saga-paper}. Second,
deployment and execution concerns are often disjoint from the
development process~\cite{dpa_grid09}. Finally, the range of tools,
programming systems, and environments is bewildering large, making
extensibility and interoperability difficult.  SAGA will help
applications and tools fulfill the goal: ``As much as possible,
software should run independent of specific hardware and should use
intuitive interfaces that do not depend upon extensive knowledge of
the operating environment,'' as recommended in a recent draft report
from the NSF workshop on Cyberinfrastructure Software Sustainability
and Reusability~\cite{CISSR}.


Against this backdrop, the distributed infrastructures available to
scientists continue to evolve in scale and capability, as well as
complexity.  Investments in legacy applications need to be preserved,
while at the same time, development of functionally novel and
architecturally different applications for new and evolving
environments needs to be facilitated. 

Developing versus Deployment versus Using

For context: it is still very difficult to develop a distributed
application that can use O(1000) processors, transfer/handle O(100) GB
data and use O(10) arbitrarily chosen sites, repeated over 10
arbitrary chosen days. The only science projects that can do so
currently are Big Science projects, with very large teams and levels
of effort.  Individual PIs/scientists are unable to do so, but there
are many who clamor and need to be able to do so This fundamental
mismatch between requirement of DA and capabilities of DCI is one of
the intellectual problems that this paper (and solutions therein) are
trying to address.



\section{Critical Perspective on DCIs}\todo{SJ}

 \begin{verbatim}
    - Engineering challenges 

    - Why are DCIs complex in the first
    place?  Many requirements, many different usage modes, no simple
    straight-forward solution (i.e. engineering approach) possible.


    - narrow grids vs. broad grids: narrow grids provide a more limited set of
    solutions (cater to more limited set of requirements), and can thus
    potentially be implemented on a lower complexity level.

    - plethora of point-wise solutions vs. end-to-end solutions

\end{verbatim}

A distributed application has some unique challenges: a distributed
 application/developer has to reason about the environment that the
 application will execute in, has to make assumptions about the
 runtime availability, and has to assume a failure model that
 transcends complete knowledge of the application or even predicates
 it upon knowledge of a system/resource.  Whereas the choice of tools
 will always be influenced by soft factors, the challenge for us is to
 think of abstractions that enable the developer to reason about the
 above hard factors, and not just the tools, services and
 capabilities.

The process of developing and deploying large-scale distributed
 applications presents a critical and challenging agenda for
 researchers and CI developers.  In spite of the tremendous potential
 of distributed systems, there have only been a limited number of
 successful distributed applications; in the case of the TG, where
 there has been success, the effort required has been {\it heroic} but
 unsustainable and thus not surprisingly heavily biased towards big
 showcase projects. It is difficult to examine under the covers of
 these prestige projects, but there has not been a trickle down
 advantage to communities of smaller users, who cannot sustain heroic
 efforts in order to accomplish their science.


 \jhanote{Some text about the mismatch between DCA requirements and
  DCI capabilities}

\section{Access Layers to DCI and Abstractions for DA}\todo{AM}

Revisiting the reasons why production DCI have not been used in more
innovative ways, reveals that there are currently gaps at several
levels: application development and runtime capabilities, as well as
system software and deployment support~\cite{dpagrid2009}.  No single
solution can address gaps at all levels, or for the entire spectrum of
applications.  

We posit, however, that a well-defined, stable general purpose API is
one critical element of the solution, as that exposes the basic
distributed capabilities to serve as an uniform access layer to the
infrastructure layer as well as a building blocks for higher-level
applications, capabilities and tools.

We will substantiate this claim from two different
viewpoints/directions: in the first, we will examine the role of an
API that acts as an Access Layer to the levels/layers of
infrastructure beneath. The second viewpoint will examine how we can
build tools \& services using such an API...

The vertical axis can be thought of as a non-continous ordering, with
the bottom level being closer to the hardware and the upper levels
being closer to the application.

It is important to note that every API can be considered as an
abstraction to underlying layers: the first challenge when designing
an API the challenge is to define the level for such an abstraction
layer.  

It is often misunderstood that placing an abstraction level at a given
level, implies that all capabilities below that level should be
supported. However this is not the case, and a second challenge is to
define the scope, viz., which features and capabilities should the API
provide an abstraction too.

\jhanote{This section should result in the motivation of the API:
  level and scope}

%\jhanote{Q: Determine whether HG metaphor is 2 dimensional or 3?}

\begin{verbatim}
   - Role of an Access Layer as defined:
      - provide abstractions to set-of/selected levels/capabilities below
      - support heterogeneity, environment (elaborate on environment..), 
      - provide uniformity

    - Role for the abstraction to DA 
      - need to be able to reason about applications without 
        considering those environment details

   - Placement of the neck, and width/scope of the HG.
     Placing the access layer/level-of-abstraction/where is the neck
     of our hour glass?

   - Define where we are placing it. THAT IS SAGA...
\end{verbatim}

\section{Design Objectives and Principles}\label{saga_impls}

 \subsection{Design Objectives and Principles}

 \jhanote{Objectives should be drawn from both the role of access
   layer and role as an abstraction/BB for DA}

 \jhanote{Manage heterogeneity, complexity, simplicity,
   sustainabiltiy, enable reasoning, contstruction}

  \begin{verbatim}
    Objective: Heterogeneity 
    Principle: ...
    Solution: Adaptor
   \end{verbatim}

  \begin{verbatim}
    Objective: sustainability
    Principle: stabilty
    Solution: standard
   \end{verbatim}

 \jhanote{take the rest of section 4 and put it into
   Objective/Principle/Solutions tuple style}

\begin{verbatim}
    -> How approached -> Design Principles, next section
    - challenge of environment (define env: fragile, lossy, heterog,
      reliability/failure, complex, dynamic, evolving, ) 
      - adaptors: approach to heterogeneity (different syntax, different
        semantics, late binding, heterogeneous in time -> dynamic), also supports 
        separation of concern
      - reliability: we have a model (define it), but is addressed elsewhere
      - continous testing is part of the answer to dynamic env /
        boundary condition

    - challenge of diverging and multitude of application
      requirements?

     -> be relevent to large user base and longish time scales
     -> good scoping API, use cases, requests, common practice
     -> still too large, apps are moving, no 1st principle apps,
        stability vs. agility, poorly understood use cases, moving targets, 

     -> Manage Heterogeneity
     -> portable, adaptor based
     -> deployment vs coding, 'semantic equivalence'
  
     -> easy to deploy / easy to integrate
     -> components, containment, loose coupling
     -> tight coupling, leaky abstractions
  
     -> extensibility
     -> adaptors, packages, higher level abstraction
     -> adaptors: hard to enforce semantic coherence
        packages: ok (mod impl)
        HLA:      hindered by impl deficiencies
  
     -> sustainable production code in academic env
     -> community involvement, vendor uptake/buy-in, small code base,
        standardization (see Zakopane paper)
     -> integration with ecosystem (technical and social), 
   \end{verbatim}

{\it The objective of SAGA, the 'Simple API for Grid Applications is
  to provide this missing critical component in the distributed
  cyberinfrastructure ecosystem.}\footnote{While the API's name
  suggests its string ties to Grid based DCIs, it is in fact a general
  purpose API for distributed -- which is historically rooted in the
  Grid community.}  SAGA can provide effective abstractions that can
hide the environmental complexity, supplement the incompleteness and
lack-of-extensibility of many tools used whilst promoting
interoperability as first-class design objective.  On the other hand
it must provide the building blocks upon which the distributed
applications, tools and frameworks can be built...


\jhanote{This section should end up motivating the Architecture and
 Implementation}

\section{SAGA: Architecture and Implementation} \todo{AM, Ole}

\jhanote{All three levels are to be covered: API, Implementation of
  the API and the adaptors}

 Realization of the prinicples; provide integrated realization of the
 principle, which leads to an overall implementation..

\section{SAGA: Implementation Challenges}\todo{Ole}

Having an architecture is only half the battle -- one also needs an
implementation of the architecture. 


\subsection{Software Challenges}

\subsection{Deployment and Integration} 

\subsubsection{Integration with Middleware}
\subsubsection{Integration with Infrastructure}


\section{Applications and Frameworks}\todo{SJ}
\label{apps_and_frameworks}


Addressing this reiterates the importance of a well-defined general
purpose API that provides the basic distributed capabilities to serve
as building blocks for further applications and tools. The status of
the current workflow tools and enactment engines provides an
illustrative example~\cite{nsf-workflow,1196459}. Had SAGA or a
SAGA-like API been around, it is very likely that many distributed
workflow engines would have utilized SAGA (or parts thereof), instead
of proprietary solutions, to implement {\it common and basic}
distributed functionality, such as distributed job submission and
distributed file movement/management. SAGA's impact on the workflow
world can be seen through the consequences of its absence: in spite of
significant effort, workflow interoperability at multiple levels --
application, tools, enactment engines and components remains difficult
if not infeasible.  Significant effort has been invested towards
workflow interoperability at these different levels -- if nothing
else, providing post-facto justification of its importance.
Additionally, workflow capabilities and engines are typically tied to
specific tools and infrastructure (e.\,g.\ DAGMan-Condor) and require
the adaption of the application/usage modes to the workflow engine as
opposed to the other way around.



The situation now is potentially very different for emerging tools and
infrastructure such as Pilot-Jobs. A well defined API for distributed
applications in the form of Pilot-API now exists; a stable
implementation of it is on the horizon; the importance of
extensibility, interoperability and lessons from the workflow
experience have hopefully been learned and there is a willingness to
adopt and integrate...

\section{Discussion}\todo{All}

\subsection{Lessons Learned}

\subsubsection{Role of Standards?}

\subsection{Related Work}

\subsubsection{The Interface Standards Landscape}\label{interface_landscape}

\subsection{DCI and DA: The Road Ahead}

\subsubsection{Emergence of Clouds}

\subsubsection{Data as First-Class Entity}

\subsubsection{Application-Level Dynamism}

\bibliographystyle{IEEEtran} \bibliography{sagapaper,saga_ogf}


\end{document}

