\documentclass[a4paper,10pt]{article}

\usepackage[utf8]{inputenc}
\usepackage{color}
\usepackage{epsfig}
\usepackage{float}
\usepackage{graphicx}
\usepackage{hyperref}
\usepackage{ifpdf}
\usepackage{listings}
\usepackage{multirow}
\usepackage{paralist}
\usepackage{subfigure}
\usepackage{times}
\usepackage{url}
\usepackage{xspace}
\usepackage{booktabs}

%\setlength\topmargin{0in}
%\setlength\headheight{0in}
%\setlength\headsep{0in}
%\setlength\textheight{9.5in}
%\setlength\textwidth{6.5in}
%\setlength\oddsidemargin{0in}
%\setlength\evensidemargin{0in}
%\setlength\parindent{0.1in}
%\setlength\parskip{0.25em}

\ifpdf
 \DeclareGraphicsExtensions{.pdf, .jpg}
\else
 \DeclareGraphicsExtensions{.eps, .ps}
\fi

\newcommand{\note}[1]{ {\textcolor{red} { ***NOTE: #1 }}}

\newif\ifdraft
\drafttrue

\ifdraft
\newcommand{\todo}[1]{     {\textcolor{red}  { ***TODO      #1 }}}
\newcommand{\amnote}[1]{   {\textcolor{gree} { ***Andre:    #1 }}}
\newcommand{\jhanote}[1]{  {\textcolor{red}  { ***Shantenu: #1 }}}
\newcommand{\onote}[1]{    {\textcolor{blue} { ***Ole:      #1 }}}
\else
\newcommand{\todo}[1]{}
\newcommand{\amnote}[1]{}
\newcommand{\jhanote}[1]{}
\newcommand{\onote}[1]{}
\fi

\newcommand{\spec}{\sagaspec}
\newcommand{\impl}{\sagaimpl}
\newcommand{\lf}{Look-\&-Feel}


\begin{document}

 \title{ \large \vspace{-3.5em} The Role of Common Interfaces in
   Distributed Applications and Systems: The SAGA Perspective}


 \author{\normalsize Shantenu Jha$^{1}$, Andre Merzky$^{2}$, Ole Weidner$^{2}$, \\
   \small{\emph{$^{1}$Rutgers University, USA}}\\
   \small{\emph{$^{2}$Center for Computation \& Technology, Louisiana State University, USA}}\\
 }
 \date{}
 \maketitle

\abstract{The abstract.}
 
\section{Introduction}
 \todo{SJ}
 \label{intro}

  Intro

 \begin{verbatim}
   - grand intellectual challenges of designing DCIs
     - how to design infrastructures to bridge gap
     - what is role of abstractions/interfaces in general, and for
       SAGA
       - interface necessary, but sufficient?
       - should be sufficient, if fully functional (contract)
       - but what are the challenges of providing a functional interface?
       - at the end deep integration!
       - from there follows design, limitations
         (possibly goes partially into conclusions)
    - roles for SAGA: 
      - shield from environment
      - need to be able to reason about applications without considering those env
        details
 \end{verbatim}

Developing versus Deployment versus Using

For context: it is still very difficult to develop a distributed
application that can use O(1000) processors, transfer/handle O(100) GB
data and use O(10) arbitrarily chosen sites, repeated over 10
arbitrary chosen days. The only science projects that can do so
currently are Big Science projects, with very large teams and levels
of effort.  Individual PIs/scientists are unable to do so, but there
are many who clamor and need to be able to do so This fundamental
mismatch between requirement of DA and capabilities of DCI is one of
the intellectual problems that we are trying to address.

A distributed application has some unique challenges: a distributed
application/developer has to reason about the environment that the
application will execute in, has to make assumptions about the runtime
availability, and has to assume a failure model that transcends
complete knowledge of the application or even predicates it upon
knowledge of a system/resource.  Whereas the choice of tools will
always be influenced by soft factors, the challenge for us is to think
of abstractions that enable the developer to reason about the above
hard factors, and not just the tools, services and capabilities.



\section{Critical Perspective on DCIs}
\todo{SJ}

 \begin{verbatim}
   - plethora of point-wise solutions vs. end-to-end solutions
   - trends and requirements
   - elaborate on introduction
 \end{verbatim}

  \begin{verbatim}
    - see engineering challenges below: *why* are DCIs complex in the first
    place?  Many requirements, many different usage modes, no simple
    straight-forward solution (i.e. engineering approach) possible.

    - narrow grids vs. broad grids: narrow grids provide a more limited set of
    solutions (cater to more limited set of requirements), and can thus
    potentially be implemented on a lower complexity level.
  \end{verbatim}

  \jhanote{Some text about the mismatch between DCA requirements and
  DCI capabilities}

\section{SAGA as Access Layer to DCI}
\todo{AM}

  Intro. OGF, DRMAA, SAGA, SAGA Extensions\\

  \subsection{The Simple API for Grid Applications}

  \begin{verbatim}
    - roles for SAGA: 
      - shield from environment
      - need to be able to reason about applications without considering those env
        details

    - interfaces 'cannot provide what is not there', they MUST have semantic limitations:
      - boundary conditions (given by lower level semantics)
      - choosing level of abstraction

    -> Requirements / Design Objectives
      - implement SAGA to meet role/challenges of previous paragraphs
      -> overarching goal: Manage Complexities
       + simplicity
       + community project
       + widely accepted

    -> How approached -> Design Principles, next section

  \end{verbatim}

 % end of this section:
 \subsection{The Interface Standards Landscape}
 \label{interface_landscape}




\section{How to implement SAGA to meet challenges of Section 3}
\label{saga_impls}

 Intro. jSAGA, JavaSAGA, SAGA-C++, Bliss, DESHL(?)

 \subsection{Design Principles}

  \begin{verbatim}
    - challenge of environment (define env: fragile, lossy, heterog,
      reliability/failure, complex, dynamic, evolving, ) 
      - adaptors: approach to heterogeneity (different syntax, different
        semantics, late binding, heterogeneous in time -> dynamic), also supports 
        separation of concern
      - reliability: we have a model (define it), but is addressed elsewhere
      - continous testing is part of the answer to dynamic env /
        boundary condition

    - challenge of diverging and multitude of application
      requirements?

     -> be relevent to large user base and longish time scales
     -> good scoping API, use cases, requests, common practice
     -> still too large, apps are moving, no 1st principle apps,
        stability vs. agility, poorly understood use cases, moving targets, 

     -> Manage Heterogeneity
     -> portable, adaptor based
     -> deployment vs coding, 'semantic equivalence'
  
     -> easy to deploy / easy to integrate
     -> components, containment, loose coupling
     -> tight coupling, leaky abstractions
  
     -> extensibility
     -> adaptors, packages, higher level abstraction
     -> adaptors: hard to enforce semantic coherence
        packages: ok (mod impl)
        HLA:      hindered by impl deficiencies
  
     -> sustainable production code in academic env
     -> community involvement, vendor uptake/buy-in, small code base,
        standardization (see Zakopane paper)
     -> integration with ecosystem (technical and social), 
  
     -> Lessons Learned
  
  \end{verbatim}

\subsection{Technical Context}

\subsection{Integration with Middleware}

\subsection{Integration with Infrastructure}

\subsection{Lessons Learned}

\section{Applications and Frameworks}
\label{apps_and_frameworks}

Intro

\section{Discussion}

\subsection{Trends and Requirements in Distributed Systems and Applications}

\subsubsection{Data as First-Class Entity}

\subsubsection{Application-Level Dynamism}

\bibliographystyle{IEEEtran} 
\bibliography{sagapaper,saga_ogf}


\end{document}

