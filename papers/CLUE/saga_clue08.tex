\documentclass[10pt,letterpaper]{article}

\usepackage{graphicx}
\usepackage{url}
\usepackage{color}
\usepackage{ifpdf}
\usepackage{wrapfig}
%\usepackage[cm]{fullpage}
\usepackage{textcomp}
\usepackage{srcltx}
\usepackage{fancyhdr}
\usepackage{setspace}

% Space saving
\usepackage[small,compact]{titlesec}
\usepackage[small,it]{caption}

\renewcommand\floatpagefraction{.9}
\renewcommand\topfraction{.9}
\renewcommand\bottomfraction{.9}
\renewcommand\textfraction{.1}
\setcounter{totalnumber}{50}
\setcounter{topnumber}{50}
\setcounter{bottomnumber}{50}

% set left and right margin, and top margin
%\setlength{\topmargin}{0cm}
%\setlength{\headheight}{0cm}
%\setlength{\headsep}{0cm}
%\setlength{\oddsidemargin}{0cm}
%\setlength{\evensidemargin}{0cm}

\textwidth = 6.5315 in
\textheight = 9.0315 in
\oddsidemargin = 0.0 in
\evensidemargin = 0.0 in
\topmargin = 0.0 in
\headheight = 0.0 in
\headsep = 0.0 in
%\parskip = 0.0in
%\parindent = 0.5cm
\parskip = 0.0716in
\parindent = 0.0cm
\textfloatsep = 0.1in

% set textwidth to remainder
%\newlength{\mytextwidth}
%\setlength{\mytextwidth}{\paperwidth}
%\addtolength{\mytextwidth}{-5cm}
%\setlength{\textwidth}{\mytextwidth}

% set text height to remainder
%\newlength{\mytextheight}
%\setlength{\mytextheight}{\paperheight}
%\addtolength{\mytextheight}{-5cm}
%\setlength{\textheight}{\mytextheight}

\ifpdf
  \pdfinfo{ /Author (Shantenu Jha  et al.)
    /Title (Programming Abstractions using SAGA)
  }

  \usepackage[pdftex,colorlinks=true, linkcolor=blue,citecolor=blue,
       urlcolor=blue]{hyperref}
\fi

\graphicspath{{images/}{../images}}

\newcommand{\projectname}{\textit{SAGA-CLUE}}

%\newcommand{comment#1{{\textcolor{magenta}#1}}

\newcommand{\projectnamefull}{\textit{\projectname: Programming Abstractions using SAGA } }

\newcommand{\upup}{\vspace*{-0.5em}}
\newcommand{\up}{\vspace*{-0.25em}}

\newcommand{\mywp}[2]{
  \hline
  \multicolumn{2}{|c|}{
    \rule[-.5em]{0em}{2em}
    \B{\large Work Package #1: }{\large #2}
  }\\
}

\newcommand{\mytask}[1]{
  \multicolumn{2}{|l|}{
    \rule[-.8em]{0em}{2em}
    \I{\large #1}
  }\\
}

\ifpdf
  \DeclareGraphicsExtensions{.pdf, .png}
\else
  \DeclareGraphicsExtensions{.ps, .eps}
\fi

\begin{document}

%\long\def\comment#1{{\textcolor{magenta}#1}}
\long\def\comment#1{{\bf \textcolor{magenta}{\bf #1}}}
\long\def\ccomment#1{{\bf \textcolor{blue}{\bf #1}}}
\newcommand{\C}{\comment}
\newcommand{\CC}{\ccomment}
%\newcommand{\C}[1]{}
%\newcommand{\CC}[1]{}
%\newcommand{\F}[1]{}

%\newcommand{\assign}[1]{\small \textcolor{magenta}{#1}}
\newcommand{\assign}[1]{}

\newcommand{\I}{\textit}
\newcommand{\B}{\textbf}
\newcommand{\T}{\texttt}
\newcommand{\F}[1]{\B{\textcolor{red}{FIXME: #1}}}

\newenvironment{shortlist}{
        \vspace*{-0.8em}
  \begin{itemize}
  \setlength{\itemsep}{-0.3em}
}{
  \end{itemize}
        \vspace*{-0.6em}
}

\newenvironment{shorttablelist}{
        \vspace*{-0.4em}
  \begin{itemize}
  \setlength{\itemsep}{-0.3em}
}{
  \end{itemize}
        \vspace*{-1.6em}
}

\pagestyle{fancy} 
\pagenumbering{arabic} 
\fancyhead{}
\fancyfoot{}

\lfoot{\footnotesize \projectname:\ Page \thepage}

\setcounter{subsection}{0}

\pagestyle{empty}
% REMOVE FOR FINAL SUBMISSION
% \LARGE
% {\center   {\textbf{\projectname: A Strategic Advancement for Grid
%       Applications}}}

% \large
% \vskip 2cm
% \noindent Submitted in response to Program Announcement NSF 06-7231\\
% {\bf Strategic Technologies for Cyber Infrastructure (STCI)} \vskip .3cm
% \noindent {\bf DUNS number: 075050765}

% \vspace{2.5cm}
% \noindent {\Large Principal Investigators:}
% \noindent {\bf  Shantenu Jha}, Louisiana State University\\

% \noindent {\Large Senior Investigators:}
% \noindent {\bf  Daniel S. Katz}, Louisiana State University\\
% \noindent {\bf  Gabrielle Allen}, Louisiana State University\\
% \noindent {\bf  Hartmut Kaiser}, Louisiana State University\\

% \vfill

% Date: July 17th 2008
% \newpage

%moved stuff into Summary.tex

\setcounter{page}{1}
\pagestyle{plain} 
\normalsize
%\section{Introduction}


\section{Project Overview and Description}

We have been leading the design, development and deployment of SAGA -- a high level
interface for creating distributed applications.  SAGA like much else of distributed computing, was initially designed for, and evolved with  compute-intensive applications in mind; it provides many of the application level abstractions that are required to create {\it regular computation} based applications.
However, an important  strand of our recent work with SAGA has been in the domain of
data-intensive applications, in particular, we have implemented the
MapReduce and AllPairs  programming abstractions to provide 
(algorithmic) frameworks -- which in turn are being used to rewrite  bioinformatics applications
such as sequence alignment and searches in ways that are not coventional.
For example, we are using MapReduce framework to find the number of occurences of a gene in 
a given genome and the All-Pairs framework to do multiple alignment.
The implementation of MapReduce using SAGA has been supported by Google (GSOC-08); it plans
to use a desktop-Grid framework called Bitdew, which provides some of mechanisms to
handle active data, ie., for computation to be scheduled based upon where
data resides. As part of the Google project, we are also implementing the ability for SAGA to have an adaptor to Hadoop; this feature already exists in the Java version of SAGA.

We have also been involved in a theoretical study of distributed applications and the abstractions that can be used to and are required to support commonly occuring patterns in distributed applications.
Our work has lead to arguably the most comprehensive analysis of distributed applications, how they are programmatically constructed, what patterns commonly occur and how they could possibly
be supported. The  work so far has focussed on the general class of distributed
scientific applications and not on data-intensive application.
Concurrently, we have been investigating theoretical ideas behind  how clouds (and thus
by generalization, clusters such as the IBM/Google cluster) differ
from traditional distributed systems, by focussing on system semantics --  importantly what the "restricted capabilities" of clouds imply for application usage modes and programming interface. 


Based upon our recent work we have the basic elements in order
and are well-placed  to address the following questions:
\begin{itemize}
\item What new programming abstractions (including models, languages, algorithms) can accentuate these fundamental capabilities?
\item Can existing tools be modified and/or new programming abstractions for such a data-intensive computing environment be developed to solve problems unsolvable any other way? 
\item What (new) applications can best exploit this computing paradigm?
\item Can old problems be solved in simpler or more efficient ways? 
\end{itemize}

We propose to address these questions  using the following  approach: \newline
We will extend our theoretical understanding of what the commonly occurring 
data-access patterns for data-intensive distributed applications are. We will
do this by analysing several common data-intensive applications as well as
several not so common data-intensive applications.  
MapReduce is a simple and clearly the best known example of such 
a pattern. It is only by the analysis of real scientific applications that we will be able
to discern possible other patterns. There are many patterns, akin to the Berkeley Dwarfs, such as fan-in/fan-out waiting to be discovered  for data-intensive applications.
Deriving such patterns
that are common across a range of several applications has not been done before.
Once we have an understanding we will seek to understand the abstractions --
both programming and system -- that can be used to support these patterns.

The next step will be to incorporate the necessary extensions in SAGA
so as to be able to provide the necessary interface to support the abstractions 
which in turn will be designed to support the the patterns we derive. 
We will  create a suite of the most common patterns that we derive 
and will use SAGA and its extensions to
actually implement these patterns. 
The ability to construct scientific applications by mixing and matching
patterns/abstractions embedded in frameworks
on the one hand, with different underlying protocols/scheduling/cluster environments
is a very important pre-requisite for empirical studies to understand which 
programming models work for data-intensive applications. The approach we
outline will be a major step in this direction.

%and the  abstractions to support these pattern
%we derive into the SAGA interface.

Our work will not only be at the theoretical level, but we will also
incorporate experiments that will provide an end-to-end solution of
the application development and deployment on active storage clusters.
We will write SAGA adaptors such that frameworks that support
patterns/abstractions can use Google's BigTable and hosting 
environment (Google App Engine).
%(need to be specific about what is what here -- one is a data structure
%and the other is a hosting environment).  
By providing the relevant  SAGA back-end interfaces to Google's App Engine, users will be able
to use the IBM-Google Cluster environment; additionally 
we will extend our deployment to work with Eucalyptus.
Thus our experiments will span home-made cloud
clusters running Eucalyptus
(and thus providing our own EC2 environment, but without the
cost of cylces!) whilst being able to use the IBM-Google cluster.
Rich Wolski (UCSB) and his group  are the developers of Eucalyptus, and are 
committed to supporting this effort
by working with us to provide Google App Engine adaptors to Eucalyptus (See Letter of
Commitment). The advantage
of having a local environment that interoperates with the IBM/Google cluster, is that amongst
other things, this will provide for the first studies in cloud interoperabiltiy.


We can then deploy and experiment with additional and alternate programming
models of active data
-- using data schedulers such as Stork and Bitdew. While it is unclear
what role such data-schedulers play in systems
that have active storage, it is clear that for any production science distributed cluster, there will be a 
need to have a local cluster with heterogenous extensions to the larger
cluster;  the heterogenity extensions arising in the context of active data policies and management.
Our approach provides this capability.

We will focus on two well known data-intensive applications -- Montage and LIGO  
and one very commonly used application  -- BLAST, but in its distributed
(and parallel) incarnation,  as outlined in the Paramedic project (winner of the
ISC2008 Best Paper Award, of which the PI and co-PI are co-authors). Montage and LIGO have not
been rewritten for or deployed on clusters with active storage; before we can aim to get the
desired performance, we need to be able to be able to re-architect these
application in order to take advantage of the underlying "architecture". All three
applications will be re-architected and developd using the extended SAGA and will be
deployed and tested on the IBM/Google active storage cluster in conjunction with 
a locally operated cluster with active data management and policies.


 
% - Clue is about clouds, but an important point there is
%about programming models and abstractions for clouds.
% - Distributed Programming abstrction stuff we've being
%doing; that along with the cloud paper gives us academic credibility
% - Not to forget SAGA! the actual tool, and in particular
%the fact that we have mapreduce implemented(!)
% - Additionally have asked miceli brothers to look at an adaptor for Hadoop
%(based upon their work with Fuse) (see below too: asked them to look into
%BigTable and Google's app engine)
% - Rich Wolski who has been making waves with eucalyptus is trying to push
%the use of eucalyptus. Rich has offered the App Enginer "Adaptors" to us for
%eucalpytus and EC2. If you are wondering why the emphasis on eucalyptus its
%because its a cloud we build locally and don't pay amazon/ibm/google..

%So it would go something like this:

%       Applications
%         |
%       Mapreduce (and other frameworks)
%         |
%        saga
%       /   \
%       App Engine
%      /     \
%    eucalyp  EC2 amazon


%- and obviously the programming models and applications would be written coded
%in saga eg map reduce.

%
% - not many sofar discuss classification, or (tataa) programming models
% from a more abstract pint of view.
%- There is of course the viable question if virtualization
%(which is prevalent in todays clouds) make programming models, or more specific make explicit support for a
%diversity of programming models obsolete.  
%I don't think that raising these questions in the proposal are an inhibitor in that call, as they explicitly ask for a
%focus on computer science, i.e. on the academic par

%Re-implementing Hadoop and MapReduce are probably not good
%enough to put us into the category of 'Data Intensive
%Applications', at least not w/o additional context.  IIUC,
%the NFS want's to see real data, and real algorithms on
%these data.  I can well imagine that the quality of the
%proposal is (partly) judged by the quality and quantity of
%these data.

%A key impediment in the accelerated development
%and deployment of applications is the scarcity of high-level
%application programming abstractions that bridges the divide between
%the needs of applications and the capabilities offered by
%middleware.  The lack of such application-level programming
%abstractions is compounded by the fact that there exist often
%incompatible Grid middleware systems in both research and production
%environments.  

%Another way in which appliction development is being retarded by a lack of
%suitable abstractions is in the development of ``first principles''

%To address these challenges and in particular to find a solution to
%the universal, apparently intractable problem of successfully
%Grid-enabling applications, several applications groups expressed the
%desire for a simple programmatic interface that is widely-adopted and
%available.  The goal of such an interface would be to provide a ``grid
%counterpart to MPI'' (at least in impact if not in details) and that
%would supply developers with a simple, uniform, and standard
%programmatic interface with which to develop applications.  Thanks to
%the efforts of many contributors, an initial specification of such a
%``grid counterpart to MPI'' now exists -- the Simple API for Grid
%Applications (SAGA)~\cite{saga-core}. SAGA is now on the threshold of
%becoming an Open Grid Forum (OGF)~\cite{ogf} technical recommendation.
%This is representative of the near universal support for SAGA, and
%agreement on its critical role in facilitating the development of Grid
%applications (see letters of support).  The need now is to demonstrate
%the unique benefits of adopting the SAGA paradigm by delivering
%concrete examples of advanced Grid applications.

%In this work we propose to develop, using SAGA as the basis, the
%programming abstractions necessary to support first principles Grid
%programming and to make those abstractions available to the TeraGrid
%user community.  We will validate and extend SAGA so that it can
%support ....

%To drive this development, we will use
%Montage and LIGO as the primary applications.

%\noindent{\bf SAGA: A High-level Distributed Programming Interface:}
%The fundamental challenge in Grid computing is reducing the barrier
%for Grid application development. To achieve this, it is critical to
%provide the right abstractions at the applications level, to enable
%applications to be developed independent of the specifics of Grid
%middleware distributions and, applications once developed, must remain
%immune to the evolution and dynamics of diverse Grid environments.
%Effective application development requires \I{simple} interfaces to
%allow \I{uniform} access to the different Grid functionalities which
%must be \I{stable} (i.e. new versions are compatible with existing
%code) and \I{easy} to deploy and use at runtime.  Grid environments
%are built on a variety of hardware and software architectures and thus
%applications need to be Grid \I{portable}, and hence require
%implementations that provide these features over a range of platforms.
%Additionally, any such programming interface should cover a broad
%range of different programming paradigms and usage scenarios and
%therefore should not be restrictive.  SAGA addresses these challenges
%by providing a programming interface that integrates the most common
%Grid programming abstractions while respecting critical {\it
%  application} level requirements, namely, those of simplicity,
%stability, portability, and uniformity.  SAGA is designed with the
%fundamental aim of enabling applications to utilize Grids, by
%providing a high-level, semantically consistent programming
%abstraction which provides a uniform interface to distinct flavors and
%versions of Grid middleware distributions.

%
%\noindent{\bf  Programming from First Principles:}
%\noindent {\bf Why SAGA?:} 

%\noindent{\bf Project Plan:} The project plans for research and
%development effort at three levels: the application level, the core
%SAGA effort and SAGA deployment (middleware bindings and outreach).

%\noindent{\it Application Level:} 

%\noindent{\it SAGA Level:} To support the true Grid nature of
%\projectname, extensions to the current SAGA core API specification
%will be required.  In order to validate and deploy \projectname\ there
%will be explicit coupling between the developers of the SAGA
%specification and the developers on the one hand, with the developers
%of GridSAT (integration efforts).

%\noindent {\it Bindings to Middleware:} 

%\noindent{\it Education and Outreach:} In order to ensure far-reaching
%impact we will develop a SAGA tutorial and user documentation and
%manual.  The preparation of SAGA educational material which requires
%small effort (relative to the specification and implementation phase)
%will go a long way in facilitating uptake of SAGA for legacy and novel
%applications. We will use LONI community and support structure for
%outreach and training as we develop this, and then TeraGrid support
%for broader outreach across the USA.

%\noindent{\bf Impact:} 

%\noindent{\bf Validation:} 
%The effectiveness of SAGA's approach will be testable both
%qualitatively and quantitatively....

%\noindent Section 2 outlines the work we have already accomplished
%that has both enabled and emboldened us to propose this ambitious
%project with far reaching impact. Section 3 presents a detailed
%coherent project plan with work-packages that have well defined scope
%and clear deliverables.  Section 4 discusses our efforts in relation to
%other attempts within the community. We provide a detailed description
%of the intellectual merit and broader impact of our proposal in
%section 5.  Section 6 highlights the Project management and justifies
%the resources requested. 

%\section{Previous Work: SAGA }

%\subsection{SAGA}

%The effort required to drive the specification of the SAGA API -- from
%the initial applications requirements capture phase, to the design
%team meetings, through to bringing it to the threshold of becoming an
%OGF standard -- has required a minimum of
%10 person years of effort. Although led by the PI and colleagues, it
%has relied on contributions from several people from different
%institutions worldwide; this has been essentially an unfunded and
%self-driven effort.

%In addition, a significant amount of effort at the CCT/LSU has been
%devoted to the design and implementation of the SAGA-engine.  Our
%proposed work builds upon this strong foundation, as well as being in
%synergy with a recent Open Middleware Infrastructure Institute (OMII)
%UK grant (Sec. ~\ref{oh-my-money}) to develop bindings for their
%middleware distribution.

%\subsubsection{SAGA API: Developing the OGF Application Standard}

%The SAGA API specification, as it currently stands, is derived from an
%analysis~\cite{saga-req} of a large number and variety of application
%use cases~\cite{saga-uc} -- approximately 20 -- submitted to the SAGA
%Research Group at OGF as part of the specification process.  {\bf The
%large number of applications driving the specification, forms the
%basis for claims that SAGA represents the needs of a broad range of
%Grid Applications}.

%The SAGA Core API specification (covering jobs, file and replica
%management, data access, data streams, remote procedure calls, and
%asynchronous operations and notification) is moving towards an OGF
%proposed recommendation status (target release date: June 2007).
%After a rigorous evaluation process, the SAGA specification has
%received the approval of the OGF steering committee.  It has now just
%exited the public review \& comment period with mostly encouraging
%responses and minor technical details and adjustments.

%The road-map for SAGA calls for it to start from a simple, very
%focused application level interface and to gradually -- with due
%testing, validation and user involvement -- evolve into a more
%comprehensive application development interface.  Additional API
%packages (the SAGA API is an extensible framework) that include
%support for checkpoint and recovery, message exchange,
%application-level information storage, and access to information and
%monitoring services are currently in draft status; a draft
%recommendation is being targeted for the end of 2007.  The next major
%areas to be covered by the SAGA API effort are support for database
%access (specifically to OGSA-DAI), and for workflow (including job and
%task dependencies).  These packages are scheduled for release in 2008.
%It is interesting to note SAGA's road-map mirrors adheres very
%accurately to the user requirements of the TeraGrid. Although a pure
%coincidence, it provides confidence in the scoping process undertaken
%by the OGF SAGA groups.

%The need for SAGA, as well as the particular specification proposed by
%the SAGA-WG have support from the global Grid community.  The PI and
%his team have led the SAGA efforts through conception, design, and
%specification stages and continue to be involved at OGF to ensure that
%SAGA will slowly but surely evolve into OGF's application domain
%flagship effort\cite{tsc}, and will be the front-end of the OGF
%standardization stack.

%\subsubsection{SAGA Engine} 

%This proposal is by the team that has not only driven the SAGA
%Research Group efforts at the OGF, but that is also responsible for
%providing initial prototype of SAGA. This ensures a high quality and
%fully conforming implementation of all SAGA related modules.  The C++
%SAGA implementation covers the entire SAGA core
%specification~\cite{saga-core}. The architecture of this
%implementation (Figure~\ref{fig:arch}) is based on a layered scheme
%requiring special components (called adaptors) to be supplied to bind
%to the available diverse middleware services.

%\begin{wrapfigure}{R}{0.5\textwidth}
%%\up\up
%  %\includegraphics[width=.5\textwidth]{saga_arch_overview}
%  \caption{\label{fig:arch} \small
%  	\B{SAGA architecture: } A lightweight engine dispatches SAGA 
%  	calls to dynamically loaded middleware adaptors. 
%  }
%  \vspace*{-1em}
%\end{wrapfigure}

%The C++ implementation builds upon a plugin-based mechanism to
%allow for bindings to a variety of Grid middleware.  
%The chosen architecture enables full portability of the application
%across different middleware and grid environments.

%SAGA C++ language bindings developed for the engine are intended to
%grow into the complete SAGA C++ reference
%implementation~\cite{saga-c++-engine}.  Additional bindings
%for Python and other languages are being planned.

%A number of partial SAGA implementations are under way (see
%Section~\ref{subsubsection:other_saga_work}).  They strive to implement
%the SAGA API specification on several middleware distributions, for a
%range of programming environments and application frameworks.

%%\subsubsection*{NSF Supported}
%%\subsection{Background Implementation Work}

%\section{Proposed Research and Development Plan  \assign{}}

%\begin{wrapfigure}{R}{0.48\textwidth}
%  \vspace*{-1em}
%%  \includegraphics[width=.48\textwidth]{wp_overview}
%  \caption{\label{fig:wps} \small
%  	\B{Architecture overview: } Pictorial representation of the scope 
%  	and distribution of work packages
%  }
%  \vspace*{-1em}
%\end{wrapfigure}

%%\input{workpack-table}

%\up\up

%\subsection*{SAGA is usable by a broad range of applications}
%\up\up

%An indicator of SAGA's strategic importance is also a measure of
%SAGA's success, namely the number of new grid applications that are
%developed, the kind of novel usage scenarios, as well as the ease and
%wider usage of existing applications that SAGA enables.  An important
%aspect of the proposed work and further confirmation of the {\it
%  strategic} role of SAGA is our aim of enabling the development of a
%significant number of diverse applications -- varied in discipline,
%type (data versus compute) and Grid functions involved.  Table
%~\ref{saga-takeover-world} provides a reduced overview of some target
%applications and application areas.

%\begin{table}
%\begin{center}
%  \begin{tabular}{|l|l||l|l|}
%      \hline \B{Area} & \B{Application}  &  \B{Areas}  & \B{Applications} \\
%      \hline Astronomy (Workflows) & Montage (JPL/NASA)
%      &  Coastal Modelling &  UCoMS \\
%      \hline Biological Sciences & Replica Exchange 
%      & Lattice Fluid  & Hypo4D (Tufts) \\
%      & (CCT/LSU) & Dynamics & \\
%      \hline Computer Science & CoreGrid (Rosa Badia) 
%      & Numerical Relativity & Cactus (Allen) \\
%      \hline Visualization & DIVA 
%      & Interactive Vis. & LONI (Hutanu) \\
%        & (John Shalf/LBNL) & & \\
%      \hline Bioinformatics &  Protein Pattern 
%        & Comp. Biology & Hybrid MC-MD \\
%        & Scanning & &  (Konerding/LBNL)  \\
%      \hline \hline
%\end{tabular} 
%\caption{\small Some of the science areas and applications that have either 
%  submitted use cases in document GFD.70~\cite{saga-uc} or are
%  possible application for development in co-operation with the SAGA
%  team}\label{saga-takeover-world}
%  \vspace*{-1em}
%\end{center}
%\end{table}

%
%%Previously we have used TeraGrid resources to conduct a binding affinity study of the protease inhibitor saquinavir with G48V and L90M containing mutatants which showed good agreement with experiment \cite{Stoica2008}. % uring SuperComputing 2007 we used
Using Lonestar (TG allocation number TG-ASC070019T) to expand this work by running 10ns-long simulations on 5 different HIV protease systems bound to 5 different FDA-approved drugs: amprenavir, indinavir, lopinavir, ritonavir, and saquinavir. We compared our binding energy results for each of the 5 protease genotypes in the MDR set to experimental values published in \cite{Ohtaka2003}. Our results showed varying correlations to experimental trends, with the 5 ritonavir systems showing good correlation to experimental whilst the 5 saquinavir systems showed poor correlation. In light of these results further work was done both on Lonestar (TG allocation numbers TG-ASC070019T and TG-DMR070014N) and more recently Ranger (TG allocation numbers A-rfUser1 and TG-ASC090009) to determine whether repeating the study as an ensemble of MD models would improve correlations. The results showed significantly improved correlations and have led us to refine our simulation protocol and reparameterise a number of inhibitors. An article focussed primarily on our work on the lopinavir bound system has recently been accepted for publication \cite{Sadiq2010}.

In addition to this we have performed a study into the comparative dynamics of different liganded forms of reverse transcriptase, concluding that the binding of the inhibitors efavirenz and nevirapine alter the motions of the catalytically important fingers domain of RT and that these motions are uncorrelated to fluctuations in the drug binding energy (as yet unpublished). We also performed preliminary investigations into the efficacy of the ensemble methodology for calculating binding affinities on NVP bound systems, again concluding that the convergence properties are improved over single trajectory. These results also allowed us to differentiate binding affinity of the L100I and L100I/K103N sequences from the wildtype using simulations of the ligand bound enzyme alone which indicates that they do not significantly affect the creation of the binding pocket. All this work was performed using Ranger. 

These simulations were orchestrated using our automated simulation tool, the Binding Affinity Calculator (BAC), which makes use of AHE \cite{coveney2007,zasada2009} to deploy and run simulations and then retrieve data. Two papers describing this tool and its potential use as a clinical support tool have been published\cite{Sadiq2008, Sadiq2008a}. The BAC allows us to distribute the simulation and analysis sections of our workflow over a number of geographically disparate locations. The mutation and hydration of crystal structure based models takes place on local resources at UCL, using the AMBER software suite, while the actual simulations can be performed in NAMD either on TeraGrid or UKNational Grid Service (NGS) computers and/or EU DEISA grid. The final analysis of the resultant trajectories, either on local recources or the NGS, is performed using the AMBER MD suite of codes. In the last year, BAC has been extended to allow the simulation of the epidermal growth factor receptor (EGFR) and the molecular dynamics code GROMACS \cite{Hess2008}. This new functionality has facilitated a study of the binding of three inhibitors (AEE788, AFN941 and getfitinib) bound to EGFR, performed on Ranger.

In the area of bio-minerals our previous simulations have utilized periodic boundaries on the clay sheets, allowing a simulation cell to represent an infinite clay platelet \cite{JPCC_2007,Thyveetil,Thyveetil_JACS, Soft_Matter1}. These large-scale, fully atomistic simulations, approached the size of a physically realistic platelet. From this, we were able to calculate mesoscopic and macroscopic properties directly from molecular dynamics simulations in the absence of finite-size effects of both clay nanocomposites\cite{JPCC_2007,Thyveetil, Soft_Matter1} and bio-composites~\cite{Thyveetil_JACS}.  Using resources from TG-ASC070019T we extended these simulations to calculate the mechanical response of poly(ethylene oxide) polymer-clay nanocomposites, separating the response into contributions from the polymer and clay mineral layer~\cite{Soft_Matter1}. This separation technique allowed us to determine how the clay-polymer elastic properties change with distance from the clay surface. This is the first time the effect of mineral layers on the elastic modulus of polymeric materials in the vicinity of a mineral surface has been calculated. This result is a vital first step to understanding the enhancement mechanism of nanocomposites and the role of the very large surface area of the clay mineral layer on the surrounding medium.  We have also examined the mechanisms by which clay mineral layers buckle in clay-polymer nanocomposites under compressive stress. We find that a clay sheet remains stable in a flat state until a critical compressive strain is reached, at which point it buckles, and regains its uncompressed area. Over this buckling transition, the Poisson ratio of the clay sheets turns negative, a property which has been predicted for 2-dimensional sheets ~\cite{Soft_Matter2}. This is the first time such behaviour has been seen in a molecular simulation of mineral layers. The large scale allowing us to probe large buckling wavelengths that are inhibited in smaller scale simulation.
% To simulate a realistically sized clay platelet required very large scale simulations, made possible using the resources available via the LRAC allocation.
%%\newpage
%%\setcounter{page}{1}
%%\pagestyle{plain} 
%%\pagenumbering{roman}

\bibliographystyle{unsrt}
\bibliography{stci_saga,grid}
\end{document}

