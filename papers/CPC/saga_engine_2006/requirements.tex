% $Header: /projects/VU-SAGA/Papers/saga_engine_2006/requirements.tex,v 1.15 2006/10/11 01:30:32 hkaiser Exp $

 As mentioned in the introduction, the SAGA C++ reference
 implementation must cope with a number of very dynamic requirements.  
 Additionally, it must provide the ``simple'' and
 ``easy-to-use'' API the SAGA standard is intended to specify.  
 We describe the resulting requirements in some detail
 motivating our SAGA implementation design described in 
 section~\ref{sec:generaldesign}.

We identified several main characteristics the SAGA C++ reference 
implementation must provide, if any of these properties is missing, 
acceptance in the targeted user community will be severely limited:

\begin{shortlist}
	\item It  must cope with evolving grid standards 
			  and changing grid environments.
  \item It must be able to cope with future SAGA
			  extensions, without breaking backward compatibility.
	\item It must shield application programmers
			  from the evolving middleware, and X
			  should allow various incarnations of grid middleware to co-exist.
	\item It must actively support fail safety mechanisms, and hide
		    the dynamic nature of resource availability.
	\item It must be portable and, both
				syntactically and semantically, platform independent.
	\item It must allow
				these and other latency hiding techniques to be implemented.
	\item It must meet other end user
				requirements outside of the actual API scope,
				such as ease of deployment, ease of configuration,
				documentation, and support of multiple language bindings.
\end{shortlist}


