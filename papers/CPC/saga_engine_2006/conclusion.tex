% $Header: /projects/VU-SAGA/Papers/saga_engine_2006/conclusion.tex,v 1.10 2006/10/11 02:55:39 gallen Exp $

We have described the C++ reference implementation of the SAGA API,
which is designed as a generic and extensible API framework:  it
allows for the extension of the SAGA API,
easily usable for other APIs); 
it allows for run-time extension of
middleware bindings, 
and it allows for orthogonal optimizations and features,
such as late binding, diverse adaptor selection strategies, and latency
hiding.  The used techniques enable
these features, amongst them the application of the PIMPL paradigm for
a complete class hierarchy and generic call routing.

These implementation techniques incur a certain overhead, however, in
grid environments the runtime overhead is usually vastly dominated by
communication latencies, so that \I{\B{this} overhead does not
matter}.  The lesson learned is that distributed environments
\I{allow} for fancy mechanisms, which are too expensive in local
environments.  Fail safety and latency hiding mechanisms are
more important than, for example, virtual functions, late
binding, and additional abstraction layers.  
