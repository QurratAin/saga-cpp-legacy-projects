\documentclass{CFD2010paper}

%\usepackage{graphicx}
\usepackage[pdftex]{graphicx,color}

\title{ INSTRUCTIONS TO PREPARE THE FULL PAPER FOR THE V~EUROPEAN CONFERENCE ON
COMPUTATIONAL FLUID DYNAMICS (ECCOMAS CFD 2010)}

\author{First A. Author$^{*}$, Second B. Author$^{\dag}$ and Third C. Coauthor$^{\dag\dag}$}

\heading{First A. Author, Second B. Author and Third Coauthor}

\address{
$^{*}$Institution of first author,\\
address\\
e-mail: name@e-mail.address
\\
$^{\dag\dag}$Institution of second (and third) author\\
address\\ e-mail: \{name2,name3\}@e-mail.address}

\keywords{Fluid Dynamics, CD-Rom, Instructions, Deadlines,
Conference site}

\abstract{Using a computational framework (also known as component architecture) for application codes can have several benefits: (1) numerical method and parallel implementation remain separated, (2) the application remains portable to many computing architectures and system configurations, and (3) the resulting modular structure can enable inter-disciplinary and multi-physics collaborations.  Cactus is one such general-purpose application framework that is seeing widespread use in relativistic astrophysics since 1997, and is also being used in CFD.
While Cactus is highly successful in relativistic astrophysics, it so far failed to attract the attention of main stream CFD researchers.  We assume that this is mainly caused by a lack of standardisation in existing CFD components, and by the lack of support for multi-block mesh systems.  To address these issues, current research focuses on designing a standard Cactus CFD Toolkit and developing its relevant components, including a multi-block driver, implementations of standard incompressible/compressible CFD methods, and applying it to large-scale CFD simulations with complex geometries.
Our toolkit design consists of three layers comprising the low-level computational toolkit, the CFD solver, and a high-level application layer.  The multi-block driver is integrated with the current Cactus driver systems as part of the computational toolkit, managing memory, parallelism, and I/O.  This low-level computational layer is shared with the astrophysics community, which uses it for highly supersonic relativistic flows.  Components supporting CFD flux schemes, time integration etc. are part of the solver layer, enabling users to either use current implementations or replace them with their own implementations as necessary.  The application layer combines these with initial and boundary conditions and other necessary elements to form complete simulations.
Our CFD toolkit has been validated with a supersonic flow field around a projectile with the base, and applied to the flow simulation around the ARA-M100 wing-body configuration.  We will also demonstrate the parallel performance of our toolkit.
== Needs correction! ==}



\begin{document}
%\maketitle



\newpage

\section{INTRODUCTION}
The proceedings will be edited in a CD-ROM, including the texts and figures, using Portable Document Format (PDF). Authors should submit their paper electronically through the web page of the Conference, \underline {http://www.eccomas-cfd2010.org/} before March 31, 2010. The paper should be written following the format of the LaTeX and Word macros that can be found in the Conference webpage under Instructions for Authors. The file has to be translated to Portable Document Format (PDF) before submission through the Conference webpage. The organizers do not commit themselves to include in the Proceedings any paper received later than the official deadline. At least one of the authors must register and pay his/her registration fee before the deadline set at April 15, 2010 for their paper to be included in the final program of the Conference.
In addition to the CD-ROM, a book of one-page abstracts will be edited containing the basic information about each paper to be presented at the Conference. Note that these instructions are for contributed manuscripts and not for the invited lectures. The manuscripts supporting the invited lectures have a different format.

\section{GENERAL SPECIFICATIONS}
The paper must be written in English within a printing box of 16 cm x 24 cm, centered in the page. The paper including figures, tables and references must have a minimum length of 4 pages and must not exceed 20 pages. Maximum file size is 6 MB.

\section{TITLE, AUTHORS, AFFILIATION, KEY WORDS}
The first page must contain the Title, Author(s), Affiliation(s), Key words and the Abstract. The second page must begin with the Introduction. The first line of the title is located 3 cm from the top of the printing box.

\subsection{Title}
The title should be written centered, in 14pt, boldface Roman, all capital letters. It should be single spaced if the title is more than one line long.

\subsection{Authors}
The author's name should include first name, middle initial and surname. It should be written centered, in 12pt boldface Roman, 12pt below the title.

\subsection{Affiliation}
Author's affiliation should be written centered, in 11pt Roman, 12pt below the list of authors. A 12pt space should separate two different affiliations.

\subsection{Key words}
Please, write no more than six key words, which will be used to compile the CD-ROM index. They should be written left aligned, in 12pt Roman, and the line must begin with the words {\bf Key words}: boldfaced. A 12pt space should separate the key words from the affiliations.

\subsection{Abstract}
Use 12pt Italic Roman for the abstract. The word {\bf Abstract} must be set in boldface, not italicized, at the beginning of the first line. The abstract text should be justified and separated 12pt from the key words, as shown in the first page of these instructions. The abstract should neither be too short nor exceed the first page.

\section{HEADINGS}
\subsection{Main Headings}
The main headings should be written left aligned, in 12pt, boldface and all capital Roman letters. There should be a 12pt space before and 6pt after the main headings.

\subsection{Secondary Headings}
Secondary headings should be written left aligned, 12 pt, boldface Roman, with an initial capital for first word only. There should be a 12pt space before and 6pt after the secondary headings.

\section{EDITORIAL HEADING}
The first page will include the Editorial Heading, as shown in the first page of these instructions. In addition, successive pages will include the name of the authors. The Editorial Heading must be included in the paper by the authors.

\section{TEXT}
The normal text should be written single-spaced, justified, using 12pt (Times New) Roman in one column. The first line of each paragraph must be indented 5 mm. There is not interparagraph spacing.

\section{PAGE NUMBERS}
A paper will be found in the CD-ROM by the author's name or using the key words. Page numbering is unnecessary for this purpose. In any case, in order to organize the paper, it is better to number the pages. Write the page number centered at the bottom of each page, with 12pt Roman. Note: Page numbers are not included in the printing box.

\section{FIGURES}
All figures should be numbered consecutively and captioned. The caption title should be written centered, in 10pt Roman, with upper and lower case letters.
A 6pt space should separate the figure from the caption, and a 12pt space should separate the upper part of the figure and the bottom of the caption from the surrounding text.
Figures should be included within the text, see figure \ref{figexample}, rather than added at the end of the paper.
%
\begin{figure}[ht]
\centering
\includegraphics[width=0.4\linewidth]{picture2}
\vskip-0.2cm
\caption{Example of figure}
\label{figexample}
\end{figure}

\section{EQUATIONS}
A displayed equation is numbered, using Arabic numbers in parentheses. It should be centered, leaving a 6pt space above and below to separate it from the surrounding text.

The following example is a single line equation:
\vskip-.6cm
\begin{eqnarray}
Ax = b
\end{eqnarray}

The next example is a multi-line equation:
\vskip-.6cm
\begin{eqnarray}
Ax = b \\
Ax = b \nonumber
\end{eqnarray}

\section{TABLES}

All tables should be numbered consecutively and captioned, the caption
should be 10pt Roman, upper and lower case letters.

\begin{table}[ht]
\begin{center}
\begin{tabular}{*{3}{|c}|}
\hline
C11 & C12 & C13 \\
\hline
C21 & C22 & C23 \\
\hline
C31 & C32 & C33 \\
\hline
C41 & C42 & C43 \\
\hline
C51 & C52 & C53 \\
\hline
\end{tabular}
\end{center}
\vskip-.4cm
\caption{Example of the construction of one table}
\vskip-.4cm
\end{table}

A 6pt space should separate the table from the caption, and a 12pt space
should separate the table from the surrounding text.

\section{FORMAT OF REFERENCES}

References should be quoted in the text by superscript
numbers$^{1,2,3}$, and grouped together at the end of the paper in
numerical order as shown in these instructions.

\section{CONCLUSIONS}

\noindent - \hskip.5cm Papers should be submitted electronically via the webpage of the Conference.\\
\noindent - \hskip.5cm Papers should be written following the format of the LaTeX and Word macros\\
\indent \hskip.3cm for submission that can be found in the Conference webpage.\\
\noindent - \hskip.5cm They must~ be translated ~to Portable Document Format~ (PDF)~ before submission\\
\indent \hskip.3cm through the Conference webpage.\\
\noindent - \hskip.5cm Deadline for the submission of the paper posted in the webpage must be respected.\\
\noindent - \hskip.5cm The organizers~ do not commit themselves to include in the Proceedings~ any paper\\
\indent \hskip.3cm received later than the deadline.\\
\noindent - \hskip.5cm  At least one of the authors should register and pay his/her registration fee before the
\indent \hskip.3cm first registration deadline for their paper to be included in the final program of the
\indent \hskip.3cm Conference.
\vskip1.5cm

\begin{thebibliography}{99}
\bibitem{Zienkiewicz} O. C. Zienkiewicz and R. C. Taylor, The finite element method, 4th Edition, \textit{McGraw Hill}, \textbf{Vol. I} (1989), \textbf{Vol. II} (1991).
\bibitem{Idelsohn} S. Idelsohn and E. O\~nate, Finite element and finite volumes, Two good friends, \textit{Int. J. Num. Meth. Engng.}, \textbf{37}, 3323--3341 (1994).
\bibitem{Abgrall} R. Abgrall, M. Ricchiuto, N. Villedieu, C. Tav\'e\ and H. Deconinck, Very High Order Residual Distribution On Triangular Grids, In proceedings of the \textit{European Conference on Computational Fluid Dynamics}, ECCOMAS CFD 2006, P. Wesseling, E. O\~nate and J. Periaux Eds., Egmond aan Zee, Netherlands, Paper n$^{\circ}$583 (2006)

\end{thebibliography}




\end{document}

