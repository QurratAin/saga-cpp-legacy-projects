\documentclass[10pt,conference,final,letterpaper,twoside,twocolumn,]{IEEEtran}

\usepackage{graphicx}
\usepackage{color}
\usepackage{url}
\usepackage{ifpdf}
\usepackage{hyperref}
\usepackage{xspace}

\setlength\parskip{-0.015em}
\setlength\parsep{-0.15em}

\newenvironment{shortlist}{
	\vspace*{-0.85em}
  \begin{itemize}
 \setlength{\itemsep}{-0.3em}
}{
  \end{itemize}
	\vspace*{-0.6em}
}

\usepackage{fancyhdr}
\setlength{\headheight}{16.0pt}
\pagestyle{fancy}
\headheight = 0pt
\headsep    = 25pt
\fancyhf{}
\fancyhead[OC]{\bf {\it \footnotesize{Jha et al: A Case for SAGA as an Access Layer for DCI}}}

\newif\ifdraft
\drafttrue
\ifdraft
 \newcommand{\amnote}[1]{  {\textcolor{magenta} {***AM: #1}}}
 \newcommand{\jhanote}[1]{ {\textcolor{red}     {***SJ: #1}}}
 \newcommand{\tmnote}[1]{  {\textcolor{blue}    {***TM: #1}}}
\else
 \newcommand{\amnote}[1]{}
 \newcommand{\jhanote}[1]{}
 \newcommand{\tmnote}[1]{}
\fi

\newcommand{\dn}{\vspace*{0.33em}}
\newcommand{\dnn}{\vspace*{0.66em}}
\newcommand{\dnnn}{\vspace*{1em}}
\newcommand{\uppp}{\vspace*{-1em}}
\newcommand{\upp}{\vspace*{-0.66em}}
\newcommand{\up}{\vspace*{-0.33em}}
\newcommand{\shift}{\hspace*{1.00em}}

\newcommand{\T}[1]{\texttt{#1}}
\newcommand{\I}[1]{\textit{#1}}
\newcommand{\B}[1]{\textbf{#1}}
\newcommand{\BI}[1]{\B{\I{#1}}}
\newcommand{\F}[1]{\B{[FIXME: #1]}}
\newcommand{\TODO}[1]{\textcolor{red}{\B{TODO: #1}}}

\begin{document}

\title{Towards Grid-Cloud Interoperabilty: A Case for SAGA as Access
  Layer for OCCI backed Cloud Infrastructures}

\author{Shantenu Jha$^{*1,2}$, Thijs Metsch$^{3}$, Andre Merzky$^{1}$\\
  \small{\emph{$^{1}$Center for Computation \& Technology, Louisiana State University, USA}}\\
  \small{\emph{$^{2}$Department of Computer Science, Louisiana State University, USA}}\\
  \small{\emph{$^{3}$Platform Computing, Germany}}\\
  \small{\emph{$^{*}$Contact Author \texttt{sjha@cct.lsu.edu}}}
  }

\maketitle

\section*{Abstract}

\section{Introduction}
\label{sec:intro}
 

Outline:
\begin{itemize}
\item Why Interoperability? [Cloud-Grid, Cloud-Cloud] discuss using
  Application exemplars
\item How? discuss ALI vs SLI, fast-track vs deep-track
\item What? We propose - using SAGA on OCCI -- which is ALI using SLI
  (protocols)
\item This document discusses how we will implement and architect our
  approach, the advantages inherent in our approach from an
  applicaiton point-of-view
\end{itemize}


 Distributed infrastructures have been used by many applications to
 advance understanding in their disciplines.  The requirements and
 characteristics of applications that motivate the usage of
 distributed infrastructures are very broad, and most often differ
 from regular, HPC and HTC type applications in several fundamental
 ways.  Distributed applications often need to be designed for more
 than simple peak-utilization, e.g., the number of coupled tasks that
 need to be completed within a time window.  Equally important,
 distributed applications have a much broader range of usage
 modes~\cite{dpa-paper}.

 In particular for grids as one posisble DCI infrastructure for
 distributed applications, standards have been playing an important,
 and in fact leading, role by ensuring application and system level
 interoperability~\cite{gin,saga-gin}, so that the required usage
 modes are (a) supported by the range of production grids available,
 and (b) that distributed application are able to scale beyond the
 boundaries of any single production grid.

 Clouds are increasingly used to complement and expand the set of
 infrastructures for distributes applications.  In parallel, efforts
 to expand application and system level interoperability from grids
 towards clouds are emerging.  This paper motivates the need for such
 interoperability, and discusses one (of many) possible approaches,
 with the significant distinction of focusing on standards.

 The remainder of the paper is structured as follows: 


\section{Interoperability}
\label{sec:interop}

 \amnote{TODO SHANTENU: select appropriate use case / project for
 interop motivation}

\jhanote{SJ to write about Biology application (BFast (data-intensive)
  and Ensemble simulations (azure) ) + ExTENCI applications}

\amnote{TODO ANDRE: define ALI/SLI}

 % - why do we discuss interop
 % - difference to interoperation
 % - relation to SAGA and OCCI (complementary focus)

 There are two major approaches to allow for interoperability: system
 level interoperability (SLI) and applicaiton level interoperability
 (ALI).  SLI is the (often standards based) approach to federate
 multiple DCIs, and to allow applications to use the combined set of
 resources as if it were a single DCI.  That implies the exchange of
 accounting, authorization, authentication, logging, brokering and
 resource management information, and others, on system level.  Very
 often, SLI is implemented as delegation from on DCI to the other.

 ALI on the other hand does not require any activity on infrastructure
 level at all, but instead relies on the ability of the application to
 utlilize multiple independent DCIs at the same time.  While ALI is
 obviously usually to achieve, it is burdening the application layer
 with the task to abstract the differences between DCIs.  Further, ALI
 based systems are inherently less sophisticated and less complete
 than SLI based systems -- for example, data transfer across DCI
 boundaries will need to be routed via the application layer.


 \subsubsection*{Cloud Interoperability and Standardization}

 The need for cloud
standards and interoperabilty can be appreciated from a technology as
well as an application perspective. One example of a technology pull
for standardization is provided by need for interoperability across
{\it Specialized Clouds}, i.e., emerging customized clouds that
support specific capabilities or services, analagous to HPC grids
(TeraGrid) versus HTC grid (Open Science Grid) or Data Grids (EGEE).
The application push can be understood by the imperative to operate on
massive amounts of data {\it in situ}, which in turn involves
computation across heterogeneous distributed platforms as part of the
same application.  For example, the Earth System Grid involves peta to
exa-bytes of data, and one cannot move all data (given current
transfer capabilities), nor compute at a centralized location.  In
addition, there exist a wide range of applications that have
decomposable but heterogeneous computational tasks. It is conceivable,
that some of these tasks are better suited for traditional grids, or
on a specific cloud over another, e.g., applications in the LEAD
project, some workloads might be better placed on a data cloud whilst
some may optimally be located on regular clouds or even grids, due to
different data-compute affinity requirements amongst the tasks.

It is worth mentioning that {\it all existing} efforts at
interoperability are at the service managment, cloud network or
federation level, i.e., there does not exist any effort at providing
application-level interoperability. We aim to provide the first such
application-level interoperability, built on the back on existing
community efforts.  OCCI aims to provide remote management API for
IaaS clouds. It complements related standards of SNIA and DMTF, while
competing with more proprietary approaches such as those from VMware
or Amazon.  

\amnote{Thijs, please re-check the next paragraph - not sure if that
is still valid}

OCCI has been able to gather significant community and industry
support and uptake, e.g., early uptake in community driven cloud
stacks (OpenNebula, OpenStack, Eucalyptus etc).  OCCI is however, a
REST-ful protocal rather than an interface definition, and will thus,
at least initially, experience different, likely non-interoperable
implementations.  OCCI will thus make cloud interoperability easier,
but not simple {\it for the application developer or user}.  But by
integrating SAGA with OCCI we can provide broad coverage for
applications, i.e., SAGA as the application/client API to OCCI should
be able to shield an application from the specific OCCI
implementations as well as provide grid-cloud interoperability.


\section{SAGA}
\label{sec:saga}

 SAGA has been presented and discussed in detail
 elsewhere~\cite{sagapub...}, and has been used for application level
 interoperability before~\cite{sagainterop...}.  This section shortly
 summarizes the relevant findings.

 SAGA is an acronym for "Simple API for Grid Applications". As the
 name suggests, a simple API which facilitates the development and
 execution of distributed applications on most types of distributed
 infrastructure.  Modern distributed computing environments are very
 complex infrastructures, and allowing applications to make use of
 these complex systems is not trivial.  By defining a simple API, one
 requires those complexities to be dealt with at levels other than
 application code and development.  Simplicity of the interface is the
 primary design principle and objective of SAGA.  The fact that SAGA
 is a (set of) OGF standard(s)~\cite{sagaspecs...} ensures the
 community-wide adoption and stability of the API.  Functional goals
 of SAGA are:

 \begin{enumerate}

  \item Provide a stable programming interface to distributed
  application programmers and tool developers
 
  \item Shield developer from heterogeneous and evolving
  infrastructures and middlewares

  \item By providing the building blocks to distributed and remote
  operations enable the expression of high-level abstractions and
  support of distributed application requirements

 \end{enumerate}

 The SAGA standardization effort is closely syncronized with other
 specification and community efforts, within and outside of OGF.  In
 particular, OGF groups ensure that SAGA semantics map well to lover
 level specifications, such as JSDL, BES, GridFTP, etc.   But also,
 and possibly more importantly, it is now widely and independenly
 acknowledged~\cite{refs?} that a uniform, simple and stable access
 layer is neccessary (but not sufficient) to improve end user
 experience on distributed computing infrastructures, and that SAGA
 can indeed play that role for a specific set of use cases.  As such,
 SAGA is now integral part of the GIN (Grid Interoperation Now)
 community effort, and also plays an active role in current efforts
 like OGF's PGI group and US's XD proposals.
  
 Although SAGA is foremost an API, the SAGA distributions support end
 users in a variety of ways.  In particular, the SAGA distributions
 also include command line tools implemented via the SAGA API, and
 higher level libraries for common distributed programming patterns,
 also basing on the SAGA API.


 \subsubsection*{The Price of Simplicty\cite{sagaprice}}

  There are three main costs associated with an approach like SAGA,
  which tries to provide a simple, stable, concise and consistent
  layer on top of a very dynamic and exceedingly complex set of
  infrastructures.  First, there are performance penalties.  We show
  in~\cite{sagaperf} that those are, for almost all of our target use
  cases, acceptable, or at least manageable.  Second, almost all
  abstractions in IT are leaking once the underlying systems become
  too complex~\cite{leaky_abstractions}.  Despite trying very hard to
  pinpoint the SAGA API semantics, and to avoid abstraction leakage,
  we have to face this uphill battle.  Third, and even more
  importantly, there remains the fact that the SAGA layer itself, i.e.
  the SAGA implementation, is an very complex component.  Although
  that fact is well understood, and an intentional design artefact, it
  needs dealing with on several levels.  Amongst others, it is
  becoming increasingly clear that SAGA deployment needs to be dealt
  with at levels distinct from the SAGA end user, namely at the DCI
  (...) provider level.


 \subsubsection*{SAGA in the Clouds}

  Also the SAGA acronym contains the G-word, it is, first of all, an
  API for programming distributed applications.  As such, it is
  suitable for classical grid and non-grid DCIs, but just as well
  usable for cloud infrastructures~\cite{sagacloud...}.  In particular
  in repsect to the 'leaky abstractions' mentioned above, one has to
  acknowledge that the dynamic resource provisioning aspects of clouds
  have certain implications for the semantic interpretation of the
  higher level SAGA API.  For example, resource life time assumptions
  which are valid on Grids have a very different meaning in clouds.
  While we have shown~\cite{sagacloud...} that SAGA can handle fairly
  simple scale-out scenarios onto clouds rather well, it will require
  some additional API semantics to more efficiently, and
  transparently, use cloud resources.  Interestingly, those cloud
  requirements match really closele additional requirements we have
  been receiving from the grid application community.  For example,
  the SAGA based pilot-job implementation 'BigJob'~\cite{bigjob} is,
  at the moment, one of the dominant usage modes for TeraGrid and
  DEISA resources, for the SAGA community.  Application level resource
  allocation and reservation paradigms like pilot job and the cloud's
  dynamic resource provisioning paradigms have, on API level, many
  things in common, so that it seems sensible to expand
  SAGA\footnote{The SAGA API is, by design, extensible.  In fact,
  multiple SAGA extensions have already been defined, and are
  undergoing the same standardization process as SAGA itself.} by a
  relatively generic resource management API.  It must be noted that
  SAGA's notion of resource management has almost \I{nothing} to do
  with resource management on infrastructure level, but rather is an
  expression on how an application is discovering and allocating
  required resources.
  

 % the extenci and use case parts should move to earlier

% \subsection{Standards promote Interoperability}
% 
%  \subsubsection*{ExTENCI}
% 
%  ExTENCI\footnote{\url{https://sites.google.com/site/extenci/}} is an
%  NSF funded 2 year project to promote interoperability between the
%  TeraGrid (and/or its successor -- XD) with the OSG (and its
%  successor).  The role of SAGA is provide a common job-submission
%  mechanism across TeraGrid and OSG, for both command-line access as
%  well as for Cactus based application via Gateway.  The different
%  application scenarios that are/will be supported are:
%  
%  \begin{itemize}
%  \item Ensemble of Cactus Simulations (NumRel, EnKF (Petroleum Eng))
%  \item Multiphysics Code (GR-MHD, CFD-MD)
%  \item Spawning Simulations (Realtime ‘outsourcing’ from
%    BlueWaters/Ranger to specialised architectures or less powerful
%    resources)
%  \end{itemize}



%  \section{Need a subsection title}
%  
%   Distributed Computing is more than just submitting isolated jobs.  It
%   is also about federating resources and application components
%   dynamically; about coordinated execution of heteregeneous and dynamic
%   workloads; it is about distributed data management etc. So while SAGA
%   is used for multiple reasons, there are three primary usage modes of
%   SAGA are the following: (i) Simplifying access layer, (ii) building
%   block for tools and distributed execution execution, and (iii) a
%   distributed scripting and programming capability.
%  
%  
%    \subsection{SAGA as a Standardized Programmatic and Access Layer:
%    Advantage to DCI Providers}
%  
%     \subsubsection*{Providing Distributed Scipting and Programming
%     Capbility}
%     % The SAGA API provides very concise and high level method calls
%     % which cover the vaste majority of distributed operations, as
%     % required by the target user community -- scientific application
%     % and tool developers.
%     The Structural Biology Grid (\url{http://SBGrid.org}) currently
%     implements sophisticated analysis and user-defined pipelines.
%     However, these are inherently localized and confined to specific
%     infrastructure.  Replacing ``local python'' calls with
%     ``distributed (SAGA) python'' calls would enable the seamless
%     utilization of DCI.  This provides a simple mode of extensibililty
%     of infrastructure, without any major refactoring of code. Further,
%     as the API specification and implementation is \I{standardized},
%     and thus stable, it allows for a 'write once, run anywhere'
%     approach, which is in general not available otherwise (or at least
%     not without \I{significantly} increase of application complexity).
%     The advantages of this to the end-user is obvious; the lowered
%     barrier-to-entry for novel users and communities will increase the
%     ease and uptake of DCI thus benefitting DCI
%     providers/organizations.  As part of the ExTENCI project, SAGA will
%     make major advances towards becoming a broadly usable programmatic
%     access layer to Condor/OSG.
%  
%  \subsection{Application Development Advantages}
%  
%  \subsubsection*{Application Prototyping and Tooling}
%  
%  The SAGA Python bindings have been proven to be immensely helpful for
%  application prototyping.  But also, they are very helpful when
%  interactively testing remote operations (in the interactive Python
%  interpreter / Python shell).  Finally, it is very easy to implement
%  small command line tools in Python, which are able to mimic and test
%  smaller portions of the overall application.  For example, it is
%  straight forward to implement a specific job control component of an
%  application in a stand alone Python script, and to later include the
%  same functionality in the application proper, with the confidence that
%  the semantics of the remote operations will be well
%  preserved.
%  
%  \subsubsection*{Application Development}
%  
%  %    According to the SAGA use case analysis\cite{saga-uc}, and to our
%  %    own experience in developing distributed applications, the set of
%  %    funtionality required for such applications is rather limited.
%  %    However, as that functionality is provided in very different ways,
%  %    by the various distributed middlewares available today, application
%  %    complexity has historically increased significantly for any single
%  %    remote operation used by the application.
%  
%  
%  % \subsubsection*{Application Deployment}
%  
%  % \subsubsection*{Runtime Configuration}



% The following three most common deployment modes were discussed:
% 
% \begin{itemize}
% 
% \item "Mode 1": SAGA is deployed on the end user's
%   computer. Non-intrusive to the Grid Middleware this is by far the
%   most popular deployment mode for SAGA today. At the same time,
%   however, it is completely beyond control, influence or support for
%   EGI.eu.
% 
% \item "Mode 2": SAGA is installed independently on VO infrastructure
%   and made available through the VO portal functionality. Likewise,
%   this mode of access is non-intrusive, but through collaboration and
%   strong binding through MoUs with EGI.eu somewhat more apt to
%   influence and coordination.
% 
% \item "Mode 3": SAGA is installed directly on the Grid infrastructure,
%   along with or part of, the Middleware. This installation mode would,
%   though highly intrusive, offer most potential to Grid end users as
%   it allows to write truly distributed (domain specific) applications
%   with relative ease.
% 
% \end{itemize}
% 
% \begin{figure}[t]
% \centering
% \begin{tabular}{ll}
% Use Case & Deployment Mode \\
% \hline
% Applications seeking Interoperability & I, II III \\
% Virtual Physiological Human & I, III\\
% FEDEX (Multi-Scale, Multi-Physics Simulations)  & I, III\\
% NeuGrid  & I, II, III\\
% BigJob based Applications (SAGA Pilot-Job)  &  III\\
% Science Gateways (eg Computational Biology)  & II, III \\
% EGI-Service Discovery  & I\\
% gLite/GANGA  & I \\
% RENKEI  & I\\
% \hline
% \end{tabular}
% \caption{An analysis of the SAGA Use Cases and the Deployment Modes
%   that they would benefit from \jhanote{I don't think all of these fit
%     here, as some of these are *not* cloud applications, or grid-cloud
%     applications}}
% \end{figure}

% \section{SAGA Future/Roadmap}
% 
%  The evolution of SAGA has two very independent components: the
%  evolution of the SAGA specification as OGF standard, and the
%  evolution of the various SAGA implementations.
% 
%  The SAGA Core API specification is very close to become a full OGF
%  recommendation (it arrived in literally the last stage of the
%  respective OGF process).  As the specification is very modular, it
%  allows for additional functional packages to be specified
%  individually.  Several such extension packages have already been
%  specified (advert, messages, service discovery, resource management),
%  and are in various stages of the standardization process - we expect
%  those packages to mature and eventually become published
%  specifications within a year (some of them are already published,
%  others are very close to that).  We don't expect any of the published
%  specifications to evolve anytime soon -- so far there seems to be no
%  need of new versions of the API, despite the increasing set of
%  implementations, users, and use cases.
% 
%  On the implementation side of things we expect a limited amount of
%  evolution on the actual API level (the standard is stable after all).
%  Most of the current development efforts are spent on adaptor level, and
%  in fact the quality and usability of SAGA stands and falls with the
%  quality of the middleware bindings, i.e. of the adaptors.  We thus
%  expect that those will continue to demand the majority of our
%  resources.  Ideally, adaptor development, and even more adaptor
%  maintainance and support, will eventually be provided by the respective
%  middleware providers, but for the time being that is not the case.  At
%  the moment it is very hard to estimate timeframe and required effort
%  for an eventual support for the future EMI and/or PGI services -- that
%  depends on many factors, such as the structure of the upcoming
%  specifications (close to BES or not, close to JSDL or not, etc), on the
%  implementation progress for these services, and on their acceptance in
%  the wider community.
% 
%  SAGA-C++ has seen significant progress on documentation and end user
%  support (deployment support, ticket management, mailing list activity
%  etc).  Those improvements are mostly caused by the increasing SAGA
%  user community, which both requires, but also supports that progress.
% 
%  Additional domains that the SAGA project will see activity moving it
%  from a research project to production-grade infrastructure, is in the
%  area of data-intensive computing and cloud-based infrastructure. In
%  the near future we will have a {\it package} for data management
%  (beyond files) and have bindings to OCCI -- which would extend the
%  functionality and capabilties provided to OCCI implmentations.

\section{Open Cloud Computing Interface}
\label{sec:occi}

\jhanote{Can we have a paragraph comparing OCCI with OVF (DMTF) and
  CDMI. How would SAGA related to OVF and SNIA-CDMI?}

\tmnote{We can do that - this is good point - do we wanna bring SAGA into that picture too?}

The Open Cloud Computing Interface comprises a set of open
community-lead specifications delivered through the Open Grid
Forum. OCCI is a RESTful \tmnote{REF} Protocol and API for all kinds
of Management tasks based on a Resource Orientated Architecture (ROA)
\tmnote{REF}. OCCI was originally initiated to create a remote
management API for IaaS model based Services, allowing for the
development of interoperable tools for common tasks including
deployment, autonomic scaling and monitoring. It has since evolved
into a flexible API with a strong focus on integration, portability,
interoperability and innovation while still offering a high degree of
extensibility. The current release of the Open Cloud Computing
Interface is suitable to serve many other models in addition to IaaS,
including e.g. PaaS and SaaS. \tmnote{REF occi-wg.org here}

The documents are divided into three categories consisting of the OCCI
Core, the OCCI Renderings and the OCCI Extensions.
 
\begin{itemize}
  \item The OCCI Core specification consist of a single document
    defining the OCCI Core Model. The OCCI Core Model can be
    interacted with {\em renderings} (including associated behaviours)
    and expanded through {\em extensions}.
  \item The OCCI Rendering specifications consist of multiple
    documents each describing a particular rendering of the OCCI Core
    Model. Multiple renderings can interact with the same instance of
    the OCCI Core Model and will automatically support any additions
    to the model which follow the extension rules defined in OCCI
    Core.
  \item The OCCI Extension specifications consist of multiple
    documents each describing a particular extension of the OCCI Core
    Model. The extension documents describe additions to the OCCI Core
    Model defined within the OCCI specification suite.
\end{itemize}

The current specification consist of three documents.  Future releases
of OCCI may include additional rendering and extension
specifications. The documents of the current OCCI specification suite
are:

\begin{description}
  \item[OCCI Core] describes the formal definition of the the OCCI
    Core Model \cite{occi:core}.
  \item[OCCI HTTP Rendering] defines how to interact with the OCCI
    Core Model using the RESTful OCCI API
    \cite{occi:http_rendering}.
  \item[OCCI Infrastructure] contains the definition of the OCCI
    Infrastructure extension for the IaaS domain
    \cite{occi:infrastructure}.
\end{description}

Since OCCI is an RESTful Protocol or API is can enhances from all the
features the HTTP defines. This includes mechanisms for authentication
and authorization, Cache-mechnisms, Content-Type definitions and
dicovery capabilities. 

\tmnote{Following paragraphs need rewording/formatting and addionatl stuff}

Another guiding principle in OCCI is to make use of existing standards
and specifications where appropriate.

OCCI and the Storage Networking Industry Association's
(SNIA)\footnote{\url{http://www.snia.org/}.} Cloud Data Management
Interface (CDMI) working groups have collaborated together so that
both specifications are interoperable with each other. It states that

\begin{quote}
"The SNIA Cloud Data Management Interface (CDMI) is the functional
  interface that applications will use to create, retrieve, update and
  delete data elements from the cloud. As part of this interface the
  client will be able to discover the capabilities of the cloud
  storage offering and use this interface to manage containers and the
  data that is placed in them. In addition, meta-data can be set on
  containers and their contained data elements through this interface"
  \cite{SSH+2010}.
\end{quote}

OCCI and the Distributed Management Task Force's
(DMTF)\footnote{\url{http://www.dmtf.org}.} Open Virtualization Format
(OVF), see \cite{CDG+2009}, can be easily integrated through the use
of the resource type Link. Where a provider wishes to supply an OVF
representation of a client's resource instance(s), they can do so by
associating the instance(s) with a mirror representation, only the
serialisation format is OVF.

Other than this the OCCI working group is closely working together
with other groups inside of the Open Grid Forum. The Distributed
Computing Infrastucture Federation (DCI-fed) working group focuses on
the creation of models and APIs for setting up distributed federated
computing environments. Other than this the OCCI working group uses
Standards like those developed by the Distributed Resource Management
Application API (DRMAA) working group for common Job operations on
Clusters via the OCCI protocol.

\section{Grid-Cloud interoperatbilty}
\label{sec:gcinterop}

The next paragraphs describe how a Cloud and Grid interoperability
could look like. To demo these interoperability use-cases OCCI and
SAGA interoperability use cases are described.

\subsection{Using OCCI for IaaS based resource provisioning}
Since SAGA features a rich set of adapters to talk to Grid Middlewares
the first use-case describes a way to autmatically provision such grid
middlewares on demand. SAGA could use OCCI to request initial or
addtional Grid Resources.

This would be a request on a IaaS based level where SAGA could request
Compute Resource with a Grid Middlware installed to be started. While
Using formats like OVF or by providing addiotional Attributes in the
request towards OCCI, SAGA could customize the request and for example
state that it needs a Virtual Machine with certain attributes,
Platform LSF installed with version >= 7.0.5 and a OGSA-BES interface
deployed. The SAGA adapters can then be used to provision Jobs using
OGSA-BES.

This first use case demoes a dynamic scalling or dynamic provisioning
scenario. But it can also be used for Cloud-Bursting/Hybrid-Cloud
approaches.

\subsection{Using SAGA and OCCI to create a PaaS Offering}
SAGA offers a rich set of calls in it's API. Since OCCI ist mostly a
Protocol which can be used to create an API (by using Extensions) SAGA
can be used to enrich the OCCI functionalities and create a PaaS
offering.

Programmers would develop their applications using the SAGA API. When
finished they would provision their applications in an OCCI defined
manner. All the resource provisioning as well as concrete
implementations are then transparently hidden from the end-user.

This could be an HPC in the Cloud offering.

\subsection{SAGA access PaaS OCCI based offering}
Finally since SAGA's adapters can submit jobs there is a third way to
demo interoperability. A SAGA adapter could be implemented which
submits Jobs to Cluster using the OCCI Protocol. It might be that a
Service Provider offers such a Job Submission service over OCCI which
can be easily intergated with OCCI. The SAGA-OCCI Adapter could make
extensive usage of the automatic discovery and billig mechanisms while
using OCCI.

\section{Lessons learned}
\label{sec:lessons}

\ldots

\section{Architecture}
\label{sec:arch}

\subsection{3-Level Architecture Diagram}

\bibliographystyle{plain}
\bibliography{inter-cloud-grid-2011}

\end{document}

