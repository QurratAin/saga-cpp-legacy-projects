\documentclass[10pt,conference,final,letterpaper,twoside,twocolumn,]{IEEEtran}

\usepackage{graphicx}
\usepackage{color}
\usepackage{url}
\usepackage{ifpdf}
\usepackage{hyperref}
\usepackage{xspace}

\setlength\parskip{-0.015em}
\setlength\parsep{-0.15em}

\newenvironment{shortlist}{
	\vspace*{-0.85em}
  \begin{itemize}
 \setlength{\itemsep}{-0.3em}
}{
  \end{itemize}
	\vspace*{-0.6em}
}

\usepackage{fancyhdr}
\setlength{\headheight}{16.0pt}
\pagestyle{fancy}
\headheight = 0pt
\headsep    = 25pt
\fancyhf{}
\fancyhead[OC]{\bf {\it \footnotesize{Jha et al: A Case for SAGA as an Access Layer for DCI}}}

\newif\ifdraft
% \drafttrue
\ifdraft
 \newcommand{\amnote}[1]{  {\textcolor{magenta} {***AM: #1}}}
 \newcommand{\jhanote}[1]{ {\textcolor{red}     {***SJ: #1}}}
 \newcommand{\tmnote}[1]{  {\textcolor{blue}    {***TM: #1}}}
\else
 \newcommand{\amnote}[1]{}
 \newcommand{\jhanote}[1]{}
 \newcommand{\tmnote}[1]{}
\fi

\newcommand{\dn}{\vspace*{0.33em}}
\newcommand{\dnn}{\vspace*{0.66em}}
\newcommand{\dnnn}{\vspace*{1em}}
\newcommand{\uppp}{\vspace*{-1em}}
\newcommand{\upp}{\vspace*{-0.66em}}
\newcommand{\up}{\vspace*{-0.33em}}
\newcommand{\shift}{\hspace*{1.00em}}

\newcommand{\T}[1]{\texttt{#1}}
\newcommand{\I}[1]{\textit{#1}}
\newcommand{\B}[1]{\textbf{#1}}
\newcommand{\BI}[1]{\B{\I{#1}}}
\newcommand{\F}[1]{\B{[FIXME: #1]}}
\newcommand{\TODO}[1]{\textcolor{red}{\B{TODO: #1}}}

\begin{document}

\title{Position Paper: A Case for SAGA as Access Layer for OCCI backed Cloud Infrastructures}

\author{Shantenu Jha$^{*1,2}$, Thijs Metsch$^{3}$, Andre Merzky$^{1}$\\
  \small{\emph{$^{1}$Center for Computation \& Technology, Louisiana State University, USA}}\\
  \small{\emph{$^{2}$Department of Computer Science, Louisiana State University, USA}}\\
  \small{\emph{$^{3}$Platform   Computing, Germany}}\\
  \small{\emph{$^{*}$Contact Author \texttt{sjha@cct.lsu.edu}}}
  }


\maketitle

\section*{Aim and Audience of this Position Paper}



\section{Introduction}

SAGA is an acronym for "Simple API for Grid Applications". As the name
suggests, a simple API which facilitates the development and execution
of distributed applications on most types of distributed
infrastructure.  Modern distributed computing environments are very
complex infrastructures, and allowing applications to make use of
these complex systems is not trivial.  By defining a simple API, one
requires those complexities to be dealt with at levels other than
application code and development.  Simplicity of the interface is the
primary design principle and objective of SAGA.  Functional goals of
SAGA are:


\begin{enumerate}

\item Provide a stable programming interface to distributed
   application programmers and tool developers
 
\item Shield developer from heterogeneous and evolving
   infrastructures and middlewares

\item By providing the building blocks to distributed and remote
   operations enable the expression of high-level abstractions
   and support of distributed application requirements

\end{enumerate}

The fact that SAGA is an OGF standard ensures the community-wide
adoption and stability of the
API.


\section{The SAGA Landscape}

 \subsection{SAGA API Specification}

  The SAGA API specification is object oriented, and language
  independent (the API is defined in IDL).  The API is structured into
  various packages (e.g. jobs, replicas, streams, etc.).  Those
  packages have limited dependencies amongst each other - not all SAGA
  implementations implement all packages.  All API packages share
  certain properties: how are synchronous methods expressed, how are
  notifications realized, how are security tokens expressed, what
  types of exceptions are defined, etc.  Those properties are
  specified in the SAGA-Core, the API's look and feel.

  That design of the SAGA API allows to specify additional API
  packages, which adhere to the same look-and-feel.  In fact, several
  such API packages have already been defined (e.g., Service
  Discovery, Remote Procedure Calls etc.), and are standardized as
  well, or are in the process of being standardized.

 \subsection{SAGA: A Community Specification}

  The SAGA API specification has been developed and guided by the
  broader distributed computing community at the
  OGF\footnote{\url{http://www.ogf.org/}}.  An analysis of the
  requirements led to abstractions that were mapped into different
  SAGA API packages, while ensuring that (a) the overall usability
  (e.g. the API look-and-feel) was consistent over the whole scope of
  the API, (b) the API functionality maps relatively well onto
  existing middleware features, and (c) the API is simple to use.  The
  API is simple, even if the semantic translation and across layers
  and maintaining implementation fidelity for middleware specific
  features is not trivial.

  The SAGA standardization effort is closely syncronized with other
  specification and community efforts, within and outside of OGF.  In
  particular, OGF groups ensure that SAGA semantics map well to lover
  level specifications, such as JSDL, BES, GridFTP, etc.   But also, 
  and possibly more importantly, it is now widely and independenly
  acknowledged that a uniform, simple and stable access layer is
  neccessary (but not sufficient) to improve end user experience on
  distributed computing infrastructures, and that SAGA can indeed play
  that role for a specific set of use cases.  As such, SAGA is now
  integral part of the GIN (Grid Interoperation Now) community effort,
  and also plays an active role in current efforts like OGF's PGI
  group and US's XD proposals.
  


\subsection{Standards promote Interoperability}

 \subsubsection*{ExTENCI}

 ExTENCI\footnote{\url{https://sites.google.com/site/extenci/}} is an
 NSF funded 2 year project to promote interoperability between the
 TeraGrid (and/or its successor -- XD) with the OSG (and its
 successor).  The role of SAGA is provide a common job-submission
 mechanism across TeraGrid and OSG, for both command-line access as
 well as for Cactus based application via Gateway.  The different
 application scenarios that are/will be supported are:
 
 \begin{itemize}
 \item Ensemble of Cactus Simulations (NumRel, EnKF (Petroleum Eng))
 \item Multiphysics Code (GR-MHD, CFD-MD)
 \item Spawning Simulations (Realtime ‘outsourcing’ from
   BlueWaters/Ranger to specialised architectures or less powerful
   resources)
 \end{itemize}



\section{Analysis of Use Cases}

Distributed Computing is more than just submitting isolated jobs.  It
is also about federating resources dynamically; about coordinated
execution of heteregeneous and dynamic workloads; it is about
distributed data management etc. So while SAGA is used for multiple
reasons, there are three primary usage modes of SAGA are the
following: (i) Simplifying access layer, (ii) building block for tools
and distributed execution execution, and (iii) a distributed scripting
and programming capability.


\subsection{SAGA as a Standardized Programmatic and Access Layer:
  Advantage to DCI Providers}

\subsubsection*{Providing Distributed Scipting and Programming Capbility}
% The SAGA API provides very concise and high level method calls which
% cover the vaste majority of distributed operations, as required by the
% target user community -- scientific application and tool developers.
The Structural Biology Grid (\url{http://SBGrid.org}) currently
implements sophisticated analysis and user-defined pipelines. However,
these are inherently localized and confined to specific
infrastructure.  Replacing ``local python'' calls with ``distributed
(SAGA) python'' calls would enable the seamless utilization of
DCI. This provides a simple mode of extensibililty of infrastructure,
without any major refactoring of code. Further, as the API
specification and implementation is \I{standardized}, and thus stable,
it allows for a 'write once, run anywhere' approach, which is in
general not available otherwise (or at least not without
\I{significantly} increase of application complexity).  The advantages
of this to the end-user is obvious; the lowered barrier-to-entry for
novel users and communities will increase the ease and uptake of DCI
thus benefitting DCI providers/organizations.  As part of the ExTENCI
project, SAGA will make major advances towards becoming a broadly
usable programmatic access layer to Condor/OSG.

\subsection{Application Development Advantages}

\subsubsection*{Application Prototyping and Tooling}

The SAGA Python bindings have been proven to be immensely helpful for
application prototyping.  But also, they are very helpful when
interactively testing remote operations (in the interactive Python
interpreter / Python shell).  Finally, it is very easy to implement
small command line tools in Python, which are able to mimic and test
smaller portions of the overall application.  For example, it is
straight forward to implement a specific job control component of an
application in a stand alone Python script, and to later include the
same functionality in the application proper, with the confidence that
the semantics of the remote operations will be well
preserved. %(remember: the SAGA specification is very strict about
   %the definition of semantics).

\subsubsection*{Application Development}

%    According to the SAGA use case analysis\cite{saga-uc}, and to our
%    own experience in developing distributed applications, the set of
%    funtionality required for such applications is rather limited.
%    However, as that functionality is provided in very different ways,
%    by the various distributed middlewares available today, application
%    complexity has historically increased significantly for any single
%    remote operation used by the application.


% \subsubsection*{Application Deployment}

% \subsubsection*{Runtime Configuration}


\subsection{SAGA Usage and Deployment Modes}

Although SAGA is foremost an API, the SAGA distributions support end
users in a variety of ways.  In particular, the SAGA distributions
also include command line tools implemented via the SAGA API, and
higher level libraries for common distributed programming patterns,
also basing on the SAGA API.


The following three most common deployment modes were discussed:

\begin{itemize}

\item "Mode 1": SAGA is deployed on the end user's
  computer. Non-intrusive to the Grid Middleware this is by far the
  most popular deployment mode for SAGA today. At the same time,
  however, it is completely beyond control, influence or support for
  EGI.eu.

\item "Mode 2": SAGA is installed independently on VO infrastructure
  and made available through the VO portal functionality. Likewise,
  this mode of access is non-intrusive, but through collaboration and
  strong binding through MoUs with EGI.eu somewhat more apt to
  influence and coordination.

\item "Mode 3": SAGA is installed directly on the Grid infrastructure,
  along with or part of, the Middleware. This installation mode would,
  though highly intrusive, offer most potential to Grid end users as
  it allows to write truly distributed (domain specific) applications
  with relative ease.

\end{itemize}

\begin{figure}[t]
\centering
\begin{tabular}{ll}
Use Case & Deployment Mode \\
\hline
Applications seeking Interoperability & I, II III \\
Virtual Physiological Human & I, III\\
FEDEX (Multi-Scale, Multi-Physics Simulations)  & I, III\\
NeuGrid  & I, II, III\\
BigJob based Applications (SAGA Pilot-Job)  &  III\\
Science Gateways (eg Computational Biology)  & II, III \\
EGI-Service Discovery  & I\\
gLite/GANGA  & I \\
RENKEI  & I\\
\hline
\end{tabular}
\caption{An analysis of the SAGA Use Cases and the Deployment Modes
  that they would benefit from}
\end{figure}

  

\section{SAGA Future/Roadmap}

 The evolution of SAGA has two very independent components: the
 evolution of the SAGA specification as OGF standard, and the
 evolution of the various SAGA implementations.

 The SAGA Core API specification is very close to become a full OGF
 recommendation (it arrived in literally the last stage of the
 respective OGF process).  As the specification is very modular, it
 allows for additional functional packages to be specified
 individually.  Several such extension packages have already been
 specified (advert, messages, service discovery, resource management),
 and are in various stages of the standardization process - we expect
 those packages to mature and eventually become published
 specifications within a year (some of them are already published,
 others are very close to that).  We don't expect any of the published
 specifications to evolve anytime soon -- so far there seems to be no
 need of new versions of the API, despite the increasing set of
 implementations, users, and use cases.

 On the implementation side of things we expect a limited amount of
 evolution on the actual API level (the standard is stable after all).
 Most of the current development efforts are spent on adaptor level, and
 in fact the quality and usability of SAGA stands and falls with the
 quality of the middleware bindings, i.e. of the adaptors.  We thus
 expect that those will continue to demand the majority of our
 resources.  Ideally, adaptor development, and even more adaptor
 maintainance and support, will eventually be provided by the respective
 middleware providers, but for the time being that is not the case.  At
 the moment it is very hard to estimate timeframe and required effort
 for an eventual support for the future EMI and/or PGI services -- that
 depends on many factors, such as the structure of the upcoming
 specifications (close to BES or not, close to JSDL or not, etc), on the
 implementation progress for these services, and on their acceptance in
 the wider community.

 SAGA-C++ has seen significant progress on documentation and end user
 support (deployment support, ticket management, mailing list activity
 etc).  Those improvements are mostly caused by the increasing SAGA
 user community, which both requires, but also supports that progress.

 Additional domains that the SAGA project will see activity moving it
 from a research project to production-grade infrastructure, is in the
 area of data-intensive computing and cloud-based infrastructure. In
 the near future we will have a {\it package} for data management
 (beyond files) and have bindings to OCCI -- which would extend the
 functionality and capabilties provided to OCCI implmentations.

% \bibliographystyle{plain}
% \bibliography{inter-cloud-grid-2011}

\end{document}

