\documentclass[a4paper,10pt]{article}
\usepackage[utf8]{inputenc}
\usepackage{graphicx}
\usepackage{url}
\usepackage{float}
\usepackage{times}
\usepackage{multirow}
\usepackage{listings}
\usepackage{times}
\usepackage{paralist}
\usepackage{epsfig}
\usepackage{subfigure}
\usepackage[hypertex]{hyperref}
\usepackage{subfigure}
\usepackage{color}
\usepackage{xspace}

%\documentclass{rspublic}

\usepackage{ifpdf}

\newcommand{\I}[1]{\textit{#1}}
\newcommand{\B}[1]{\textbf{#1}}
\newcommand{\BI}[1]{\textbf{\textit{#1}}}
\newcommand{\T}[1]{\texttt{#1}}

\newcommand{\sagaspec}{\textit{SAGA}\xspace}
\newcommand{\sagaimpl}{\textit{SAGA}\xspace}

\newcommand{\spec}{\sagaspec}
\newcommand{\impl}{\sagaimpl}

\setlength\topmargin{0in}
\setlength\headheight{0in}
\setlength\headsep{0in}
\setlength\textheight{9.5in}
\setlength\textwidth{6.5in}
\setlength\oddsidemargin{0in}
\setlength\evensidemargin{0in}
\setlength\parindent{0.1in}
\setlength\parskip{0.25em}


\ifpdf
 \DeclareGraphicsExtensions{.pdf, .jpg}
\else
 \DeclareGraphicsExtensions{.eps, .ps}
\fi

\newcommand{\note}[1]{ {\textcolor{red} { ***NOTE: #1 }}}

\newif\ifdraft
\drafttrue

\ifdraft
\newcommand{\amnote}[1]{   {\textcolor{magenta} { ***Andre:    #1 }}}
\newcommand{\jhanote}[1]{  {\textcolor{red}     { ***Shantenu: #1 }}}
\newcommand{\onote}[1]{  {\textcolor{blue}     { ***Ole: #1 }}}

\else
\newcommand{\amnote}[1]{}
\newcommand{\jhanote}[1]{}
\fi

\begin{document}

 \title{ \large \vspace{-3.5em} Towards a Framework for Autonomic
   Computational Sciences on Production Distributed Infrastructure}

 \author{\normalsize Shantenu Jha$^{1,2,3}$, Yaakoub el-Khamra, Andre Luckow$^{1}$, Ole Weidner$^{1}$, \\
   \small{\emph{$^{1}$Center for Computation \& Technology, Louisiana State University, USA}}\\
   \small{\emph{$^{2}$Department of Computer Science, Louisiana State University, USA}}\\
   \small{\emph{$^{3}$e-Science Institute, University of Edinburgh,
       UK}} } \date{}
 \maketitle
 
\subsection*{Introduction}
\vspace{-0.6em}

Strategic investments coupled with technological advances are rapidly
realizing a pervasive cyberinfrastructure, both nationally and
globally, that integrates computers, networks, data archives,
instruments, observatories, experiments, and embedded sensors and
actuators. Such a computational ecosystem has the potential to
catalyze new thinking in virtually all areas of computational science
and engineering, which can lead to unprecedented insights into
natural, engineered and human systems. For example, application
formulations can holistically investigate any phenomena of interest by
combining computations, experiments, observations, and real-time
information, for example, to understand and manage natural and
engineered systems. These emerging computational paradigms and
practices enabled by this cyber-ecosystem are naturally distributed
and collaborative and fundamentally data intensive and data driven, as
they explore coupled multi-physics, multi-scale formulations,
end-to-end application workflows.

Unfortunately, the computational ecosystem also entails unprecedented
complexity. These includes increasing scale, heterogeneity and
dynamism of the system, emerging new and potentially disruptive
hardware trends such multi/many cores, on-chip heterogeneity, deep
non-uniform memory hierarchies, accelerators such as GPGPUs, clouds,
etc., ever-increasing data volumes, as well as emerging first-order
concerns such as fault-tolerance and energy efficiency. Furthermore,
the new application formulations also bring complex structures,
behaviors, and interactions.

In addition to the ability to be able to address the complexity of
computational infrastructure effectively and efficiently in the
traditional sense, there are emerging trends that will soon become
important concerns in computational science: these are the role of
dynamic data and that of energy-efficient
computing~\cite{exa-kogge09}. 

\onote {IMHO the issue of energy-aware computing \textbf{has} to 
be addressed on system level -- preferably purely in silicon.
This should not be a part of our discussion about autonomics.
Of course I'm open for discussion on this topic ;-)}

It is unrealistic to expect that the
ability to address these issues percolates upwards to the application
development layer, just as it is not possible to address it completely at
the application level, 

\onote {This sentence doesn't make sense. You need an interface at the
\textit{application development layer} (\textit{very vague term})
in order to address the problem on the application level. Both are 
\textbf{not} mutual exclusive. }

the heterogeneous middleware, varying software
\& sevices and other elements of the computationl environments that
contribute towards the making a complex cyberinfrastructure.  Clearly,
there is a need for exploring alternate paradigms that can help
scientists fundamentally manage these challenges.

The autonomic computing paradigm aims at developing systems and
application that can manage and optimize themselves using only
high-level guidance or interference from users. Autonomic computing
systems dynamically adapt to changes in accordance with business
policies and objectives, and take care of routine elements of
management similar to the unconscious self-regulation behavior of
biological systems. Central to the autonomic paradigm are three
fundamental separations: (1) a separation of computations from
coordination and interactions; (2) a separation of non-functional
aspects (e.g. resource requirements, performance) from functional
behaviors, and (3) a separation of policy and mechanism - policies in
the form of rules are used to orchestrate a repertoire of mechanisms
to achieve context-aware adaptive runtime computational behaviors and
coordination andto interaction relationships based on functional,
performance, and QoS requirements. We operate under the following
presumption: Autonomic Computing concepts provide a pragmatic and
effective approach to addressing complexity.

As outlined above, the scale, complexity, heterogeneity, and dynamism
of emerging computational infrastructures and applications, coupled
with the resulting uncertainty presents application programming and
runtime management complexities that are not addressed by current
paradigms and this gap between user/applications and system.  A key
goal of Ref.~\cite{SOM-CISE10} was to investigate how autonomic
computing can be applied to address these challenges.  The aim of this
work is to show how SAGA -- a distributed programming system can be
generalised and extended to support autonomic computational science
capabilities, either directly or via the construction of frameworks.

\section{A Formal Discuss of Autonomic Properties \& Capabilities}

\section{Introduction to SAGA}

\jhanote{This should be a cut and paste job from THE saga paper}

\subsection{Application Programming Interface}

\subsection{Runtime System}

\subsection{Middleware Adaptors}

\subsection{Performance Aspects: The Cost of Flexibility, Late Binding
  and Dynamic Execution}

\section{SAGA and Autonomic Capabilities}


\section{SAGA-based Autonomic Frameworks}

\jhanote{We should talk about FutureGrid and what capabilities and
  autonomic features can be derived/implemented there. We should also
  get information on the XD TAIS award that has gone to NCSA and
  Buffalo. This will have ``services'' which can be cobbled upto to
  support ``autonomic capabilities''.}

\subsection{SAGA-based Pilot-Job aka BigJob}

Supports dynamic execution -- decouples workload from job submission.

\subsection{Faust}

BigJob in a general purpose way. (ie BigJob++)

\subsection{Lazarus}

A Fault-tolerant Mephisto (Mephisto++)

\subsection{Other Frameworks}


 \bibliographystyle{IEEEtran} 
 \bibliography{saga_autonomics}


\end{document}

