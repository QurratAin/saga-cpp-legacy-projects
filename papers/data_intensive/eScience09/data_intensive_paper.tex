\documentclass[a4paper,11pt]{article}
\pagestyle{empty}
\usepackage{amssymb}
\usepackage{amsmath}
\usepackage{graphicx}
\usepackage{epstopdf}
\usepackage{color}
\usepackage{hyperref}

\pdfpagewidth 9.75in
\pdfpageheight 12.0in 

\setlength\topmargin{-0.5in}
\setlength\headheight{-0.0in}
\setlength\headsep{0.2in}
\setlength\textheight{10.5in}
\setlength\textwidth{6.75in}
\setlength\oddsidemargin{0in}
\setlength\evensidemargin{0in}
\setlength\parindent{0.1in}
\setlength\parskip{0.25em}

\usepackage{fancyhdr,lastpage}
\pagestyle{fancy}
\chead{Understanding Performance of Distributed File Systems~\thepage~of \pageref{LastPage}}
\fancyhead[L]{}
\fancyhead[R]{}
\cfoot{Page \thepage~of \pageref{LastPage}}

\newif\ifdraft
%\drafttrue
\ifdraft
\newcommand{\fixme}[1]{ { \bf{ ***FIXME: #1 }} }
\newcommand{\jhanote}[1]{ {\textcolor{red} { ***Jha: #1 }}}
\else
\newcommand{\jhanote}[1]{}
\newcommand{\fixme}[1]{}
\fi

\begin{document}

\begin{center}

\textbf{\large Understanding Performance Implications of Distributed Filesystems in a Data-Intensive Application}

\vspace{0.05in}

\textbf{Christopher Miceli$^{1}$, Michael Miceli$^{1}$, Bety Rodriguez-Milla$^{1}$, Shantenu Jha$^{1,2,*}$}

\small{\emph{$^{1}$Center for Computation \& Technology, Louisiana State University, USA}}

\small{\emph{$^{2}$Department of Computer Science, Louisiana State University, USA}}

{\footnotesize {\hspace{0.0 in} $^*$Corresponding Author sjha@cct.lsu.edu}}

\end{center}

\section{Abstract}

Grids and, more recently, Clouds and Cloud-like infrastructure are capable of supporting large problems. While the capability of these systems is great, unique performance issues appear as data-sets get extremely large, such as Google's 20 petabytes of data processed per day~\cite{google}, and trends show continuing growth. Against this backdrop, it has become ever more important that a distributed application developer take precautions when placing, scheduling and managing such large volumes of data, or to state the obvious, performance could be adversely affected greatly. As the volume of data, and especially distributed data increases, scalable placement and management techniques are required.  Distributed File-Systems (DFS) simplify the management of distributed data, for example, providing a single access protocol and a common name-space. With the advent of several stable Open Source DFS projects (motivated in part, by developments in Cloud Computing), these can be deployed without explicit vendor support and are now thus, potentially useful and effective tools to consider for data-intensive scientific applications.  DFS typically handle replication and distribution of files across multiple machines internally and although this abstraction simplifies the management of data, contrasted with explicit distributed aware placement, there are potential performance trade-offs, in that the user can no longer control data placement to optimise performance, and possibly has performance implications.  The most common parameters determining the performance of using DFS are the (i) number of replicas of each data/file, (ii) number of servers. The aim of this paper is to understand the performance trade-offs of a DFS compared to ``regular'' distribution and placement techniques, and how sensitive the performance is in the context of a real data-intensive distributed applications.

\section{Introduction}
When working with distributed systems and large data-sets together, determining whether to move input data to the computational resource, or the computational workload to the input data becomes very important. Frequently there is more than one copy of the input data for fault-tolerance reasons, then the added issue of deciding between the two or more replicas becomes relevant. While DFS remove the responsibility of replica management, the abstraction often makes determining where in the DFS the data is being stored difficult, and thus relying on the DFS protocol and internal algorithms to perform well. Despite this, the DFS replication may alleviate these issues by placing replicas in locations where computational resources reside.

%There are at least two types of data-intensive applications: the first where the actual data generated is large; the second type is where the data generated is small, but the volume of data on which computation occurs is very large. The application we used, has relatively small input and relatively small output, but the manner of processing causes many data reads. This type of application can be classified as having a large data throughput. \jhanote{Can you elaborate on different types of data-intensive applications? What kind is
%an ImageMagic based application?}

In this paper, we use an application based upon a Grid-enabled All-Pair abstraction~\cite{Interop, AllPairs}. This application applies an operation on the input data-set such that every possible pair in the set is input to the operation. The operation we chose is to compare images using ImageMagick and the result is a numerical value between 0 and 1~\cite{imagemagick}. It has relatively small input [O(MB)] and relatively small output [O(KB)]. However, the manner of processing causes many data reads. This type of application can be classified as having a large data throughput. To handle seamlessly the DFS and gridftp based data stores, the application uses the Simple API for Grid Applicatons (SAGA) ~\cite{saga_web}. This allows the same exact application to be used for all of our experiments. The result of this application is stored in a matrix. The application spawns distributed jobs to run sets of these pairs. The problem becomes determining which pairs to put into a set, and with which distributed resource to run that set. If transferring data to the job takes too long, we spend more time on data management than computation. There may be a resource capable of the work that may be slower than others, but network-close (able to be accessed in a relatively quick manner via the network supporting the distributed system) enough to the data to make up for its lack of computational ability. In our experiments, we used CloudStore (formerly KFS), an open-source high performance distributed filesystem that builds upon ideas from Google's distributed filesystem GFS~\cite{kfs_web}. CloudStore was chosen for its high performance focus, C++ implementation, and its source code availability.

We use the results of three different experiments in order to assess the effect of distributed filesystems. In the first experiment, we run the SAGA-based All-Pairs application on up to 4 machines on a Grid (LONI), without any specific data placement strategy; also, no replication or fault-tolerance takes place. The application sequentially assigns data to the first available computational resource, and all data is accessed via the gridftp protocol. The second experiment is similar to the first, except the All-Pairs application takes the data's location into consideration when determining whether or not to assign a certain data-set to an idle job. Inspired by earlier work~\cite{netperf}, this version of the application performs an extra step that determines the performance of the network by pinging hosts that may be involved, and utilises this information when deciding which data-set to assign to a job requesting work. Though not very sophisticated, it is a first-approximation to performance model aware data-placement strategy.  If there is an unprocessed data-set collocated or network-close with the job, then the assignment of that worker to that data-set would have benefits. If there is no unprocessed data-set network-close to the job, still we assign data that may be network-far, in case the network-close job failed or there is no available jobs network-close to the data-set. The third experiment provides information into DFSs performance in handling data locality issues. The same All-Pairs application as in Experiments 1 and 2 is used, except all data is stored in the distributed filesystem CloudStore with varying replication factors, starting from two (each file is guaranteed to be replicated at least twice). All read and writes also utilise the distributed filesystem. Since the performance of this test is dependent on where the DFS stores data relative to the computational resources of the system, we place block-servers on every machine that may be capable of performing work. This places all responsibility on the DFS in determining where to place data.

Our results show that a DFS greatly changes the performance of a distributed application in a positive manner. Our experiments that utilised the DFS to access and store data outperformed their gridftp counterparts by an order of magnitude in most cases. Our results also indicate that the DFS scales better as file sizes and number of files grow, although both seem to scale linearly. Before any conclusions may be drawn, there are issues that need to be addressed. Our application utilised SAGA to access DFS based and gridftp based files. Although there is clearly a SAGA induced overhead, this overhead is constant.  \jhanote{ We need to discuss performance issue: Ole has performance numbers that contradict this, i.e., the overhead that SAGA introduces for file/gridftp is very small compared to native globus calls. This may be due to SAGA's adaptive nature, where lots of computation is spent determining how to make a distributed call. Other reason's may be the gridftp SAGA adaptor being separated from the application, being able to make only general decisions, allowing no performance tweaks. } \jhanote{Accessing files through this abstraction with gridftp seemed to perform sub-optimally in comparison to using the globus tools directly.} In addition, we were unable to utilise our entire distributed system, using at most 8 jobs to handle our work. With a replication level of two in the DFS, data was almost certainly co-located with the computational resource. In the second experiment, utilizing information from first staging the network did improve upon the results of the naive first experiment, but still did not approach the DFS's performance levels.

Distributed filesystems are important abstractions for a data-intensive distributed application developers to consider. It also appears that staging is worth the time required to build a graph representing the network. Also to note, the second experiment is also naive in the way that it attempts to optimise data and work assignments. Our staging only performed pings, not data transfer trials or reliability tests. A job could have low latency, but poor bandwidth. Perhaps CloudStore's performance can be attributed to recent work that has shown that data in large scale distributed applications tends to be accessed together, despite being seemingly unrelated in the input data-set. Such correlation in data-access
has been observed elsewhere, and specific abstractions to support the
access of ``aggregation of such files'' has been referred to as a filecule, an application specific group of files~\cite{filecule}. Attempting to determine 
if analogous abstractions could enhance performance for the All-Pair application could be interesting.  In a DFS, however, if the data store is also capable of data processing, then the DFS is placing commonly used files together on machines needing them for work; in essence, the DFS is finding these groups for the developer. The fault tolerance, for which distributed filesystems are already well renowned for, also has added benefits to grid application developers in terms of performance. The distributed application does not have to be aware of where data has been copied to previously when assigning work; the distributed filesystem uses the best replica when data is being accessed.

\section{SAGA}
\section{AllPairs}
\section{KFS and GridFTP}
\section{Experiments}
\subsection{Experiment I}
\subsection{Experiment II}
\subsection{Experiment III}
\section{Conclusion}
\section{Acknowledgement}

\bibliographystyle{IEEEtran} 
\bibliography{data_intensive_abstract}
\end{document}
