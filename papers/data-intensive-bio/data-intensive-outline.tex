%\documentclass[3p,twocolumn]{article}
\documentclass[12pt]{article}

\usepackage{graphicx}
\usepackage{color}
\usepackage{url}
\usepackage{ifpdf}
\usepackage{hyperref}
\usepackage{xspace}
%\usepackage[draft]{pdfdraftcopy}

\setlength\parskip{-0.015em}
\setlength\parsep{-0.15em}

\newenvironment{shortlist}{
	\vspace*{-0.85em}
  \begin{itemize}
 \setlength{\itemsep}{-0.3em}
}{
  \end{itemize}
	\vspace*{-0.6em}
}

\usepackage{fullpage}
%\usepackage[top=tlength, bottom=blength, left=llength, right=rlength]{geometry} %http://en.wikibooks.org/wiki/LaTeX/Page_Layout
%\usepackage[margin=1in, paperwidth=5.5in, paperheight=8.5in]{geometry}

\usepackage{fancyhdr}
\setlength{\headheight}{16.0pt}
\pagestyle{fancy}
\headheight = 0pt
\headsep    = 25pt
\fancyhf{}
%\fancyhead[OC]{\bf {\it \footnotesize{Jha et al: A Case for SAGA as an Access Layer for DCI}}}

\newif\ifdraft
\drafttrue
\ifdraft
 \newcommand{\amnote}[1]{  {\textcolor{magenta} {***AM: #1}}}
 \newcommand{\jhanote}[1]{ {\textcolor{red}     {***SJ: #1}}}
 \newcommand{\olenote}[1]{ {\textcolor{blue}    {***OW: #1}}}
\else
 \newcommand{\amnote}[1]{}
 \newcommand{\jhanote}[1]{}
 \newcommand{\olenote}[1]{}
\fi

\newcommand{\dn}{\vspace*{0.33em}}
\newcommand{\dnn}{\vspace*{0.66em}}
\newcommand{\dnnn}{\vspace*{1em}}
\newcommand{\uppp}{\vspace*{-1em}}
\newcommand{\upp}{\vspace*{-0.66em}}
\newcommand{\up}{\vspace*{-0.33em}}
\newcommand{\shift}{\hspace*{1.00em}}

\newcommand{\T}[1]{\texttt{#1}}
\newcommand{\I}[1]{\textit{#1}}
\newcommand{\B}[1]{\textbf{#1}}
\newcommand{\BI}[1]{\B{\I{#1}}}
\newcommand{\F}[1]{\B{[FIXME: #1]}}
\newcommand{\TODO}[1]{\textcolor{red}{\B{TODO: #1}}}

\begin{document}

\title{Towards a Scalable Architecture for Deep Sequencing Analytics}

\author{Joohyun Kim$^{1}$, Sharath Maddineni$^{1}$, Shantenu Jha$^{*1,2}$, \\
  \small{\emph{$^{1}$Center for Computation \& Technology, Louisiana State University, USA}}\\
  \small{\emph{$^{2}$Department of Computer Science, Louisiana State University, USA}}\\
  \small{\emph{$^{*}$Contact Author \texttt{sjha@cct.lsu.edu}}}
  }


\maketitle

\section*{Abstract}

We investigate the use of distributed computing environments, a
production HPC grid and a cloud environment for the genome-wide
mapping with BFast.  The main goal of this work is to understand the
characteristics of these two distributed computing environments and
compare and contrast their strengths and suitability to support the
computational requirements of deep sequencing.
% regarding computational requirements of the target bioinformatics
% application whose 

We investigate two model genomes -- human genome and a microbe,
Burkerholderia Glumae, that might represent a eukaryote and a
prokaryote system.  The computational complexity of execution of bioinformatics calculation, mapping with BFast, 
depends upon the size of a reference genome, the data size of short
reads from high-throughput technologies of the Next Generation
Sequencing platforms, and their biological genome contexts such as distinctive differences between prokaryotes vs eukaryotes.

The two distributed environments, the Louisiana Optical Network
Initiative (LONI) grid and a Cloud system from the FutureGrid, were
used primarily focusing on different and unique challenges in HPC and
Cloud computing conditions.

The time to completion are analyzed by comparing advantages as well as
limitations of each distributed computing infrastructure in
conjunction with distinctively different two different genome systems.
With our results, we discuss the importance of an effective runtime
environment that facilitates a rapid development of
cyberinfrastructure of distributed computing resources and supports
optimized execution patterns for a target scientific application, in
particular, of the data-intensive genome-wide analysis.


\section{Introduction}

% \bibliographystyle{plain}
% \bibliography{egi-white-paper}

High-throughput sequencing techniques including deep sequencing
approaches such as ChIP-seq and RNA-seq have changed biological
sciences and biomedical research dramatically with their genome-wide
information.  Consequently, the need of computational methods for
resolving new challenges and requirements of processing and analyzing
genome sequencing data is increasingly taken into account
indispensable for successful outcomes but the development of
infrastructure and required software remain major challenges.  The
complexity of biological information as well as the size of relevant
genomics data as well as difficulty of the utilization of
heterogeneous distributed computing resources constitute such
challenges.

We primarily focus on the mapping process of short reads from NGS
platforms against a reference genome.

\jhanote{we want to present a strawman of an architecture based upon
  requirements and a reference implementation of the architecture} In
this work, we present our work on the infrastructure development for
the use of High Performance Computing (HPC) grids and Cloud
environment for genome-wide analysis.

The development of an efficient runtime environment, which is the key part of our development, requires an understanding of biological information that the target scientific application aims to produce as well as of execution patterns and scalability of the computing resources of interest to be deployed.  First, we note that in general, genome-wide mapping procedure is an good example of data-intensive scientific applications and can be efficiently carried out with parallel or concurrent executions if there exist ways of data fragmentation with intact biological information.  For example, a reference genome is likely to be composed of many chromosomes or plasmids, and thus mapping on each chromosomes and plasmids can be executed separately.  The challenge, however, is that the way of data fragmentation, and thus system configuration for required parallel/concurrent runs should be considered the characteristics of system environment. Those characteristics differ distinctively between a HPC grid and a Cloud and also vary with a specific target system in each class.  

\section{Applications: An Overview}
First, for genome-wide mapping procedure with a mapping application, BFast, we need to carry out two different stages in general, genome indexing with a reference genome, mapping of short reads sequencing data against a indexed reference genome.  The simple workflow for this is illustrated in Fig.~\ref{} and the target step for parallel/concurrent execution is indicated. 

In Table~\ref{}, we summarize the data sizes for two genome mapping.  


\jhanote{Joohyun: Please organize each description addressing each of
  the following points: (i) Brief outline of the scientific problem,
  (ii) What are the challenges, (iii) estimates of volumes of data
  involved, distributed or not?, number of tasks, are they coupled or
  uncoupled -- what is the level of coupling between tasks?}

\begin{table}
\begin{tabular}{|c|cc|} 
\hline 
Genome Species & Human  & Burkerholderia Glumae  \\ \hline
 Size of Reference Genome (in base pair (bp)) &  3 G & 7.3 M \\
 Type of Genome Analysis &  Exome  & Whole Genome Resequencing \\
 Size of Sequencing Data (in Byte) & 8.7G & 5.4 G \\
 Sequencing Platform & ABI SOLiD  &  Illumina GA2 \\
\hline
\end{tabular} \caption{Specification of target genomes and sequencing data from Next Generation Sequencing (NGS) platforms.}
 \label{table:two-genomes} 
\end{table}

 \begin{table}
 \begin{tabular}{|c|cc|} 
 \hline 
Distributed Environment &  HPC Grid &  Cloud \\ \hline
System  &  Louisiana Optical Network Initiative & FutureGrid \\
Name &  QB/Eric   &  INDIA/SIERRA \\
 \hline
 \end{tabular}
\caption{Specification of two distributed environments}
\label{table:two-systems} 
\end{table}
 
\section{Existing Solutions: Limitations and Challenges}

\jhanote{Here we need to define what solutions are currently employed, what works well
  what doesn't}

\section{Runtime Environment}

The DARE framework, comprising an open source Web application, Pylons
and the runtime environment for distributed scientific applications,
was employed for this work.  The framework enables us to develop a
lightweight, extensible, full-fledged science gateway effectively in
which the runtime environment built upon SAGA and BigJob abstraction
manages efficiently distributed computing-driven execution patterns of
target scientific applications.

\end{document}

