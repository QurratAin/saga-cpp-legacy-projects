\documentclass[a4paper,10pt]{article}
\usepackage[utf8]{inputenc}
\usepackage{graphicx}
\usepackage{url}
\usepackage{float}
\usepackage{times}
\usepackage{multirow}
\usepackage{listings}
\usepackage{times}
\usepackage{paralist}
\usepackage{epsfig}
\usepackage{subfigure}
\usepackage{hyperref}
\usepackage{subfigure}
\usepackage{color}
\usepackage{xspace}

%\documentclass{rspublic}

\usepackage{ifpdf}

\newcommand{\I}[1]{\textit{#1}}
\newcommand{\B}[1]{\textbf{#1}}
\newcommand{\BI}[1]{\textbf{\textit{#1}}}
\newcommand{\T}[1]{\texttt{#1}}
\newcommand{\F}[1]{\textbf{FIXME: #1}}

\newcommand{\sagaspec}{\textit{SAGA}\xspace}
\newcommand{\sagaimpl}{\textit{SAGA}\xspace}

\newcommand{\spec}{\sagaspec}
\newcommand{\impl}{\sagaimpl}
\newcommand{\lf}{Look-\&-Feel}

\setlength\topmargin{0in}
\setlength\headheight{0in}
\setlength\headsep{0in}
\setlength\textheight{9.5in}
\setlength\textwidth{6.5in}
\setlength\oddsidemargin{0in}
\setlength\evensidemargin{0in}
\setlength\parindent{0.1in}
\setlength\parskip{0.25em}


\ifpdf
 \DeclareGraphicsExtensions{.pdf, .jpg}
\else
 \DeclareGraphicsExtensions{.eps, .ps}
\fi

\newcommand{\note}[1]{ {\textcolor{red} { ***NOTE: #1 }}}

\newif\ifdraft
\drafttrue

\ifdraft
\newcommand{\amnote}[1]{   {\textcolor{magenta} { ***Andre:    #1 }}}
\newcommand{\jhanote}[1]{  {\textcolor{red}     { ***Shantenu: #1 }}}
\newcommand{\onote}[1]{  {\textcolor{blue}     { ***Ole: #1 }}}
\else
\newcommand{\amnote}[1]{}
\newcommand{\jhanote}[1]{}
\newcommand{\onote}[1]{}
\fi


\newcommand{\forhartmut}[1]{{\textcolor{magenta}{\textbf{For Hartmut:} #1 }}}
\newcommand{\forole}[1]{{\textcolor{green}{\textbf{For Ole:} #1 }}}
\newcommand{\foroleandhartmut}[1]{{\textcolor{blue}{\textbf{For Ole \& Hartmut:} #1 }}}

\newcommand{\thesagaimpl}{\textcolor{red}{SAGA-C++} }
\newcommand{\thesagaimplrt}{{Runtime Support System} }
\newcommand{\thesagaimpladap}{\textcolor{red}{Middleware Adaptors} }


\begin{document}

 \title{ \large \vspace{-3.5em} SAGA: The Next 5 Years}


 \author{\normalsize Andre Merzky$^{1}$, Andre Luckow$^{1}$, Ole Weidner$^{1,3}$, Shantenu Jha$^{1,4}$\\
   \small{\emph{$^{1}$Center for Computation \& Technology, Louisiana State University, USA}}\\
   \small{\emph{$^{2}$Department of Computer Science, Louisiana State University, USA}}\\
   \small{\emph{$^{3}$School of Informatics, University of Edinburgh,
       UK}}\\
   \small{\emph{$^{3}$ECE, Rutgers University}}\\
%   \small{\emph{$^{4}$e-Science Institute, University of Edinburgh, UK}}
 }
 \date{}
 \maketitle
 

%\newpage 

\subsection*{Abstract}
%\vspace{-0.6em}



This paper aims to provide a status of the SAGA project and charts out
its roadmap over the next 5 years.  This paper stakes out the
rationale for a SAGA based approach.  In the process we will establish
that of all the layers at which an abstraction-layer can be provided,
the ``SAGA-level'' is the most important one. This stems from a
trade-off of with simplicity extensibility on the one-hand and
performance and xxxx on the other.  We demonstrate how SAGA provides
the basis for ``innovation at the right levels''. We show why SAGA is
a necessary component for the completeness of the stack for production
CI (as defined by XD).

\newpage 
\section{Introduction}


\jhanote{What is the ``global vision'' of CI? Positioning and scoping
  of SAGA w.r.t related work and efforts}
\textbf{Emphasize on the landscape of distributed
  cyber-infrastructure.} \textbf{S.J, O.W.}

\subsection{Overview}




\subsection{The need for proper abstractions for DCI: the scope and
  role of SAGA}

\jhanote{Add brief note about the standardization process and why we
  chose to make this a standard}

\jhanote{SAGA moving forward.. (try to avoid short term projects) but
  focus on longer-term wish-list}

However, despite the need, there is a both a perceived and genuine
lack of distributed scientific computing applications that can
seamlessly utilize distributed infrastructures in an extensible and
scalable fashion.  The reasons for this exist at several levels.  We
believe that at the root of the problem is the fact that developing
large-scale distributed applications is fundamentally a difficult
process.  Commonly acceptable and widely used models and abstractions
remain elusive. Instead, many ad-hoc solutions are used by application
developers.  The range of proposed tools, programming systems and
environments is bewildering large, making integration, extensibility
and interoperability difficult.

Against this backdrop, the set of distributed infrastructure available
to scientists continues to evolve, both in terms of their scale and
capabilities as well as their complexity.  Support for, and
investments in, legacy applications need to be preserved, while at the
same time facilitating the development of novel and architecturally
different applications for new and evolving environments, such as
clouds. Whereas deployment and execution details should not complicate
development, they should not be disjoint from the development process
either, i.e., tools in support of deployment and execution of
applications should be cognizant of the approaches employed to develop
applications.

\jhanote{Other barriers: (i) the way DCI is provisioned and often
  federated -- most often not {\it a priori}. Thus middleware is
  critical and heterogeneity is inevitable, (ii) Insufficient
  abstractions are multiple levels (revisit this when talking about
  how SAGA provides abstractions at these multiple levels), (iii) }


\section{Using SAGA: Applications, Tools and Frameworks} \textbf{S.J.}


\section{Conclusion and Future Directions}

\section{Acknowledgements}
\bibliographystyle{IEEEtran} 
%\bibliography{the_saga_paper,saga_ogf}


\end{document}

