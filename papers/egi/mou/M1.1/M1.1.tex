\documentclass[10pt]{article}

\usepackage{graphicx}
\usepackage{color}
\usepackage{url}
\usepackage{ifpdf}
\usepackage{hyperref}
\usepackage{xspace}

\newcommand{\T}[1]{\texttt{#1}}
\newcommand{\I}[1]{\textit{#1}}
\newcommand{\B}[1]{\textbf{#1}}
\newcommand{\BI}[1]{\B{\I{#1}}}
\newcommand{\F}[1]{\B{[FIXME: #1]}}

\begin{document}

 \section*{M1.1: Input from SAGA to the regular revisions of the UMD Roadmap}

 \section*{Introduction}

  The Simple API for Grid Applications (SAGA) aims at providing a
  simple, stable and standardized client API to support the
  development of distributed applications and application frameworks.
  As such, SAGA is contributing to the \I{Client Capabilities}
  component of the UMD Roadmap.  The SAGA Project in particular
  intents to provide the following components as input to the next
  version of the UMD:

  \begin{enumerate}

   \item \BI{saga-core:} provides the actual
   SAGA client API libraries and headers in C++, along with API
   documentation and a wide variety of associated tools for supporting
   the application build process, for supporting the development of
   middleware adaptors, and to interact with distributed
   infrastructures via the command line.

   \item \BI{saga-bindings-python:} provides the python language
   bindings on top of the SAGA C++ API.

   \item \BI{saga-adaptor-globus} provides bindings of the SAGA API to
   globus based infrastructures.  The provided functionality covers
   job submission and management, file management, and replica
   management.

   \item \BI{saga-adaptors-bes} provides access to a wide variety of
   middlewares which expose the Basic Execution Service (BES)
   interface.  In particular, that adaptor can interface to ARC,
   Unicore, Genesis, SMOA, and others.

   \item \BI{saga-adaptors-ssh} provides ssh based job and file
   management capabilities to the SAGA API.  Although ssh based
   infrastructures are not easy to scale up, ssh still represents a
   very pervasive access technology.

   \item \BI{saga-glite} provides adaptors to the gLite service
   registry to the SAGA service discovery API, and also an adaptor for
   gLite job submission.

  \end{enumerate}

  We follow with great interest the recently increasing interest in
  adding virtualization technologies to EGI infrastructures.  Towards
  that goal, we plan to increase both API level and adaptor level
  support for cloud based infrastructures for the following release of
  the UMD.  In particular will we adapt the SAGA resource management
  API to cater to the EGI virtualization use cases, with the ability
  to map that API to, for example, OCCI based infrastructures.


  It must be noted that the SAGA release cycle, and in particular for
  releases planned for EGI, depend heavily on the integration with
  the underlying middleware infrastructures.  In the scope of this
  document, this is specifically globus (via IGE) and
  gLite/ARC/Unicore (via EMI).  The actual release dates for the
  respective software packages are as much defined by the IGE and EMI
  release and integration work cycles as by our own development cycles.


  Below we include an updated version of Section 11 of the original
  UMD Roadmap document, which reflects the current state of the UMD
  Client Capabilities from our perspective.

 
 \setcounter{section}{10}
 \section{Client Capabilities}

 \subsection{Client API Capability}

  Instead of addressing interface heterogeneity on the service level,
  an alternative approach proposes the abstraction of distributed
  services on the client side, providing a common interface to client
  application developers.  Adopting a client API may benefit domain
  specific application developers from evolving middleware while it
  may all the way easier to maintain a client side API for the most
  common Grid Use Cases that keeping track of and synchronizing
  middleware interfaces.


 \subsubsection{Supported Interfaces}

  OGF provides the SAGA API as an approach to a common, lightweight
  and simple API for client-side abstraction of distributed computing
  resource.  The SAGA API itself maps semantically very well onto
  interfaces which are standardized on a lower level of the middleware
  stack, such as GLUE, BES, DRMAA, GridFTP, and several others.  In
  general, SAGA can be implemented on any DCI which provides (a subset
  of) SAGA semantic capabilities.


 \subsubsection{Implementation Roadmap}

  This crosscutting capability needs close interaction and
  synchronization with all distributed service types it interfaces
  with.  Particularly, the implementation of middleware bindings (i.e.
  adaptors to SAGA) should be under control of the middleware service
  implementers, a model that has already been adopted for the
  development of Nagios plugins for the current EGI monitoring
  infrastructure.  The main SAGA implementation in C++ and Python is
  maintained at the Louisiana State University Center for Computation
  and Technology (LSU/CCT).  It currently support, amongst others,
  GT5, gLite, Condor and ssh.  In addition, the LSU implementation
  continues to support, to an increasing degree, the OGSA-BES and HPC
  Basic Profile standards in addition to JSDL.  This aligns with IGE's
  plans to support these standards in GT5.2 and GridWay in 2012.  

\end{document}

