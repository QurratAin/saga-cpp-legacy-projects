% $Header: /projects/VU-SAGA/Papers/saga_engine_2006/intro.tex,v 1.13 2006/10/11 02:55:39 gallen Exp $

  Relatively few existing grid-enabled applications exploit the full
  potential of grid environments. This is mainly caused by the
  difficulties faced by programmers trying to master the complexities
  of grids (see section~\ref{sec:requirements}).  Several projects
  concentrate on the development of high-level, application-oriented
  toolkits that free programmers from the burden of adjusting their
  software to different and changing grids. The Simple API for Grid
  Applications (SAGA)~\cite{saga_spec} is a prominent recent API
  standardization effort which intends to simplify the development of
  grid-enabled applications, even for scientists with no background in
  computer science, or grid computing.  SAGA was heavily influenced by
  the work undertaken in the GridLab project~\cite{gridlab_webpage},
  in particular by the Grid Application Toolkit (GAT)~\cite{gat}, and
  by the Globus Commodity Grid~\cite{cog}.  The concept of high level
  grid APIs has proved to be very useful in several projects
  developing cyberinfrastructures, such as the SURA Coastal Ocean
  Observing Program (SCOOP) which uses GAT to interface to large data
  archives~\cite{CS_Huang06a} using multiple access protocols.
  
  The C++ implementation of the SAGA API presented in this paper
  leverages the experience we obtained from developing the GAT and will
  provide a reference implementation for the OGF standardization
  process.  As the SAGA API is originally specified using the
  Scientific Interface Description Language (SIDL)~\cite{sidl}, the
  implementation also represents a first attempt to develop the SAGA
  C++ language bindings.  It has a number of key features, described
  later in detail:

  \begin{shortlist}
     
    \item Synchronous, asynchronous and task oriented versions of every
    operation are transparently provided.

    \item Dynamically loaded adaptors bind the API to the appropriate
    grid middleware environment, at runtime. Static pre-binding at link
    time is also supported.
          
    \item Adaptors are selected on a call-by-call basis (late binding,
    supported by a object state repository), which allows for
    incomplete adaptors and inherent fail safety.
    
    \item Latency hiding (e.g. asynchronous operations and bulk
    optimizations) is generically and transparently applied.

    \item A modular API architecture minimizes the runtime 
		memory footprint.
		
    \item API extensions are greatly simplified by a
    generic call routing mechanism, and by macros resembling 
    (SIDL)~\cite{sidl} used in the 
    SAGA specification. 
    
    \item Strict adherence to Standard-C++ and the utilization of 
    Boost~\cite{boost_website} allows for excellent portability 
    and platform independence.

  \end{shortlist}


