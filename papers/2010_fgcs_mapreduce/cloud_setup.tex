
 In order to fully utilize cloud infrastructures for SAGA
 applications, the VM instances need to fullfill a couple or
 prerequisites: the SAGA libraries and its dependencies need to be
 deployed, as need some external tools which are used by the SAGA
 adaptors at runtime, such as ssh, scp, and sshfs.  The latter needs
 the FUSE kernel module to function -- so if remote access to the
 cloud compute node's file system is wanted, the respective kernel
 module needs to be installed as well.

 There are two basic options to achieve the above:  either a
 customized VM image which includes the respecitve software is used;
 or the respective packages are installed after VM instantiation, on
 the fly.  Hybrid approaches are possible as well of course.

 We support the runtime configuration of VM instances by staging a
 preparation script to the VM after its creation, and executing it
 with root permissions.  In particular for apt-get linux distribution,
 the post-instantiation software deployment is actually fairly
 painless, but naturally adds a significant amount of time to the
 overall VM startup\footnote{The long VM startup times encourage the
 use of SAGA's asynchronous operations.}.

 For the presented experiments, we prepared custom VM images with all
 prerequisites pre-installed.  We utilize the preparation script
 solely for some fine tuning of parameters: for example, we are able
 to deploy custom saga.ini files, or ensure the finalization of
 service startups before application deployment\footnote{For example,
 when starting SAGA applications are started befor the VM's random
 generator is initialized, our current uuid generator fails to
 function properly -- the preperation script checks for the
 availability of proper uuids, and delays the application deployment
 as needed.}.

 % as needed:
 Eucalyptus and Nimbus VM images \amnote{please confirm for Nimbus}
 are basically customized Xen hypervisor images, as are amazons VM
 images.  Customized means in this context that the images are
 accompanied by a set of metadata which tie it to specific kernel and
 ramdisk images.  Also, the images contain specific configurations and
 startup services which allow the VM to bootstrap cleanly in the
 respective Cloud enviroment, e.g. to obtain the enccessary user
 credentials, and tp perform the wanted firewall setup etc.

 As these systems all use Xen based images, a conversion of these
 images for the different cloud systems should be straight forward.
 The sparse documentation and lack of automatic tools, however, amount
 to a certain challenge to that, at least to the average end user.
 Compared to that, the derivation of customized images frim existing
 images is well documented and tool supported, as long as the target
 image is to be used in the same Cloud system as the original one.

 % add text about gumbo cloud / EPC setup here, if we need / want it

