\documentclass{article}

\usepackage{graphicx}
\usepackage{epsfig}
\usepackage{subfigure}
\usepackage{subfigure}  
\usepackage[usenames,dvipsnames]{color}
\usepackage{srcltx}
\usepackage{url}
\usepackage{ifpdf}
\usepackage[small,it]{caption}

\ifpdf
 \usepackage{hyperref}
%\usepackage[hpdftex]{hyperref}
\else
 \usepackage[hypertex]{hyperref}
\fi

\newenvironment{shortlist}{
  \vspace*{-0.5em}
  \begin{itemize}
  \setlength{\itemsep}{-0.3em}
}{
  \end{itemize}
  \vspace*{-0.5em}
}

\newcommand{\I}[1]{\textit{#1}}
\newcommand{\B}[1]{\textbf{#1}}
\newcommand{\BI}[1]{\textbf{\textit{#1}}}
\newcommand{\T}[1]{\texttt{#1}}

\newcommand{\nind}{\noindent}
\newcommand{\todo}[1]{{\textcolor{red}{\B{TODO:} #1 }}}

\newif\ifdraft
\drafttrue
\ifdraft
 \newcommand{\jhanote}[1]{{\textcolor{red}{     \B{Shantenu:} #1 }}}
 \newcommand{\amnote}[1]{{\textcolor{green}{   \B{AndreM:  } #1 }}}
 \newcommand{\alnote}[1]{{\textcolor{blue}{    \B{AndreL:  } #1 }}}
 \newcommand{\ownote}[1]{{\textcolor{Brown}{   \B{Ole:     } #1 }}}
 \newcommand{\smnote}[1]{{\textcolor{Mulberry}{\B{Sharath: } #1 }}}
 \newcommand{\msnote}[1]{{\textcolor{magenta}{ \B{Mark:    } #1 }}}
 \newcommand{\hknote}[1]{{\textcolor{orange}{  \B{Hartmut: } #1 }}}
\else
 \newcommand{\jhanote}[1]{}
 \newcommand{\amnote}[1]{}
 \newcommand{\alnote}[1]{}
 \newcommand{\ownote}[1]{}
 \newcommand{\smnote}[1]{}
 \newcommand{\msnote}[1]{}
 \newcommand{\hknote}[1]{}
\fi


\usepackage{ifpdf}
\ifpdf
 \DeclareGraphicsExtensions{.pdf, .jpg}
\else
 \DeclareGraphicsExtensions{.eps, .ps}
\fi

\newcommand{\up}{\vspace*{-1em}}

\begin{document}
\begin{center}
{\Large SAGA -- The Next Decade}
\end{center}

{\large
        Shantenu Jha$^{12}$,
        Andre Luckow$^{2}$,
        Sharath Maddineni$^{2}$,\\
        Andre Merzky$^{2}$,
        Mark Santcroos$^{4}$,
        Ole Weidner$^{23}$
        \\[1em]
        %
        $^1$\tiny
         Rutgers University, 
         Piscataway, NJ 08854\\[-0.3em]
        $^2$ \tiny
          Center for Computation and Technology, 
          Louisiana State University\\[-0.3em]
        $^3$ \tiny
          School of Informatics, 
          University of Edinburgh \\[-0.3em]  
        $^4$ \tiny
         Bioinformatics Laboratory, Academic Medical Center, 
         University of Amsterdam\\[-0.3em]
       }

\begin{abstract}
 Evolution of SAGA landscape, and in particular of the SAGA project.\\

 "Never let the future disturb you. You will meet it, if you have to, 
  with the same weapons of reason which today arm you against the present."
  -- Marcus Aurelius Antoninus, 'Meditations'\\
\end{abstract}


\section{Introduction / Scope \todo{AM, SJ}}


Define what we mean by the SAGA Project (i) SAGA - the standard,
incl. relation to other specs, (ii) SAGA the implementation -- engine,
bindings, adaptors, packaging, testing and deployment, (iii) SAGA the
distributed programming and development paradigm (bigjob, DAG-based,
DARE)




\section{Old Introduction}

\alnote{Advert API: - How to reconcile an existing approach as advert
  api with the changing landscape?  Change of landscape in the last
  6-7 years - changed capabilities provided by companies (e.g. AWS,
  Google, Redis, ZeroMQ) - SAGA relationship in comparison to this
  commercial developments - trends in academia and commercial
  environments - original space: - developer of first principal
  applications - vs. framework developers - vs. usage in access layers
  (science gateways) - original space is occupied by infrastructure
  provider - nowadays with MR, Globus Online the level of abstraction
  much higher - infrastructure provider (such XSEDE) should run advert
  service Should one do a message api in place of the availability of
  a commercial solution?  - only very bad apis around current state of
  SAGA - whether Software is funded or not introduction \& scope -
  what is our saga project: api, implementation, framework talking
  about trends?  - trend toward standardization or opposite -
  unambiguous - role of virtualization in production distributed
  infrastructure - for what infrastructure should we harden our
  binding?  - if standards track is not attractive anymore, this has
  consequences to access layer - change in the role of data!  -
  consequence of of data - how should SAGA respond to the
  data-paradigm: - programming abstraction: pilot data - dynamic,
  distributed data }

 "Panta Rhei" -- that phrase is ever so true for human technology.
 The last decade has seen that confirmed for distributed computing as
 for any other field of computing or technology in general.  While it
 is notoriously hard and error prone to predict the future, we dare a
 biased and limited (aka focused) glimpse into the expected evolution
 of our specific area of interest: programming abstractions for
 distributed scientific computing.  We have been trying to enrich and
 support activities in that area for the past decade, most notably by
 defining and implementing the Simple API for Grid Applications
 (SAGA), which spawned an interesting and agile ecosystem of
 distributed programming abstractions.

 The original target scope for SAGA, however, almost disappeared -- we
 have not seen as many 1st principle distributed applications as
 expected, by far.  Instead, we have seen that application frameworks
 on the one (upper) end, and application oriented DCI capabilities on
 the (lower) end, have increasingly and successfully been used by our
 target community over the past years.  At the same time, the hurdles
 for application *users* and framework coders are still significant,
 and many of the complaints voiced by those communities do not differ
 very much from what we heard a decade ago, when they initially
 prompted our Grid Application Toolkit activities in the GridLab
 project.

 With that background, we will try to discuss the roadmap for our SAGA
 based activities, in respect to standardization, implementation, and
 integration.  We will focus our discussion on three different aspects
 of the SAGA landscape: the SAGA standardization 

\subsection{Re-Active versus Pro-Active Evolution \todo{AM}}
\jhanote{this needs to be reduced to one or two sentence at most}
\todo{AM}

The SAGA project historically tried to find a balance between a
reactive and proactive approach: while the initial GAT and SAGA API
efforts have clearly been a reaction to end user demands, the SAGA
evolution has also significantly been driven by the attempt to predict
DCI evolution and cater to expected upcoming application programming
paradigms.  For example, the SAGA Job API is reactively defined, and
provides a rendering of concepts which are well known in distributed
computing.  On the other hand, the SAGA Advert API is not directly
mappable to previously established concepts, but is rather an
expression of functionality which the SAGA project group \I{expected}
to be useful for its target application and user community.

  \todo{How to reconcile pro-active APIs with reality later on?  See
  advert API $->$ redis/nosql; message API $->$ amqp, amazon queues
  etc.}
  
  We expect that the SAGA project will continue to try to find that
  balance, in the hope that the reactive components are sure to cater
  to \I{real} (vs. expected or hyped) technology development, and that
  the proactive components allow to stay close to the state of the
  art, and to avoid to become fossilized.  It must be noted, however,
  that in particular on implementation level, the proactive component
  is often very hard to justify, and chronically suffers from a lack
  of resources.


\section{State}

 \subsection{Implementation}
 - only cpp/python or should we mention JSAGA, javasaga, etc as well?\\
 - describe engine, adaptors (who contributed?), python bindings\\
 - describe deployment 1.6/1.7 \\
 - testing: nmi integration \\
 - big, hard to deploy, CLUNKY (only usable if pre-deployed). in theory: great, in practice: hardly usable. \\
 - in retrospect: c++ probably not the best language choice? \\
 


 \nind
 - previous decade: what is state right now? (extrinsics)\\
 --- state of DCI (software + hardware + provisioning + ...) (\todo{AM, SJ})\\
 
We need a redefinition of application. The "traditional" application that ran on people's desktop is still very present.
Because of scale the usage mode of these applications did change, but it didn't change the application themselves.
They are in most cases still black boxes that take input and create input.
Explicitly not distributed as a primary primitive.

 --- state / overview of user community (\todo{SJ})\\
 
 Growth of users and data volumes is coming from communities that didn't even exists 10 years ago.
 
 
 --- state of SAGA project in relation to/as caused by above (\todo{AM})

The abstraction level as developed by GAT / SAGA ended up not being used by the envisioned users.
Higher level of abstractions were developed to cater for that. Note that the fact that underlying standards exist is still very valuable to create good higher level abstractions.

More importantly, the shift from job centric to data centric e-science is not well represented in SAGA (or in any distributed infrastructure in general).

\subsection{Frameworks}

Various frameworks have been built on top of the SAGA API, e.\,g.\ the BigJob
framework~\cite{saga_bigjob_condor_cloud}, which provides a SAGA-based
Pilot-Job implementation. The Pilot-Job abstraction has been shown to be an
effective abstraction to address many requirements of scientific applications.
Specifically, Pilot-Jobs support the decoupling of workload submission from
resource assignment; this results in a flexible execution model, which in turn
enables the distributed scale-out of applications on multiple and possibly
heterogeneous resources. In contrast to SAGA BigJob, most other Pilot-Job
implementations however, are tied to a specific infrastructure. BigJob has 
been successfully deployed in various scientific applications, e.\,g.\ in 
replica-exchange simulations, multi-physics simulations and genome sequencing.


\section{Vision}

 \subsection{Implementation}
 - focus on usability rather than new features\\
 - focus on Python-bindings + small set of heavily-used adaptors\\
 - spawning of a new light-weight research prototype (outside saga-C++/Python) that allows to quickly evaluate new ideas and support for concepts outside the standard-scope (i.e., support for data-intensive eScience)\\


 \nind
 - SAGA Project Roadmap:\\

 %   foreach (extrinsics)
 %   {
 %     vision of extrinsic evolution $>>>$ influence on our SAGA project
 %   }
 
 \nind

 "The only thing we know about the future is that it is going to be
 different." -- Peter Drucker, 'Management: Tasks,
 Responsibilities, Practices' (1973), Part 1, Chapter 4\\ 

 % instead of prediction, discuss actual trends or
 % options/parameters (role of standards, role of virtualization,
 % role of access layers) 
 % 
 % how should SAGA respond to change in data paradigm

 - evolution of DCI $>>>$ influence on SAGA project\\
 --- evolution of Grids, Clouds (\todo{SJ, AM})\\
 --- evolution of middlewares (\todo{AM})\\
 --- evolution of national / international infrastructures (\todo{SJ})\\

"Three Worlds", the US, Europe and "the rest".
In Europe, in the last decade the model has changed from a centrally managed Grid to a real on national borders distributed Grid.

 
 \nind
 - evolution of communities  $>>>$ influence on SAGA project\\
 --- evolution of applications (\todo{AL, SM, MS})\\
 
Basically no real evolution at all. Although much research has gone into programming paradigms for distributed computing, little of that runs in production.
 
 --- evolution of app domains (\todo{AL, SJ, MS})\\
 
 New science domains, which different kind of characteristic.
 Combination of simulations and data analysis. A lot of progress in science fields that create high volumes of data.
Data, data, data.
More tight coupling to systems from the science domain.
 
 --- evolution of usage modes (\todo{AL, SJ})\\
 
 Research moved to desks. Data analysis has became even more than 10 years ago a primary task of researchers.
 Data analysis in more phases of research cycle, see e-science paradigm.
Semantics.
Provenance and reproducible science.
  
 --- evolution of project contributions (\todo{SM, OW})\\
 
 --- evolution of programming abstractions (level) (\todo{SJ, AL, MS})\\
 
No real uptake of distributed applications, but much progress on glue layers to link legacy applications together.
Implicit parallellisation on data input.

\subsection{P* Abstractions}

\subsection{Abstractions for Data-intensive Applications}



\section{Conclusions -- distributed programming in the next decade}







\footnotesize
\bibliographystyle{plain}
\bibliography{saga,saga-related}

\end{document}

