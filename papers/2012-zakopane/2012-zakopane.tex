\documentclass{article}

\usepackage{graphicx}
\usepackage{epsfig}
\usepackage{subfigure}
\usepackage{subfigure}  
\usepackage[usenames,dvipsnames]{color}
\usepackage{srcltx}
\usepackage{url}
\usepackage{ifpdf}
\usepackage[small,it]{caption}

\ifpdf
 \usepackage{hyperref}
%\usepackage[hpdftex]{hyperref}
\else
 \usepackage[hypertex]{hyperref}
\fi

\newenvironment{shortlist}{
  \vspace*{-0.5em}
  \begin{itemize}
  \setlength{\itemsep}{-0.3em}
}{
  \end{itemize}
  \vspace*{-0.5em}
}

\newcommand{\I}[1]{\textit{#1}}
\newcommand{\B}[1]{\textbf{#1}}
\newcommand{\BI}[1]{\textbf{\textit{#1}}}
\newcommand{\T}[1]{\texttt{#1}}

\newcommand{\nind}{\noindent}
\newcommand{\todo}[1]{{\textcolor{red}{\B{TODO:} #1 }}}

\newif\ifdraft
\drafttrue
\ifdraft
 \newcommand{\sjnote}[1]{{\textcolor{red}{     \B{Shantenu:} #1 }}}
 \newcommand{\amnote}[1]{{\textcolor{green}{   \B{AndreM:  } #1 }}}
 \newcommand{\alnote}[1]{{\textcolor{blue}{    \B{AndreL:  } #1 }}}
 \newcommand{\ownote}[1]{{\textcolor{Brown}{   \B{Ole:     } #1 }}}
 \newcommand{\smnote}[1]{{\textcolor{Mulberry}{\B{Sharath: } #1 }}}
 \newcommand{\msnote}[1]{{\textcolor{magenta}{ \B{Mark:    } #1 }}}
 \newcommand{\hknote}[1]{{\textcolor{orange}{  \B{Hartmut: } #1 }}}
\else
 \newcommand{\sjnote}[1]{}
 \newcommand{\amnote}[1]{}
 \newcommand{\alnote}[1]{}
 \newcommand{\ownote}[1]{}
 \newcommand{\smnote}[1]{}
 \newcommand{\msnote}[1]{}
 \newcommand{\hknote}[1]{}
\fi


\usepackage{ifpdf}
\ifpdf
 \DeclareGraphicsExtensions{.pdf, .jpg}
\else
 \DeclareGraphicsExtensions{.eps, .ps}
\fi

\newcommand{\up}{\vspace*{-1em}}

\begin{document}

\title{\large SAGA -- The Next Decade}

\author{
        Shantenu Jha$^{12}$,
        Andre Luckow$^{2}$,
        Sharath Maddineni$^{2}$,\\
        Andre Merzky$^{2}$,
        Mark Santcross$^{4}$,
        Ole Weidner$^{23}$
        \\[1em]
        %
        $^1$\small
         Rutgers University, 
         Piscataway, NJ 08854\\[-0.3em]
        $^2$ \small
          Center for Computation and Technology, 
          Louisiana State University\\[-0.3em]
        $^3$ \small
          School of Informatics, 
          University of Edinburgh \\[-0.3em]  
        $^4$ \small
         Bioinformatics Laboratory, Academic Medical Center, 
         University of Amsterdam\\[-0.3em]
       }

\maketitle

\begin{abstract}
 Evolution of SAGA landscape, and in particular of the SAGA project.\\

 "Never let the future disturb you. You will meet it, if you have to, 
  with the same weapons of reason which today arm you against the present."
  -- Marcus Aurelius Antoninus, 'Meditations'\\

 "The only thing we know about the future is that it is going to be
 different."
 -- Peter Drucker, 'Management: Tasks, Responsibilities, Practices'
 (1973), Part 1, Chapter 4\\ 
\end{abstract}

\section{Introduction / Scope}

 "Panta Rhei" -- that phrase is ever so true for humann technology.
 The last decade has seen that confirmed for distributed computing as
 for any other field of computing or technology in general.  While it
 is notoriously hard and error prone to predict the future, we dare a
 biased and limited (aka fokused) glimpse into the expected evolution
 of our specific area of interest: programming abstractions for
 distributed scientific computing.  We have been trying to enrich and
 support activities in that area for the past decade, most notably by
 defining and implementing the Simple API for Grid Applications
 (SAGA), which spawned an interesting and agile ecosystem of
 distributed programming abstractions.

 The original target scope for SAGA, however, almost disappeared -- we
 have not seen as many 1st principle distributed applications as
 expected, by far.  Instead, we have seen that application frameworks
 on the one (upper) end, and application oriented DCI capabilities on
 the (lower) end, have increasingly and successfully been used by our
 target community over the past years.  At the same time, the hurdles
 for application *users* and framework coders are still significant,
 and many of the complaints voiced by those communities do not differ
 very much from what we heard a decade ago, when they initially
 prompted our Grid Application Toolkit activities in the GridLab
 project.

 With that background, we will try to discuss the roadmap for our SAGA
 based activities, in respect to standardization, implementation, and
 integration.

  \nind
  - what is scope here for next decade (intrinsics, our SAGA project)\\
  --- standard, incl. relation to other specs (\todo{AM, SJ} )\\
  --- impl - engine, bindings, adaptors, packaging, testing, ... (\todo{OW})\\
  --- higher level - bigjob, portals (\todo{AL, SM, MS})\\
      

 \subsection{Re-Active versus Pro-Active Evolution \todo{AM}}

  The SAGA project historically tried to find a balance between a
  reactive and proactive approach: while the initial GAT and SAGA Grid
  API efforts have clearly been a reaction to end user demands, the
  SAGA evolution has also significantly been driven by the attempt to
  predict DCI evolution and cater to expected upcoming application
  paradigms.  For example, the SAGA job API is reactively defined, and
  provides a rendering of concepts which are well known in distributed
  computing.  On the other hand, the SAGA advert API is not directly
  mappable to previously established concepts, but is rather an
  expression of functionality which the SAGA project group
  \I{expected} to be useful for its target application and user
  community.

  We expect that the SAGA project will continue to try to find that
  balance, in the hope that the reactive components are sure to cater
  to real (vs. expected or hyped) technology development, and that the
  proactive components allow to stay close to the state of the art,
  and to avoid to become fossilized.  It must be noted, however, that
  in particular on implementation level, the proactive component is
  often very hard to justify, and chronically suffers from a lack of
  resources.


\section{State}

    \nind
    - previous decade: what is state right now? (extrinsics)\\
    --- state of DCI (software + hardware + provisioning + ...) (\todo{AM, SJ})\\
    --- state / overview of user community (\todo{SJ})\\


\section{Vision}

    \nind
    - SAGA Project Roadmap:\\

     
    %   foreach (extrinsics)
    %   {
    %     vision of extrinsic evolution $>>>$ influence on our SAGA project
    %   }
  
    \nind
    - evolution of DCI $>>>$ influence on SAGA project\\
    --- evolution of Grids, Clouds (\todo{SJ, AM})\\
    --- evolution of middlewares (\todo{AM})\\
    --- evolution of national / international infrastructures (\todo{SJ})\\
  
    \nind
    - evolution of communities  $>>>$ influence on SAGA project\\
    --- evolution of applications (\todo{AL, SM, MS})\\
    --- evolution of app domains (\todo{AL, SJ, MS})\\
    --- evolution of usage modes (\todo{AL, SJ})\\
    --- evolution of project contributions (\todo{SM, OW})\\
    --- evolution of programming abstractions (level) (\todo{SJ, AL, MS})\\


\section{Conclusions -- distributed programming in the next decade}

\footnotesize
\bibliographystyle{plain}
\bibliography{saga,saga-related}

\end{document}

