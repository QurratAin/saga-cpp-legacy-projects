\documentclass[]{paper}
\usepackage{}

% type user-defined commands here

\begin{document}

\title{FutureGrid 2012 Student Project Challenge} 
\author{Pradeep Kumar Mantha 
  \and Sivakarthik Natesan 
  \and Melissa Romanus 
  \and Sai Saripalli 
  \and Ashley Zebrowski
}
\date{May 15th, 2012}
\maketitle

\begin{abstract}
\end{abstract}

\section{Introduction}
FutureGrid provides students and researchers with new possibilities to
engage in science relating to the state-of-the-art in cloud and grid computing.
As members of the RADICAL group, we have taken full advantage
of the opportunities that FutureGrid provides.  Here are some of the ways
we are using FutureGrid resources to push the envelope and pursue exciting
new discoveries.
\section{Pilot MapReduce}
\section{Replica Exchange}

\section{Cactus Spawner - Ashley Zebrowski}
The Cactus Spawner project envisions application frameworks with simulations that
can be broken down to their constituent components and run across multiple
distributed systems.  A chief consideration is that of ``intelligent computing'',
of knowing when and where to run simulation components separately from the main
simulation.  Work is being done to enable this by modelling simulation components
and predicting the time to run locally vs. the time to transport them and execute
them remotely.  To model real-world problems, the Cactus framework is used, and
actual black hole simulations are executed and spawned from.  Contributions
to the field involve algorithms in modelling, spawning, and performance evaluation on
the I/O systems of FutureGrid hardware.

\section{Bliss Development - Ashley Zebrowski}
SSH adaptor, Eucalyptus adaptor enable cloud interoperability.  Working on 
Bliss itself lowers the barrier of entry to distributed computing for developers
of both Bliss-enabled software and Bliss plugins/extensions.

\section{Conclusion}
Here we will evaluate each application based on how they fit into the FutureGrid proposal
criteria.
\begin{enumerate}
\item Interoperability
\item Scalability
\item Contribution to Education
\item Research (innovation, quality of papers, new software, algorithms, insightful performance measurements, etc.)

\end{enumerate}
\begin{thebibliography}{9}
\end{thebibliography}

\end{document}
