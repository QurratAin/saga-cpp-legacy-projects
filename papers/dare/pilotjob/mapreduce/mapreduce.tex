\documentclass[]{article}

% Use utf-8 encoding for foreign characters
\usepackage[utf8]{inputenc}

% Setup for fullpage use
\usepackage{fullpage}

% Uncomment some of the following if you use the features
%
% Running Headers and footers
%\usepackage{fancyhdr}

% Multipart figures
%\usepackage{subfigure}

% More symbols
%\usepackage{amsmath}
%\usepackage{amssymb}
%\usepackage{latexsym}

% Surround parts of graphics with box
\usepackage{boxedminipage}

% Package for including code in the document
\usepackage{listings}

% If you want to generate a toc for each chapter (use with book)
\usepackage{minitoc}

% This is now the recommended way for checking for PDFLaTeX:
\usepackage{ifpdf}

%\newif\ifpdf
%\ifx\pdfoutput\undefined
%\pdffalse % we are not running PDFLaTeX
%\else
%\pdfoutput=1 % we are running PDFLaTeX
%\pdftrue
%\fi

\ifpdf
\usepackage[pdftex]{graphicx}
\else
\usepackage{graphicx}
\fi
\title{MapReduce}
\author{  }

\date{2011-05-14}

\begin{document}

\ifpdf
\DeclareGraphicsExtensions{.pdf, .jpg, .tif}
\else
\DeclareGraphicsExtensions{.eps, .jpg}
\fi

\maketitle


\begin{abstract}
\end{abstract}

\section{Introduction}


\section{Pilot-Job and Pilot-Store as Runtime Environment for MR}

MapReduce applications can be classified as follows~\cite{weissman2011,ramakrishnan2011}:
\begin{itemize}
    \item Data-intensive applications show a significant higher I/O rate in proportion to the used CPU cycles. Such applications can be well described using Amdahl's number~\cite{gray2000}. Another characteristic particular useful for MapReduce is the data aggregation and flow through the system:
        \begin{itemize}
            \item High aggregation: The output of the MapReduce job is significant smaller than the input.
            \item Zero Aggregation: The MapReduce output is the same as the 
            input.
            \item Ballooning data: The output is larger than the input.
        \end{itemize}
    \item Compute-intensive applications are dominated by CPU cycles. Input 
    files are usually small and thus, I/O is not an issue.
    \item Memory-intensive require that a significant amount of data is load
    into the memory, which is usually the case for computer- and data-intensive 
    applications.
\end{itemize}
Further characteristics of data-intensive applications refer to~\cite{jha2011}.


Scenarios:
\begin{itemize}
	\item 1 BJ 16 core 1 machine
	\item 2 BJ 8 cores - 1 machines
	\item 2 BJ 8 cores - 2 machines
\end{itemize}



Other parameters:
\begin{itemize}
	\item placement of data
	\item cloud
\end{itemize}

\bibliographystyle{plain}
\bibliography{}
\end{document}
