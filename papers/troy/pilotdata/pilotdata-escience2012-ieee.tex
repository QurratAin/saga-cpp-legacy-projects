\documentclass[conference]{IEEEtran}

\usepackage[numbers, sort, compress]{natbib}
\usepackage{graphicx}
\usepackage{amsmath}
\usepackage{amssymb}
\usepackage{color}
\usepackage{ifpdf}

\usepackage{dcolumn}
\usepackage{float}
\usepackage[utf8]{inputenc}
\usepackage{multirow}
\usepackage{rotating}

\usepackage[tight]{subfigure}



%\usepackage[numbers, sort, compress]{natbib}
%\usepackage{latex8}
%\usepackage{float}
%\usepackage{times}    
\usepackage{url}
\usepackage{listings}   
\usepackage{paralist}    
\usepackage{wrapfig}    
%\usepackage[small,it]{caption}
\usepackage{multirow}
\usepackage{ifpdf}
%\usepackage{srcltx}
%\usepackage{subfigure}
\usepackage{xspace}
\usepackage{keyval}  
\usepackage{color}

\definecolor{listinggray}{gray}{0.95}
\definecolor{darkgray}{gray}{0.7}
\definecolor{commentgreen}{rgb}{0, 0.4, 0}
\definecolor{darkblue}{rgb}{0, 0, 0.4}
\definecolor{middleblue}{rgb}{0, 0, 0.7}
\definecolor{darkred}{rgb}{0.4, 0, 0}
\definecolor{brown}{rgb}{0.5, 0.5, 0}

\usepackage[normalem]{ulem}
\makeatletter
\def\cyanuwave{\bgroup \markoverwith{\lower3.5\p@\hbox{\sixly \textcolor{cyan}{\char58}}}\ULon}
\def\reduwave{\bgroup \markoverwith{\lower3.5\p@\hbox{\sixly \textcolor{red}{\char58}}}\ULon}
\def\blueuwave{\bgroup \markoverwith{\lower3.5\p@\hbox{\sixly \textcolor{blue}{\char58}}}\ULon}
\font\sixly=lasy6 % does not re-load if already loaded, so no memory problem.
\makeatother

%!TEX root = sc12/pstar-sc2012-ieee.tex
\newif\ifdraft
%\drafttrue
\ifdraft
\newcommand{\onote}[1]{ {\textcolor{cyan} { (***Ole: #1) }}}
\newcommand{\terminology}[1]{ {\textcolor{red} {(Terminology used: \textbf{#1}) }}}
\newcommand{\owave}[1]{ {\cyanuwave{#1}}}
\newcommand{\jwave}[1]{ {\reduwave{#1}}}
\newcommand{\alwave}[1]{ {\blueuwave{#1}}}
\newcommand{\jhanote}[1]{ {\textcolor{red} { ***shantenu: #1 }}}
\newcommand{\alnote}[1]{ {\textcolor{blue} { ***andreL: #1 }}}
\newcommand{\amnote}[1]{ {\textcolor{blue} { ***andreM: #1 }}}
\newcommand{\smnote}[1]{ {\textcolor{brown} { ***sharath: #1 }}}
\newcommand{\msnote}[1]{ {\textcolor{cyan} { ***mark: #1 }}}
\newcommand{\note}[1]{ {\textcolor{magenta} { ***Note: #1 }}}
\else
\newcommand{\onote}[1]{}
\newcommand{\terminology}[1]{}
\newcommand{\owave}[1]{#1}
\newcommand{\jwave}[1]{#1}
\newcommand{\alnote}[1]{}
\newcommand{\amnote}[1]{}
\newcommand{\athotanote}[1]{}
\newcommand{\smnote}[1]{}
\newcommand{\jhanote}[1]{}
\newcommand{\msnote}[1]{}
\newcommand{\note}[1]{}
\fi

\newcommand{\pilot}{Pilot\xspace}
\newcommand{\pilots}{Pilots\xspace}
\newcommand{\pilotjob}{Pilot-Job\xspace}
\newcommand{\pilotjobs}{Pilot-Jobs\xspace}
\newcommand{\computeunit}{Compute Unit\xspace}
\newcommand{\computeunits}{Compute Units\xspace}
\newcommand{\cu}{CU\xspace}
\newcommand{\cus}{CUs\xspace}
\newcommand{\cs}{Compute Service\xspace}
\newcommand{\css}{Compute Services\xspace}
\newcommand{\pcs}{Pilot Compute Service\xspace}
\newcommand{\dataunit}{Data Unit\xspace}
\newcommand{\dataunits}{Data Unit\xspace}
\newcommand{\du}{DU\xspace}
\newcommand{\dus}{DUs\xspace}
\newcommand{\pilotdata}{Pilot-Data\xspace}
\newcommand{\pd}{PD\xspace}
\newcommand{\pds}{Pilot Data Service\xspace}
\newcommand{\pdss}{Pilot Data Services\xspace}
\newcommand{\su}{SU\xspace}
\newcommand{\sus}{SUs\xspace}
\newcommand{\schedulableunit}{Schedulable Unit\xspace}
\newcommand{\schedulableunits}{Schedulable Units\xspace}
\newcommand{\cc}{c\&c\xspace}
\newcommand{\CC}{C\&C\xspace}

\lstdefinestyle{myListing}{
  frame=single,   
  backgroundcolor=\color{listinggray},  
  %float=t,
  language=C,       
  basicstyle=\ttfamily \footnotesize,
  breakautoindent=true,
  breaklines=true
  tabsize=2,
  captionpos=b,  
  aboveskip=0em,
  belowskip=-2em,
  %numbers=left, 
  %numberstyle=\tiny
}      

\lstdefinestyle{myPythonListing}{
  frame=single,   
  backgroundcolor=\color{listinggray},  
  %float=t,
  language=Python,       
  basicstyle=\ttfamily \footnotesize,
  breakautoindent=true,
  breaklines=true
  tabsize=2,
  captionpos=b,  
  %numbers=left, 
  %numberstyle=\tiny
}

\newcommand{\up}{\vspace*{-1em}}
\newcommand{\upp}{\vspace*{-0.5em}}
\newcommand{\numrep}{8 }
\newcommand{\samplenum}{4 }
\newcommand{\tmax}{$T_{max}$ }
\newcommand{\tc}{$T_{C}$ }
\newcommand{\tcnsp}{$T_{C}$}
\newcommand{\bj}{BigJob}

%  \setlength{\parskip}{0.05ex} % 1ex plus 0.5ex minus 0.2ex}
%  \setlength{\parsep}{0pt}
%  %\setlength{\headsep}{0pt}
%  \setlength{\topskip}{0pt}
%  \setlength{\topmargin}{0pt}
%  %\setlength{\topsep}{0pt}
%  \setlength{\partopsep}{0pt}

% This is now the recommended way for checking for PDFLaTeX:


\ifpdf
\DeclareGraphicsExtensions{.pdf, .jpg, .tif}
\else
\DeclareGraphicsExtensions{.eps, .jpg}
\fi

\tolerance=1000
\hyphenpenalty=10


\begin{document}
%\conferenceinfo{WOODSTOCK}{'97 El Paso, Texas USA}
% \conferenceinfo{ECMLS'11,} {June 8, 2011, San Jose, California, USA.}
% \CopyrightYear{2011}
% \crdata{978-1-4503-0702-4/11/06}
% \clubpenalty=10000
% \widowpenalty = 10000

\title{Pilot-Data: An Abstraction for Managing Distributed Data}

\author{}

\date{}
\maketitle

\begin{abstract} 

Alternate title: ``Pilot-Data: An Abstraction for Distributed Data''




\end{abstract}

\section{Introduction and Overview} 




\section{Pilot-Data Overview}

Talk about concept

\begin{itemize}

\item Similar levels of heterogeneity in the data infrastructure

\begin{itemize}
\item File systems, storage, transport protocols, …
\end{itemize}

\item Support application level capabilities to specify dependencies
  at a logical level rather than specific file level

\begin{itemize}
\item First class support for Affinities (D-C, D-D)
\end{itemize}

\item Typically placement and scheduling of data is decoupled from the compute-tasks

\begin{itemize}
\item Integrated approach to compute and data ?
\end{itemize}

\item Dynamic decision for data

\begin{itemize}
\item Analogous  to late-binding of data
\item Fluctuating resources as a fundamental property of DCI
\end{itemize}

\item Abstraction for other factors and not application specific way
\begin{itemize}
\item Varying data sources, fluctuating data rates, etc
\end{itemize}

\end{itemize}












Reasoning about PilotData

Implementation

Substantiate with description of design decisions

\section{Pilot-API}
Pilot-API as a unifying piece


\section{Experiments}

How do we support affinity?

PD on heterogeneous backends

Distributed systems/data

Similar coordination graph as in the past

"Intelligent" data placements:

- ComputeDataService decides whether data, compute or both have to move

- 3 different cases:

i) move all data everywhere

ii) move data when task has been placed

iii) move data and then place task

Assumption:
How is initial data placed?
- distributed
- central in one place

Second application track:
- Data movement in the context of MD jobs

Which application?
- exclusively bfast?
- bwa? another good application?
- in some cases reference data is decomposable...


\section{Conclusion and Future Work}

\section*{Acknowledgements}
%\up
\footnotesize \footnotesize{This work is funded by NSF CHE-1125332
  (Cyber-enabled Discovery and Innovation), HPCOPS NSF-OCI 0710874
  award, NSF-ExTENCI (OCI-1007115) and NIH Grant Number P20RR016456
  from the NIH National Center For Research Resources. Important
  funding for SAGA has been provided by the UK EPSRC grant number
  GR/D0766171/1 (via OMII-UK) and the Cybertools project (PI Jha)
  NSF/LEQSF (2007-10)-CyberRII-01, NSF EPSCoR Cooperative Agreement
  No. EPS-1003897 with additional support from the Louisiana Board of
  Regents.  SJ acknowledges the e-Science Institute, Edinburgh for
  supporting the research theme. ``Distributed Programming
  Abstractions'' \& 3DPAS. MS is sponsored by the program of BiG Grid,
  the Dutch e-Science Grid, which is financially supported by the
  Netherlands Organisation for Scientific Research, NWO. SJ
  acknowledges useful related discussions with Jon Weissman
  (Minnesota) and Dan Katz (Chicago). We thank J Kim (CCT) for
  assistance with BFAST.  This work has also been made possible thanks
  to computer resources provided by TeraGrid TRAC award TG-MCB090174
  (Jha) and BiG Grid.  This document was developed with support from
  the US NSF under Grant No. 0910812 to Indiana University for
  ``FutureGrid: An Experimental, High-Performance Grid Test-bed''.}

  
\bibliographystyle{IEEEtran}
\bibliography{pilotjob,saga,saga-related}


\end{document}

